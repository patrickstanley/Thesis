% !TEX root = ../mainthesis.tex

%Abstract Page
\pdfbookmark{Abstract}{abstract}
\hbox{\ }

\renewcommand{\baselinestretch}{1}
\small \normalsize

\begin{center}
\large{{ABSTRACT}}

\vspace{3em}

\end{center}
\hspace{-.15in}
\begin{tabular}{ll}
Title of dissertation:    & {\large  CHARACTERIZATION OF MECHANICAL}\\
&				      {\large  PROPERTIES AND DEFECTS OF  } \\
&				      {\large  SOLID-OXIDE FUEL CELL MATERIALS } \\
\ \\
&                          {\large  Patrick Owen Stanley} \\
&                           {\large Doctor of Philosophy, 2018}\\
\ \\
Dissertation directed by: & {\large  Professor Eric D. Wachsman} \\
&  				{\large	 Materials Science and Engineering } \\
\end{tabular}

\vspace{3em}

\renewcommand{\baselinestretch}{2}
\large \normalsize

%350 Word Maximum

\Glspl{sofc} have the potential to help the meet global energy demands by efficiently converting fuel to electricity.
The technology currently requires high temperatures and has reliability limitations.
A critical concern is the structural integrity of the cell after redox cycling at operating temperatures.
As new materials are developed reduce operating temperatures and improve redox stability, the effect of the environment on their mechanical properties must be studied.
Ceria-based systems have allowed the operating temperature to be decreased to the \temp{600} range.
For this reason, a three-point bend apparatus was developed which could test materials at \temp{650} in reducing environments.

Using this apparatus, it was shown how pore shape effected strength of porous \gls{gdc} at \temp{650}.
The strength of Ni-GDC half-cell coupons showed no dependence  on loading orientation at elevated temperatures in air but were 47\% weaker when the electrolyte was placed in tension under H\textsubscript{2}.
It was also determined that a reduced Ni-GDC cell could be exposed to air for an extended period of time and reheated under H\textsubscript{2} with no effect to the strength.

A new anode material, \gls{sfcm}, was investigated for oxygen non-stoichiometry.
\Gls{sfcm} maintains phase purity under reducing conditions, with little changes to lattice parameter between oxidation and reduction.
Under oxidation \gls{sfcm} forms phase impurities.
\Gls{sfcm} supports the formation of a large number of oxygen vacancies with a $\Delta\delta = 0.176$ due to the low enthalpy of formation for oxygen vacancies at \SI{39.1}{\kilo\joule\per\mol}.

Fracture toughness of \gls{sfcm} was determined to be \SI[separate-uncertainty = true]{0.124 +- 0.023}{\mega\pascal\sqrt{m}} in air at room temperature and \SI[separate-uncertainty = true]{0.286 +- 0.038}{\mega\pascal\sqrt{m}} at \temp{600}.
The strength of SFCM-GDC half-cells increased by 31\% upon heating to \temp{600} after which reduction decreased strength by 29\%.
Reduction and redox cycling were shown to only decrease the characteristic strength, not alter the structural flaw distribution.
