% !TEX root = ../mainthesis.tex

%Abstract Page

\hbox{\ }

\renewcommand{\baselinestretch}{1}
\small \normalsize

\begin{center}
\large{{ABSTRACT}}

\vspace{3em}

\end{center}
\hspace{-.15in}
\begin{tabular}{ll}
Title of dissertation:    & {\large  EVALUATION OF STRENGTH AND  }\\
&				      {\large  RELIABILITY OF SOLID-OXIDE FUEL } \\
&				      {\large  CELLS AT OPERATING CONDITIONS} \\
\ \\
&                          {\large  Patrick Stanley, Doctor of Philosophy, 2018} \\
\ \\
Dissertation directed by: & {\large  Professor Eric D. Wachsman} \\
&  				{\large	 Materials Science and Engineering } \\
\end{tabular}

\vspace{3em}

\renewcommand{\baselinestretch}{2}
\large \normalsize

Solid-oxide fuel cells (SOFC) have the potential to help the energy economy transition to a more efficient generation method.
A challenge in SOFCs is that the ceramic must maintain a seal between two gas flows while minimizing the thickness to improve performance.
This seal must hold through heating and changes in environment which change the physical and mechanical properties of the cell.
To better understand these effects, it is desirable to investigate the effects of environment, microstructure, and macrostructure on the cell's strength and modulus.
By obtaining data on the materials and cells at operating conditions, among other intermediate environments, an understanding of the stress-strength relationships in the cell can be obtained.
This new knowledge into the changing mechanical properties will allow for the design and optimization of SOFC devices to maximize reliability and performance.
