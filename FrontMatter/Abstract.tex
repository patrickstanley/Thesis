% !TEX root = ../mainthesis.tex

%Abstract Page

\hbox{\ }

\renewcommand{\baselinestretch}{1}
\small \normalsize

\begin{center}
\large{{ABSTRACT}}

\vspace{3em}

\end{center}
\hspace{-.15in}
\begin{tabular}{ll}
Title of dissertation:    & {\large  CHARACTERIZATION OF MECHANICAL}\\
&				      {\large  PROPERTIES AND DEFECTS OF  } \\
&				      {\large  SOLID-OXIDE FUEL CELL MATERIALS } \\
\ \\
&                          {\large  Patrick Owen Stanley} \\
&                           {\large Doctor of Philosophy, 2018}\\
\ \\
Dissertation directed by: & {\large  Professor Eric D. Wachsman} \\
&  				{\large	 Materials Science and Engineering } \\
\end{tabular}

\vspace{3em}

\renewcommand{\baselinestretch}{2}
\large \normalsize

%350 Word Maximum

Solid-oxide fuel cells (SOFC) have the potential to help the energy economy transition to a more efficient generation method.
A challenge in SOFCs is that the composite ceramic cell must maintain a gas-tight seal between the anode and cathode while minimizing the thickness to improve performance.
This seal must withstand heating and changes in oxygen content, which affect the physical and mechanical properties of the materials.
To understand these effects, it is needed to investigate the effects of environment, microstructure, and macrostructure interplay on the cell’s strength, modulus, and fracture toughness.

The mechanical properties of SOFC materials are difficult to study at operating conditions.
Most research to date have investigated the material properties at ambient conditions after cycling a cell or at elevated temperatures in air.
As part of this work, an enclosed three point bend apparatus has been built which can be heated to the same temperature and environment as an operating SOFC, measuring their properties in situ.

With this technique, among others, a variety of materials and systems have been characterized, including established materials, nickel-oxide and gadolinium doped ceria, to new materials such as a strontium iron cobalt molybdenum oxide.
It has been determined how the choice of pore former effects the strength up to elevated temperatures, how fracture toughness and strength can increase with temperature due to relaxation of intrinsic stresses, the most likely time for failure of an SOFC is under reduction due to uniform growth of microstructure flaws and how cell orientation does not impact mechanical properties.

This new knowledge into the changing mechanical properties of the different materials and structures tested will allow for the design and optimization of SOFC devices to maximize reliability and performance.
In addition, the apparatus built to measure in-situ properties can also be applied to other temperatures and gaseous environments for different applications.
This work helps bring SOFCs from a theoretical lab-based technology to a reliable means of efficient energy generation for use by society.
