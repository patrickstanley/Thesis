% !TEX root = ../mainthesis.tex

\chapter[Defect Chemistry and Oxygen Non-Stoichiometry of \ce{SrFe_{0.2}Co_{0.4}Mo_{0.4}O_{3-\delta}}]{Defect Chemistry and Oxygen Non-Stoichiometry of Double Perovskite \ce{SrFe_{0.2}Co_{0.4}Mo_{0.4}O_{3-\delta}}}

\section{Introduction}
    \gls{sfcm} is a new material that is of particular interest because of its high conductivity and stability in hydrocarbon fuels.\cite{Pan}
    SFCM takes advantage of the perovskite structure and multivalent cations similar to \gls{smm}, but the combination of both Fe and Co further increase the conductivity of the material.
    It has been shown that an SOFC using SFCM as an anode support material is redox stable up to 30 cycles at \temp{600} between H\textsubscript{2}/3\%H\textsubscript{2}O and nitrogen.\cite{Hussaina,Hussain}

    This work expands upon the fundamental knowledge of SFCM and aims to understand the defect chemistry which leads to its high conductivity as an \gls{miec}.
    \Gls{xrd} and Rietveld refinement show the phase stability and lattice parameter changes of SFCM across \glspl{po2}.
    \Gls{dc} conductivity is measured as a function of \po2{} to determine changes electrical conductivity types.
    \Gls{tpd} spectroscopy was used to characterize the oxygen desorption as it occurs from the lattice.
    The non-stoichiometry of \gls{sfcm} was measured under oxidizing and reducing environments at various temperatures via thermogravimetric analysis and a defect equilibrium model and diagram is proposed from the data with results compared to \gls{smm} and other perovskite materials.

\section{Results and Discussion}
    \subsection{Phase Purity and Chemical Expansion}
    The \gls{xrd} patterns obtained after reduction in 1.9\% H\textsubscript{2}, 2.9\% H\textsubscript{2}O at \temp{600} and oxidation in pure oxygen at \temp{600}, in addition to the fresh as synthesized powder and the pattern obtained from Rietveld refinement, are presented in Figure \ref{fig:structure}a.
    A tetragonal, unordered B-site, double perovskite structure was used as a starting point for Rietveld refinement based on the structures of \ce{Sr_2CoMoO_6}, \ce{Sr_2NiMoO_6} and \ce{Sr_2Fe_{0.75}Co_{0.25}MoO_6}.\cite{Huang2009,Ritter2004}
    Results from the Rietveld refinement calculations are given in Table \ref{tab:xrdrefine}.
    The refined unit cell is presented in Figure \ref{fig:structure}b and was used to create the theoretical \gls{xrd} pattern for comparison.

    \begin{figure}[p]
      \includegraphics[width=0.6\textwidth]{XRD.jpg}
      \includegraphics[width=0.6\textwidth]{Structure.png}
      \caption{a) Powder \gls{xrd} patterns of SFCM samples taken after synthesis, reduction, and oxidation compared to pure phase SFCM diffraction pattern b) Crystal structure of unordered double perovskite SFCM}
      \label{fig:structure}
    \end{figure}

    \begin{table}
        \centering
        \caption{Refinement results of SFCM before and after exposure to reducing and oxidizing environments}
        \label{tab:xrdrefine}
        \begin{tabular}{llll}
        Condition & Oxidized & Fresh   & Reduced  \\
        \hline
        Space Group                & I 4$\overline{\text{m}}$    & I 4$\overline{\text{m}}$   & I 4$\overline{\text{m}}$    \\
        a (\si{\angstrom})        & 5.5636(2)   & 5.5558(3)  & 5.5544(2)   \\
        c (\si{\angstrom})        & 7.9310(2)   & 7.9184(3)  & 7.9204(2)   \\
        Volume (\si{\angstrom\cubed}) & 245.49  & 244.41 & 244.35  \\
        Sr-O1 (\si{\angstrom})    & 2.794  & 2.790 & 2.790   \\
        Sr-O2 (\si{\angstrom})    & 2.789  & 2.785 & 2.784   \\
        M1-O1 (\si{\angstrom})    & 1.993  & 1.990 & 1.990   \\
        M1-O2 (\si{\angstrom})    & 1.782  & 1.780 & 1.780   \\
        M2-O1 (\si{\angstrom})    & 1.950  & 1.948 & 1.947   \\
        M2-O2 (\si{\angstrom})    & 2.183  & 2.180 & 2.180   \\
        Phase Purity (\% SFCM)    & 85.24  & 76.08 & 100.0   \\
        R\textsubscript{w} (\%)   & 14.277 & 15.519 & 17.324
        \end{tabular}
        \end{table}

    SFCM proved to be stable under reducing conditions, being pure phase based on the Rietveld refinement.
    Overall phase purity remains at greater than 76\% as synthesized, 85\% under oxidizing conditions and at 100\% after reduction.
    \ce{Sr_2Co_{1.2}Mo_{0.8}O_6} was found in SFCM under ambient and oxidizing conditions.
    This secondary phase reversibly disappears when returning to reducing conditions after oxidation.

    Changes in SFCM's lattice parameters are attributed to the oxidation states of the B-site cations and the concentration of vacancies in the sample after exposure to different \po2{}s.
    Reduction of SFCM from an as synthesized state causes a 0.025\% decrease in $a$ and the volume of the unit cell and a 0.025\% increase in $c$.
    These are a result of the bond lengths between strontium and oxygen in site 2 (Sr-O2) and the distance between metal site 2 and oxygen site 1 (M2-O1) decreasing. %don't like the way I did this
    %These are result of the bond lengths between Sr-O2 and M2-O1 decreasing, where is the species at a unique site. %don't like the way I did this
    Oxidizing the SFCM results in a 0.14\% increase in $a$ and a 0.16\% increase in $c$ resulting an 0.44\% increase in cell volume.
    This is the result of all bond lengths increasing by \SI{0.002} to \SI{0.004}{\angstrom}.
    These small lattice parameter shifts are in line with the general nature of perovskites to display little to no change due to chemical expansion.\cite{Bishop2014}
    Additionally, when changes do occur, perovskites have been shown to have anisotropic chemical expansion, as such with \gls{sfm} where $a$ increases upon reduction while $c$ decreases.\cite{Tsvetkov2016}
    In summary, SFCM shows very little change to lattice parameters as a result of changing metal oxidation states or vacancy creation as it under goes reduction or oxidation.

    \subsection{Conductivity}
    SFCM has a conductivity dependence typical of MIEC conductors, as shown in Figure \ref{fig:conductivity}.
    At high \po2{} ranges (\SI{e-1.5}{atm} to \SI{1}{atm}) it shows p-type conductivity, where conductivity increases with \po2{}.
    Down to \SI{e-1}{atm}, the change in conductivity appears to be linear with a slope of \SI{0.089} (log-log scale).
    Below \SI{e-1}{atm} the slope changes away from the previous trend, suggesting a change in the predominant charge carrier or phase of the system.
    At low \po2{} ranges (\SI{e-24}{atm} to \SI{e-19}{atm}) the conductivity behaves linearly with n-type conductivity, increasing at lower \po2{}, with a slope of \SI{-0.11} (log-log scale).
    The conductivity measured in this work is comparable to the \SI{30}{S\per\centi\meter} at \temp{600} as reported by Pan et. al.\cite{Pan}
    %Due to the inability to measure the intermediate \po2{} we were unable to determine in this work at what point the \po2{} dependence changes from n-type to p-type and if there is an ionic regime in between.

    \begin{figure}[p]
      \includegraphics[width=0.6\textwidth]{conductivity.jpg}
      \caption{Total conductivity as \po2{} changes under a) reducing and b)oxidizing conditions at \SI{600}{\celsius}}
      \label{fig:conductivity}
    \end{figure}

    Based on the slopes in the high \po2{} region, at least two defect regimes exist in that area transitioning near \SI{e-1}{atm}.
    Both high and low \po2{} regions have slopes much lower than +1/4 or -1/4 respectively, which indicates that the relationship between electronic charge carriers and oxygen is dependent of the other defect species in the material.
    In fact, the lower than 1/4 slope suggest that the electronic charge carries are taken up by the other species, decreasing the expected number based on \po2{} change.
    At low \po2{}, n-type conductivity is promoted by the generation of oxygen vacancies, freeing electrons as negative electronic charge carriers to maintain charge balance.
    Some of these electrons are consumed by the reduction of B-site metals to lower oxidation states.
    At high \po2{}, oxygen vacancies are filled, consuming electrons and allowing for the creation of holes, promoting p-type conductivity.
    The different slopes between high and low \po2{} are the result of different defect species interacting with the electronic carriers at different points in the reduction.

    \subsection{Temperature Programmed Desorption}
    \Gls{tpd} was performed in conjunction with \gls{tga} to determine the extent of oxygen desorption from \gls{sfcm}.
    Figure \ref{fig:TPD} presents the data from the \gls{tga} on top with the mass and rate of mass change, while the \SI{32}{m\per z} signal from oxygen \gls{tpd} in the mass spectrometer is on bottom.
    The cyclic noise present in both mass and \gls{ms} signals is due to transients of the PID controller on the TGA furnace, but the noise remained much lower than the measured signal.
    The rate of mass loss matches the oxygen signal from the \gls{ms}, confirming that oxygen is generated from \gls{sfcm} when heated under N\textsubscript{2} and is the cause for mass loss in the sample.
    \gls{sfcm} shows two maxima for the rate of oxygen loss during heating.
    The first, \textalpha, occurs at \temp{405} with the second peak, \textbeta, occurring near \temp{800}.
    The \gls{ms} monitored for other species, such as carbon (\ce{CO}, \ce{CO_2}) and water, but no significant amounts were observed.

    \begin{figure}[p]
      \includegraphics[width=\textwidth]{TPD.jpg}
      \caption{Temperature programmed desorption of oxygen in SFCM with mass loss from TGA(top) and oxygen desorption from MS (bottom) as it is heated to \SI{800}{\celsius} in N\textsubscript{2}}
      \label{fig:TPD}
    \end{figure}

    Perovskite materials have oxygen desorption classified based on the temperature at which it occurs.
    Low temperature desorption (referred to as \textalpha{}), occurs due to the reduction of metals and desorption of oxygen near the surface.
    High temperature desorption (referred to as \textbeta{}), takes place when oxygen is able to diffuse through the bulk and be released.\cite{Levasseur2009}
    SFCM shows both \textalpha{} and \textbeta{} desorption, indicating that both the surface and lattice can generate oxygen vacancies due to the reduction of the different B-site cations and that those vacancies are mobile to move from the bulk to the surface.
    \gls{smm} shows only \textbeta{} desorption above \temp{800}, while \gls{sfm} has \textalpha{} desorption from \temp{400} to \temp{800} and \gls{scm} desorbs starting below \temp{400}.\cite{Liu2011, Vasala2010}
    In the case of \gls{sfm} and \gls{scm}, the use of the multivalent Fe or Co allows for reduction at a lower temperature than \gls{smm}.
    For SFCM, the same effect is seen through the combined use of Fe and Co.
    This demonstrates SFCM's ability to begin to create oxygen vacancies at \temp{400} and exchange oxygen with the bulk lattice starting as low as \temp{450} and peaking near \temp{800}.

    \subsection{Oxygen Non-Stoichiometry}
    Oxygen non-stoichiometry and oxygen vacancy concentrations are shown in Figure \ref{fig:TGA600} at \temp{600} for high and low \po2{} regions.
    At high \po2{}, SFCM has two linear regions, changing between them near \SI{e-0.75}{atm}.
    The transition between regions in oxygen stoichiometry occurs at a similar \po2{} to where the conductivity changes in Figure \ref{fig:conductivity}.
    This change is the result of a change in defect regime regions.
    In the low \po2{} region, a linear trend occurs down to \SI{e-21}{atm} where the non-stoichiometry rate increases.
    Due to the pure phase XRD under reducing conditions, it is not likely that the increase in non-stoichiometry is due to phase decomposition and instead is a change in regimes where more oxygen vacancies are present.

    \begin{figure}[p]
      \includegraphics[width=3.25in]{TGA600-2.jpg}
      \caption{Non-stoichiometry (circles, left axis) and corresponding oxygen vacancy concentration (diamonds, right axis) of SFCM under a) reducing conditions and b) oxidizing conditions at \SI{600}{\celsius}. Error bars are smaller than points for non-stoichiometry and for vacancy concentration below \SI{e-0.5}{atm}.}
      \label{fig:TGA600}
    \end{figure}

    SFCM shows similar trends in non-stoichiometry to \gls{smm} and \gls{sfm}, but with the added complexity expected from the additional B-site species.
    \gls{smm} approaches a plateau in oxygen content as it reached \SI{1}{atm} of oxygen the same as SFCM but possesses a constant slope of -1/6 when plotted on a log-log scale.\cite{Marrero-lopez2010}
    In comparison, SFCM has multiple slopes associated with the degree of vacancy formation, as shown in Figure \ref{fig:TGA600}b.
    These different slopes can be attributed to the reduction of the three different B-site cations at different \po2{}.
    As an oxygen vacancy forms, the electrons required to maintain charge balance can be accommodated by the reduction of Fe, Co, or Mo, each with an independent equilibrium constant, whereas \gls{smm} only has Mo reduction to accommodate electrons from oxygen vacancy generation.
    The high slopes of Figure \ref{fig:TGA600}b show how SFCM can generate more oxygen vacancies at lower \po2{} because of the Fe and Co cations.
    SFM and SMM show similar amounts of non-stoichiometry at the same \po2{} but at \temp{1000}.\cite{Kircheisen2012}
    SFCM shows almost twice the non-stoichiometry at a much lower temperature of \temp{600}.
    SFCM's ability to create vacancies at higher \po2{} results in a larger accumulated vacancy concentration at low \po2{}.

    TGA non-stoichiometry measurements were performed on the same sample at \temp{400},  \temp{500}, and \temp{600} which are temperatures in the low temperature-SOFC range.
    Figure \ref{fig:TGAtemps} shows the non-stoichiometry measurements for the sample  tested at three temperatures in the low \po2{} (a) and high \po2{} (b) regions.
    As expected, lowering the temperature reduces the non-stoichiometry at a given \po2{} and decreases the \po2{} at which transitions between regimes occur.
    At \temp{400} oxygen vacancy formation decreases considerably compared to \temp{500} and \temp{600}
    This aligns with the results of the TPD (Figure \ref{fig:TPD}) where SFCM does not start desorbing oxygen until \temp{350} to \temp{400}.

    \begin{figure}[p]
      \includegraphics[width=0.6\textwidth]{TGAtemps.jpg}
      \caption{Non-stoichiometry of SFCM as \po2{} changes at \SI{400}{\celsius}, \SI{500}{\celsius}, and \SI{600}{\celsius}. Error bars are smaller than data points.}
      \label{fig:TGAtemps}
    \end{figure}

    \subsection{Defect Equilibrium Model and Diagram}
    To best understand what is occurring in SFCM a defect equilibrium diagram and model need to be created to relate the changes in conductivity and non-stoichiometry to the concentration of the various defects present in the material.
    Defect reactions with their corresponding equilibrium equations are given in Equations \ref{rxn:intrinsic}{--}\ref{rxn:metalred}, which use Kr\"oger–Vink notation, where a species is denoted by the site it sits on (subscript) and the relative charge to the site's expected valency (superscript).
    For example, $\ch{V_O^{**}}$ is a vacancy in an oxygen site with a 2+ charge and $\ch{B_B^'}$ is a B-site metal in a B-site with a 1$-$ charge.
    K represents the equilibrium, or mass-action, constant for the reaction.
    To simplify the set of equations, Equation \ref{rxn:metalred} represents the combined reduction of any B-site cation (Fe, Co, or Mo).
    \textbeta{} is the ratio of A-site vacancies to B-site vacancies from Schottky defects.
    \begin{align}
        \label{rxn:intrinsic}
        \emptyset& \ch{<-> e' + h^{*}}&   K_i&  = np\\
        \label{rxn:schottky}
        \emptyset& \ch{<->  3 V_O^{**} + V_Sr^{''} + V_B^{''''}}& K_s& = \lbrack\ch{V_O^{**}}\rbrack^3 \lbrack\ch{V_Sr^{''}}\rbrack \lbrack\ch{V_B^{''''}}\rbrack = \lbrack\ch{V_O^{**}}\rbrack^3 \lbrack\ch{V_B^{''''}}\rbrack ^{\beta + 1}\\
        \label{rxn:external}
        \ch{O_O^x}& \ch{<-> 1/2 O2 + V_O^{**} + 2 e'}& K_r& = p_{O_2}^{1/2} \lbrack\ch{V_O^{**}}\rbrack \lbrack\ch{O_O^{x}}\rbrack^{-1} n^2\\
        \label{rxn:metalred}
        \ch{B_B^x + e^'}& \ch{<-> B_B^'}& K_{B}& = \lbrack\ch{B_B^{'}}\rbrack n^{-1}
    \end{align}

    The Duncan-Wachsman approach allows for the solution of defect equations across \po2{} regions using dominant defect triads, compared to the dominant defect pairs of the Brouwer approach which lead to an inaccurate results near transition regions.\cite{Duncan2007}
    This approach is needed in this work to model the defect equilibria because the measured \po2{} range crosses various Brouwer regions, as indicated by the non-linearity of Figure \ref{fig:TGA600}.
    Using this method, the defect equations can be solved to give models of the concentration across \po2s.
    The equations used are taken directly from Duncan et. al paper.
    To apply this approach to SFCM, the interactions of electronic defects with B-site cations as given by Equation \ref{rxn:metalred} were ignored for the time being.
    It was assumed that all electrons generated from vacancy formation remained in the conduction band.
    It was also assumed that a portion of B-site cations were reduced at all conditions, acting as a constant dopant to the structure, where regardless of \po2{}.
    The degree of constant B-site reduction could then be fit from the model.

    % \begin{table}
    % \centering
    % \caption{Equations for defect equilibrium under reducing conditions}
    % \label{tab:defectequ}
    % \begin{tabular}{c|c}
    % Defect & Equation\\
    % \hline
    % n  & $K_r^{\frac{1}{2}} p_{O_2}^{-\frac{1}{4}}\left(\frac{3}{4} K_r^{\frac{1}{2}}p_{O_2}^{-\frac{1}{4}}+\left(\frac{1}{2}\lbrack\ch{B_B^'}\rbrack\right)^{\frac{3}{2}}\right)^{-\frac{1}{3}}$  \\[10pt]
    %
    % p  & $K_iK_r^{-\frac{1}{2}} p_{O_2}^{\frac{1}{4}}\left(\frac{3}{4} K_r^{\frac{1}{2}}p_{O_2}^{-\frac{1}{4}}+\left(\frac{1}{2}\lbrack\ch{B_B^'}\rbrack\right)^{\frac{3}{2}}\right)^{\frac{1}{3}}$  \\[10pt]
    %
    % $\lbrack\ch{V_O^{**}}\rbrack$   & $\left(\frac{3}{4} K_r^{\frac{1}{2}}p_{O_2}^{-\frac{1}{4}}+\left(\frac{1}{2}\lbrack\ch{B_B^'}\rbrack\right)^{\frac{3}{2}}\right)^{\frac{2}{3}}$\\[10pt]
    %
    % $\lbrack\ch{V_A^{''}}\rbrack$   & $\beta{}K_s^{\frac{1}{\beta+1}}\left(\frac{3}{4} K_r^{\frac{1}{2}}p_{O_2}^{-\frac{1}{4}}+\left(\frac{1}{2}\lbrack\ch{B_B^'}\rbrack\right)^{\frac{3}{2}}\right)^{-\frac{2}{\beta+1}}$ \\[10pt]
    %
    % $\lbrack\ch{V_B^{''''}}\rbrack$ & $K_s^{\frac{1}{\beta+1}}\left(\frac{3}{4} K_r^{\frac{1}{2}}p_{O_2}^{-\frac{1}{4}}+\left(\frac{1}{2}\lbrack\ch{B_B^'}\rbrack\right)^{\frac{3}{2}}\right)^{-\frac{2}{\beta+1}}$ \\[10pt]
    %
    % $\lbrack\ch{B_B^'}\rbrack$ & Constant \\
    %
    % \end{tabular}
    % \end{table}

    Using the TGA and conductivity data, the reaction equilibrium constants can be fitted for the defect model.
    The results of the fit are shown in Figure \ref{fig:fitting}.
    Starting with the oxygen vacancies least squares fitting approach, the parameters are fitted in the low \po2{} region first, adding terms and parameters from the Duncan-Wachsman method until the entire equation had been fit.
    A similar process was followed to fit the conductivity data to the model.
    Separate equations were used to fit n-type and p-type conductivities, based on their respective predominant defects.

    \begin{figure}[p]
      \includegraphics[width=3in]{Fitting.jpg}
      \caption[a) Fitting results of oxygen vacancies concentration to the measured non-stoichiometry b) Fitting results of n-type and p-type conductivity to measurements]{a) Fitting results of oxygen vacancies to the measured non-stoichiometry with the parameters $K_{r}=\SI{1.1e-15}{atm^{0.5}\per{F.U.}^2}$, $K_{i}=\SI{5.43e-7}{\per{F.U.}^2}$, $K_{s}=\SI{1e-25}{\per{F.U.}^5}$, $\lbrack\ch{B_B^'}\rbrack = \SI{0.108}{\per{F.U.}}$, $\beta = 100$. b) Fitting results of n-type and p-type conductivity to measurements based on previous parameters, $\mu_e=\SI{1.43}{\centi\meter\per\volt\per\second}$, $\mu_{\ch{V_O}}=\SI{0.4887}{\centi\meter\per\volt\per\second}$, and $\mu_h=\SI{0.0806}{\centi\meter\per\volt\per\second}$.}
      \label{fig:fitting}
    \end{figure}

    The defect equilibria values were fitted and calculated from thermogravimetry and found to be to be \SI{1.1e-15}{atm^{0.5}\per{F.U.}^2} for $K_{r}$, \SI{5.43e-7}{\per{F.U.}^2} for  $K_{i}$, and \SI{1e-25}{\per{F.U.}^5} for $K_{s}$, where F.U. is the formula unit of SFCM.
    The minimum concentration of \ch{B_B^'} defects (\ch{M_B^'}) was determined to be \SI{0.108}{\per{F.U.}} and \textbeta{} was 100.
    A deviation from the model is observed at oxidizing conditions and can be attributed to the phase impurity which forms at those conditions.
    The parameters relating to the Schottky defects (\textbeta{} and K\textsubscript{s}) did not have a significant impact on the fit of either the oxygen vacancies conductivity, and thus are not considered to be well fit.
    This is because Schottky defects are not expected in high concentration except at high temperatures and very high \po2s, neither of which were tested in this work.
    The defect mobilities were calculated from the fitted data at \temp{600} to be \SI{1.432}{\centi\meter\squared\per\volt\per\second} for electrons \SI{0.4887}{\centi\meter\per\volt\per\second} for oxygen vacancies and \SI{0.0806}{\centi\meter\per\volt\per\second} for holes.
    A deviation in slope is again visible from the model for the conductivity data at oxidizing conditions, which can be attributed to the impurity phase.
    This difference in slope adds uncertainty for the value of the hole mobility.
    The electron mobility is significantly greater than the value reported for small polaron defect mobility in SMM.\cite{Marrero-lopez2010}

    The complete defect equilibrium diagram for electronic defects and vacancies based on the fitted values is shown in Figure \ref{fig:ded}a.
    The model predicts an electrolytic domain in the mid-\po2{} region.
    In comparison to the traditional Brouwer regions, the measured \po2s only cover the boundary between Region I and IIa up to the lower portion of Region IIb.

    \begin{figure}[p]
      \includegraphics[width=3in]{DEDresults.jpg}
      \caption{a)Defect equilibrium diagram of SFCM based on fitted parameters. b)Plot of natural log of external defect equilibrium versus inverse temperature, to calculate enthalpy of formation oxygen vacancies.}
      \label{fig:ded}
    \end{figure}

    To determine the enthalpy of oxygen formation, the natural log of the equlibrium constant for the external reaction (Equation \ref{rxn:external}) was plotted against the reciprocal temperature.
    The enthalpy was found to be \SI{44.3}{\kilo\joule\per\mol} in SFCM under reducing conditions.
    SFCM has an enthalpy of \ch{V_O^{**}} formation lower than that of comparable materials.
    SFM and SMM have much higher enthalpies of \SI{253}{\kilo\joule\per\mol} and \SI{110}{\kilo\joule\per\mol} respectively.\cite{Kircheisen2012,Marrero-lopez2010}
    \ce{SrFeO_{$3-\delta$}} has a more comparable, but still greater, enthalpy of \SI{80}{\kilo\joule\per\mol}.\cite{Holt1999}
    This low enthalpy of formation of oxygen vacancies in SFCM allows for a greater range of non-stoichiometry and thus SFCM's higher performance.


\section{Conclusions}
    SFCM has potential as an electrode material in a number of electrochemical device applications.
    Specifically, it has a potential to replace nickel-based anodes in SOFCs due to its high conductivity and redox stability, creating an all-ceramic anode.
    SFCM forms impurity phases when heated at \temp{600} under oxidizing environments but becomes pure phase with reduction.
    It also exhibits very small chemical expansion with changing \po2{}, in line with its perovskite structure.
    SFCM is an improvement over previous double perovskite materials because conductivities above \SI{40}{S\per\centi\meter} in reducing conditions below \temp{600}.
    This is in part because SFCM supports the desorption of oxygen from both surface and lattice positions shown by overlapping \textalpha{} and \textbeta{} oxygen desorption, starting as low at \temp{350} and continuing up until above \temp{800}.
    Thermogravimetry at lower temperatures demonstrated SFCM's activity down to \temp{400} at which the amount of oxygen vacancy formation slows dramatically.
    Oxygen vacancies form more rapidly than other MIEC materials with decreasing \po2{} as combination of B-site cations enables reduction though a large range of \po2{} resulting in a low enthalpy of formation for oxygen vacancies at  \SI{44.3}{\kilo\joule\per\mol}.
    Defect equilibrium and mobility values were calculated from the themogravimetric and conductivity measurements and a defect equilibrium diagram for SFCM at \temp{600} was created.
