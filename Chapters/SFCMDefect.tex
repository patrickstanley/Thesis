% !TEX root = ../mainthesis.tex

\chapter[Defect chemistry and oxygen non-stoichiometry of \ce{SrFe_{0.2}Co_{0.4}Mo_{0.4}O_{3-\delta}}]{Defect chemistry and oxygen non-stoichiometry of double perovskite \ce{SrFe_{0.2}Co_{0.4}Mo_{0.4}O_{3-\delta}}}

\section{Introduction}
    Solid-state electrochemical devices that perform energy conversion and storage rely on a material's abilities to transport charged carriers through the material.
    Devices such as solid-oxide fuel cells (SOFCs), oxygen separation membranes, and gas sensors can utilize materials which conduct both oxygen and electronic conductors, known as mixed ionic electronic conductors (MIEC), in the electrodes where both conductions need to occur simultaneously.\cite{Huang2006}
    SOFCs in particular are limited in their application due to high operating temperatures, performance degradation due to fuel contamination, and inability to tolerate thermal and redox cycling.
    To improve the performance and reliability of SOFCs the operating temperature needs to be lowered from the intermediate temperature range (\SI{650}{\celsius} to \SI{800}{\celsius}) to the low temperature range (\textless\SI{650}{\celsius}).\cite{Wachsman2011a}
    Additionally, the use of an all-ceramic anode, as opposed to the traditional nickel metal anode, would improve long term performance, resistance to poisoning or coking, and better match the thermal expansion of the rest of the cell.\cite{Goodenough2007}

    Perovskite structures have yielded number of MIECs with potential uses as electrode materials. \cite{Yamamoto1987,Anderson1992,Ishihara2009}
    \ce{Sr_2MgMoO_6} (SMM) and its related family of materials (\ce{Sr_2MMoO_6}, where M is a transition metal dopant) yield good conductivities and catalytic actives.\cite{Huang2009}
    The mixed valence state of Mo(VI)/Mo(V) provides high electronic conduction while supporting oxygen vacancy formation.\cite{Huang2006a}
    If a dopant is used which has an overlapping redox couple band, such as Fe, electronic conduction can be further improved.
    Mo(VI) and Mo(V) are stable in both octahedral and tetrahedral coordination, further adding stability to oxygen vacancy formation.\cite{Bernuy-Lopez2007}

    A new material, \ce{SrFe_{0.2}Co_{0.4}Mo_{0.4}O_{3-\delta}} (SFCM), is of particular interest because of its high conductivity and stability in hydrocarbon fuels.\cite{Pan}
    SFCM takes advantage of the perovskite structure and multivalent cations similar to SMM, but the combination of both Fe and Co further increase the performance of the material.
    It has been shown that an SOFC using SFCM as an anode support material which had been infiltrated with nickel-gadolinium doped ceria nanoparticles is redox stable up to 30 cycles at \temp{600} and that it provided long term catalytic activity without detriment to performance.\cite{Hussaina,Hussain}

    This work expands upon the fundamental knowledge of SFCM and aims to understand the defect chemistry and electronic structure which leads to its high conductivity.
    X-ray diffraction and Rietveld refinement show the phase stability and lattice parameter changes of SFCM across partial pressures of oxygen (\po2{}).
    Electrical conductivity is measured as a function of \po2{} to determine changes electrical conductivity types.
    Temperature programmed desorption spectroscopy was used to characterize the oxygen desorption as it occurs from the lattice.
    The non-stoichiometry of SFCM was measured under oxidizing and reducing environments at various temperatures via thermogravimetric analysis and a defect equilibrium model and diagram is proposed from the data with results compared to SMM and other perovskite materials.

\section{Experimental}
    \subsection{Sample Preparation}
        SFCM was created from stoichiometric amounts of strontium carbonate (\ce{SrCO_3}, Sigma-Aldrich), iron oxide (\ce{Fe_2O_3}, Sigma-Aldrich), cobalt oxide (\ce{Co_2O_3}, Inframat Advanced Materials), and molybdenum oxide (\ce{MoO_3}, Alfa-Aesar) using conventional solid-state methods.
        The constituents were ball milled for 24 hours in ethanol and dried using a \SI{100}{\celsius} oven.
        Afterwards the powder was calcined at \SI{1100}{\celsius} for four hours.

        Dense SFCM bars were used to maximize the total mass and mass changes during thermogravimetric testing.
        Samples were made by combining SFCM powder with 0.6 wt\% polyethylene glycol 600, 1.8 wt\% ethylene glycol, and 0.6 wt\% glycerol in isopropyl alcohol and ball milling overnight.
        After drying at \SI{100}{\celsius}, the powder was ground by mortar and pestle, then pressed uniaxially into rectangular bars at \SI{30}{\mega\pascal}, followed by isostatic pressing at \SI{30}{\mega\pascal}.
        Bars were sintered by heating to \SI{400}{\celsius} for one hour and then \SI{1340}{\celsius} for four hours, using a \SI{3}{\celsius\per\minute} heating and cooling rate.
        This produced bars with 97\% theoretical density by Archimedes' technique.

    \subsection{X-ray Diffraction}
        X-ray diffraction (XRD) was used confirm phase purity of the SFCM after synthesis and during the testing process.
        A Bruker D8 Advance with LynxEye was used with a Cu K\textsubscript{\textalpha{}} source.
        A step size of \SI{0.02}{\degree} was used with a dwell of \SI{0.8}{\second} was used.
        Rietveld refinement was performed on powder samples as synthesized, after oxidation in pure \ce{O_2} and after reduction in pure \ce{H_2} at \temp{600} for both.
        GSAS-II was used to perform the refinement calculations and VESTA was used to visualize the unit cell.\cite{Toby2013,Momma2011}

    \subsection{Conductivity}
        Electrical conductivity was measured using the four-wire technique and a Stanford SR 830 lock-in amplifier.
        A bar shaped sample with the dimensions of \SI{6.46x3.3x1.3}{\milli\meter} was used.
        Gold paste was used as a current collector, and the current range was between \SI{0.005} to \SI{0.05}{A}.
        A yttria-stabilized zirconia (YSZ) oxygen sensor operating at \temp{800} was used to monitor the changes in oxygen partial pressures.
        Intermediate \po2{} ranges were not tested due to instability between SFCM and the \ce{CO} and \ce{CO_2} required to obtain those \po2{}.

    \subsection{Thermogravimetric Analysis (TGA)}
        Changes in mass of SFCM were measured by a Cahn D200 microbalance with the sample suspended down a quartz tube into a furnace.
        The furnace used to heat the sample and quartz tube was controlled by a PID controller with a K-type thermocouple placed immediately below the sample inside the quartz tube.
        Gas flow was controlled at consistent \SI{50}{sccm} by mass flow controllers, which mixed dry nitrogen, oxygen, hydrogen, and humidified nitrogen to obtain the \po2{} desired.
        Measurements of the \po2{} were taken by a calibrated YSZ oxygen sensor at \temp{800} located before the sample.
        Again, intermediate \po2{} ranges were not able to be tested due to SFCM's incompatibility with the species required to create that environment.

        To prepare the sample, it was pre-weighed, wrapped in platinum wire and suspended from the balance, placed in the furnace with simulated air (21\% O\textsubscript{2}, 79\% N\textsubscript{2}) flowing.
        Once the mass had stabilized, the furnace was heated to \SI{800}{\celsius} to allow for any organic contamination to burn off.
        After the mass stabilized at this elevated temperature the sample was introduced to various environments.
        The mass of the sample would be noted only after steady state had been reached for a condition.
        After testing a sample, a bar of alumina was cut to the same dimensions as the sample and the process was repeated to obtain a blank which could be subtracted from the measurements to remove buoyancy effects.

        Oxygen non-stoichiometry was calculated using Equation \ref{eq:TGA}, where $\Delta\delta$ is the change in oxygen stoichiometry, $MW_{SFCM}$ is the molecular weight of SFCM (\SI{208.74}{\gram\per\mol}), $MW_O$ is the molecular weight of oxygen (\SI{16.0}{\gram\per\mol}), $w_{sample}$ is the weight of the sample, and $\Delta{}w$ is the weight change as recorded by the TGA.
        The oxygen vacancy concentration ($\lbrack\ch{V_O^{**}}\rbrack$) is calculated using Equation \ref{eq:TGAtoV}, where $\rho$ is the density of SFCM, and $N_A$ is Avogadro's number.
        To calculate the oxygen vacancy concentration, the non-stoichiometry of SFCM at a point needs to be established.
        For this work, based on the plateau present in the data at oxidizing conditions, it is assumed that in a pure oxygen environment ($log(\po2{})=0$) all oxygen vacancies are filled with no oxygen interstitial or surface species, thus $\delta=0$.
        \begin{equation}
            \Delta\delta = \frac{MW_{SFCM}}{MW_O\ w_{sample}}\Delta{}w
            \label{eq:TGA}
        \end{equation}
        \begin{equation}
            \lbrack\ch{V_O^{**}}\rbrack =\frac{\delta\rho{}N_A}{MW_{SFCM}}
            \label{eq:TGAtoV}
        \end{equation}

    \subsection{Temperature Programed Desorption}
        The effluent from the TGA was used as the inlet to a mass spectrometer (MS) to perform temperature programmed desorption.
        The sample was prepared as before and heat treated to remove any carbon contaminants but was allowed to cool under simulated air.
        It was then heated to \SI{800}{\celsius} at \SI{5}{\celsius\per\minute} under a \SI{50}{sccm} flow of nitrogen as the MS measured the \SI{32}{m/z} signal which corresponded to O\textsubscript{2} desorption.
        Additional m/z signals were monitored to observed for other species.

\section{Results and Discussion}
    The XRD patterns obtained after reduction and oxidation, in addition to the fresh as synthesized powder and the pattern obtained from Rietveld refinement, are presented in Figure \ref{fig:structure}a.
    An unordered tetragonal, perovskite structure was used as a starting point for Rietveld refinement based on the structures of \ce{Sr_2CoMoO_6}, \ce{Sr_2NiMoO_6} and \ce{Sr_2Fe_{0.75}Co_{0.25}MoO_6}.\cite{Huang2009,Ritter2004}
    Results from the Rietveld refinement calculations are given in Table \ref{tab:xrdrefine}.
    The unordered double perovskite unit cell from the refinement is presented in Figure \ref{fig:structure}b and was used to create the theoretical XRD pattern for visual comparison.

    \begin{figure}
      \includegraphics[width=3in]{XRD.jpg}
      \includegraphics[width=3in]{Structure.png}
      \caption{a) Powder XRD patterns of SFCM samples taken after synthesis, reduction, and oxidation compared to pure phase SFCM diffraction pattern. b) Crystal structure of ordered double perovskite SFCM. }
      \label{fig:structure}
    \end{figure}

    \begin{table}
        \centering
        \caption{Refinement results of SFCM before and after exposure to reducing and oxidizing environments}
        \label{tab:xrdrefine}
        \begin{tabular}{llll}
        Condition & Oxidized & Fresh   & Reduced  \\
        \hline
        Space Group                & I 4$\overline{\text{m}}$    & I 4$\overline{\text{m}}$   & I 4$\overline{\text{m}}$    \\
        a (\si{\angstrom})        & 5.5636(2)   & 5.5558(3)  & 5.5544(2)   \\
        c (\si{\angstrom})        & 7.9310(2)   & 7.9184(3)  & 7.9204(2)   \\
        Volume (\si{\angstrom\cubed}) & 245.49  & 244.41 & 244.35  \\
        Sr-O1 (\si{\angstrom})    & 2.794  & 2.790 & 2.790   \\
        Sr-O2 (\si{\angstrom})    & 2.789  & 2.785 & 2.784   \\
        M1-O1 (\si{\angstrom})    & 1.993  & 1.990 & 1.990   \\
        M1-O2 (\si{\angstrom})    & 1.782  & 1.780 & 1.780   \\
        M2-O1 (\si{\angstrom})    & 1.950  & 1.948 & 1.947   \\
        M2-O2 (\si{\angstrom})    & 2.183  & 2.180 & 2.180   \\
        Phase Purity (\% SFCM)    & 85.24  & 76.08 & 100.0   \\
        R\textsubscript{w} (\%)   & 14.277 & 15.519 & 17.324
        \end{tabular}
        \end{table}

    SFCM proved to be stable under reducing conditions, being pure phase based on the Rietveld refinement.
    \ce{Sr_2Co_{1.2}Mo_{0.8}O_6} was found to be an impurity in SFCM under ambient and oxidizing conditions.
    This secondary phase reversibly disappears when returning to reducing conditions.
    Overall phase purity remains at greater than 76\% as synthesized, 85\% under oxidizing conditions and at 100\% after reduction.

    Changes in the SFCM component of the XRD pattern occur after different environmental treatments.
    These are attributed to the changes in oxidation state, defect concentrations, and thus lattice parameter of the sample after exposure to different \po2{}.
    The changes to the lattice were calculated from Rietveld refinements of the XRD data and summarized in Table \ref{tab:xrdrefine}.
    Reduction of SFCM from an as synthesized state causes a 0.025\% decrease in $a$ and the volume of the unit cell and a 0.025\% increase in $c$.
    These are result of the bond lengths between Sr-O2 and M2-O1 decreasing.
    Oxidizing the SFCM results in a 0.14\% increase in $a$ and a 0.16\% increase in $c$ resulting an 0.44\% increase in cell volume.
    This is the result of all bond lengths increasing all bond lengths by \SI{0.002} to \SI{0.004}{\angstrom}.
    Perovskite structures can have small to no changes in lattice parameter due to chemical expansion, which is observed in this sample, where the largest observed change in lattice parameter is 0.16\%.\cite{Bishop2014}
    Additionally, when changes do occur, perovskites have been shown to have anisotropic chemical expansion, as such with SFM where $a$ increases upon reduction while $c$ decreases.\cite{Tsvetkov2016}
    As a result, SFCM shows very little change to lattice parameters as a result of changing metal oxidation states or vacancy creation as it under goes reduction or oxidation.

    SFCM has a conductivity dependence typical of MIEC conductors, as shown in Figure \ref{fig:conductivity}.
    At high \po2{} ranges (\SI{e-1.5}{atm} to \SI{1}{atm}) it shows p-type conductivity, where conductivity increases with \po2{}.
    Down to \SI{e-1}{atm}, the change in conductivity is linear with a slope of \SI{0.089} (log-log scale).
    Below \SI{e-1}{atm} the slope changes away from the previous trend, suggesting a change in the predominant charge carrier.
    At low \po2{} ranges (\SI{e-24}{atm} to \SI{e-19}{atm}) the conductivity behaves linearly with n-type conductivity, increasing at lower \po2{}, with a slope of \SI{-0.11} (log-log scale).
    The sample measured in this work exhibited a lower conductivity than that reported in previous work (\SI{2}{S\per\centi\meter} compared to \SI{30}{S\per\centi\meter} at \temp{600}).\cite{Pan}
    This difference is likely due to sample preparation creating additional porosity, reducing the absolute conductivity, but the trend with \po2{} remains consistent.
    %Due to the inability to measure the intermediate \po2{} we were unable to determine in this work at what point the \po2{} dependence changes from n-type to p-type and if there is an ionic regime in between.

    \begin{figure}
      \includegraphics[width=6in]{conductivity.jpg}
      \caption{Total conductivity as \po2{} changes under a) reducing and b)oxidizing conditions at \SI{600}{\celsius}.}
      \label{fig:conductivity}
    \end{figure}

    Based on the slopes in the high \po2{} region, at least two defect regimes exist in that area transitioning near \SI{e-1}{atm}.
    Both high and low \po2{} regions have slopes much lower than +1/4 or -1/4 respectively, which indicates that the relationship between electronic charge carriers and oxygen is dependent of the other defect species in the material.
    In fact, the lower than 1/4 slope suggest that the electronic charge carries are taken up by the other species, decreasing the expected number based on \po2{} change.
    At low \po2{}, n-type conductivity is promoted by the generation of oxygen vacancies, freeing electrons as negative electronic charge carriers to maintain charge balance.
    Some of these electrons are consumed by the reduction of B-site metals to lower oxidation states.
    At high \po2{}, oxygen vacancies are filled and B-site metals oxidize, consuming electrons and allowing for the creation of holes, promoting p-type conductivity.
    The different slopes between high and low \po2{} are the result of different defect species interacting with the electronic carriers at different points in the reduction.

    TPD was performed in conjunction with TGA to determine the extent of oxygen desorption from SFCM.
    Figure \ref{fig:TPD} presents the data from the TGA on top with the mass and rate of mass change, while the \SI{32}{m\per z} signal from oxygen in the mass spectrometer is on bottom.
    The cyclic noise present in both mass and MS signals is due to an improperly tuned PID controller on the TGA furnace, but the noise remained much lower than the measured signal.
    The rate of mass loss matches the oxygen signal from the MS, confirming that oxygen is generated from SFCM when heated under mild reducing conditions and is the cause for mass loss in the sample.
    SFCM shows two maxima for the rate of oxygen loss during heating.
    The first, \textalpha, occurs at \temp{405} with the second peak, \textbeta, occurring near \temp{800}.
    The MS monitored for other species, such as carbon and water, but no significant amounts were observed.

    \begin{figure}
      \includegraphics[width=4in]{TPD.jpg}
      \caption{Temperature programmed desorption of oxygen in SFCM with mass loss from TGA(top) and oxygen desorption from MS (bottom) as it is heated to \SI{800}{\celsius} in N\textsubscript{2}.}
      \label{fig:TPD}
    \end{figure}

    Perovskite materials have oxygen desorption classified based on the temperature which it occurs.
    Low temperature desorption (referred to as \textalpha{}), occurs due to the reduction of metals and desorption of oxygen near the surface.
    High temperature desorption (referred to as \textbeta{}), takes place when oxygen is able to diffuse though through the bulk and be released.\cite{Levasseur2009}
    SFCM shows both \textalpha{} and \textbeta{} desorption, indicating that both the surface and lattice can generate oxygen vacancies due to the reduction of the different B-site cations and that those vacancies are mobile to move from the bulk to the surface.
    SMM shows only \textbeta{} desorption above \temp{800} and while \ce{Sr_2Fe_{1.5}Mo_{0.5}O_6} (SFM) has \textalpha{} desorption from \temp{400} to \temp{800} and \ce{Sr_2CoMoO_6} (SCM) desorbs starting below \temp{400}.\cite{Liu2011, Vasala2010}
    In the case of SFM and SCM, the use of the multivalent Fe or Co allows for reduction at a lower temperature than SMM.
    For SFCM, the same effect is seen through the combined use of Fe and Co.
    This demonstrates SFCM's ability to begin to create oxygen vacancies at \temp{400} and exchange oxygen with the bulk lattice starting as low as \temp{450} and peaking near \temp{800}.

    Oxygen non-stoichiometry and oxygen vacancy concentrations are shown in Figure \ref{fig:TGA600} at \temp{600} for high and low \po2{} regions.
    At high \po2{}, SFCM has two linear regions, changing between them near \SI{e-0.75}{atm}.
    The transition between regions in oxygen stoichiometry occurs at a similar \po2{} to where the conductivity changes in Figure \ref{fig:conductivity}.
    This change is the result of a change in defect regime regions.
    In the low \po2{} region, a linear trend occurs down until \SI{e-21}{atm} where the non-stoichiometry rate increases.
    Due to the pure phase XRD under reducing conditions, it is not likely that the increase in non-stoichiometry is due to phase decomposition and instead is a change in regimes where more oxygen vacancies are present.

    \begin{figure}
      \includegraphics[width=5in]{TGA600.jpg}
      \caption{Non-stoichiometry (top) and corresponding oxygen vacancy concentration (bottom) of SFCM under oxidizing conditions (right) and reducing conditions (left) at \SI{600}{\celsius}.}
      \label{fig:TGA600}
    \end{figure}

    SFCM shows similar trends in non-stoichiometry to SMM and SFM, but with the added complexity expected from the additional B-site species.
    SMM approaches a plateau in oxygen content as it reached \SI{1}{atm} of oxygen the same as SFCM but possesses a constant slope of -1/6 when plotted on a log-log scale.\cite{Marrero-lopez2010}
    In comparison, SFCM has multiple slopes associated with the degree of vacancy formation, as shown in Figures \ref{fig:TGA600}c and \ref{fig:TGA600}d.
    These different slopes can be attributed to the reduction of the three different B-site cations at different \po2{}.
    As an oxygen vacancy forms, the electrons required to maintain charge balance can be accommodated by the reduction of Fe, Co, or Mo, each with an independent equilibrium constant, whereas SMM only has Mo reduction to accommodate electrons from oxygen vacancy generation.
    The high slopes of Figure \ref{fig:TGA600}d show how SFCM can generate more oxygen vacancies at lower \po2{} because of the Fe and Co cations.
    SFM and SMM show similar amounts of non-stoichiometry at the same \po2{} but at \temp{1000}.\cite{Kircheisen2012}
    SFCM shows almost twice the non-stoichiometry at a much lower temperature of \temp{600}.
    SFCM's ability to create vacancies at higher \po2{} results in a larger accumulated vacancy concentration at low \po2{}.

    TGA non-stoichiometry measurements were performed on the same sample at \temp{400},  \temp{500}, and \temp{600} which are temperatures in the low temperature-SOFC range.
    Figure \ref{fig:TGAtemps} shows the non-stoichiometry measurements for the sample  tested at three temperatures in the low \po2{} (a) and high \po2{} (b) regions.

    \begin{figure}
      \includegraphics[width=6in]{TGAtemps.jpg}
      \caption{Non-stoichiometry of SFCM as \po2{} changes at \SI{400}{\celsius}, \SI{500}{\celsius}, and \SI{600}{\celsius}.}
      \label{fig:TGAtemps}
    \end{figure}

    As expected lowering the temperature reduces the non-stoichiometry at a given \po2{} and decreases the \po2{} at which transitions between regimes occur.
    At \temp{400} oxygen vacancy formation decreases considerably compared to \temp{500} and \temp{600}
    This aligns with the results of the TPD (Figure \ref{fig:TPD}) where SFCM does not start desorbing oxygen until \temp{350} to \temp{400} and then only from the surface.

    To best understand what is occurring in SFCM a defect equilibrium diagram and model need to be created to relate the changes in conductivity and non-stoichiometry to the concentration of the various defects present in the material.
    Defect reactions with their corresponding equilibrium equations are given in Equations \ref{rxn:intrinsic}{--}\ref{rxn:metalred}, which use Kr\"oger–Vink notation, where a species is denoted by the site it sits on (subscript) and the relative charge to the site's expected valency (superscript).
    For example, $\ch{V_O^{**}}$ is a vacancy in an oxygen site with a 2+ charge and $\ch{B_B^'}$ is a B-site metal in a B-site with a 1$-$ charge.
    K represents the equilibrium, or mass-action, constant for the reaction.
    To simplify the set of equations, Equation \ref{rxn:metalred} represents the combined reduction of any B-site cation (Fe, Co, or Mo).
    \textbeta{} is the ratio of A-site vacancies to B-site vacancies from Schottky defects.
    \begin{align}
        \label{rxn:intrinsic}
        \emptyset& \ch{<-> e' + h^{*}}&   K_i&  = np\\
        \label{rxn:schottky}
        \emptyset& \ch{<->  3 V_O^{**} + V_A^{''} + V_B^{''''}}& K_s& = \lbrack\ch{V_O^{**}}\rbrack^3 \lbrack\ch{V_A^{''}}\rbrack \lbrack\ch{V_B^{''''}}\rbrack = \lbrack\ch{V_O^{**}}\rbrack^3 \lbrack\ch{V_B^{''''}}\rbrack ^{\beta + 1}\\
        \label{rxn:external}
        \ch{O_O^x}& \ch{<-> 1/2 O2 + V_O^{**} + 2 e'}& K_r& = p_{O_2}^{1/2} \lbrack\ch{V_O^{**}}\rbrack n^2\\
        \label{rxn:metalred}
        \ch{B_B^x + e^'}& \ch{<-> B_B^'}& K_{B}& = \lbrack\ch{B_B^{'}}\rbrack n^{-1}
    \end{align}

    The Duncan-Wachsman approach allows for the accurate solution of defect equations across \po2{} regions using dominant defect triads, compared to the dominant defect pairs of the Brouwer approach which lead to an inaccurate results near transition regions.\cite{Duncan2007}
    The Duncan-Wachsman approach is needed in this work to model the defect equilibria because the measured \po2{} range crosses various Brouwer regions, as indicated by the non-linearity of Figure \ref{fig:TGA600}c.
    Using this method, Equations \ref{rxn:intrinsic}{--}\ref{rxn:metalred} can be solved to give models of the defect concentrations across \po2s.
    Due to the fact that impurity phases form under oxidizing conditions, only the pure phase, reducing \po2s were solved for.
    Table \ref{tab:defectequ} gives the resulting equations from the Duncan-Wachsman approach for the defect concentrations at these \po2{}.
    To allow for the solution of the equations, it was assumed that all B-site metals will have already reduced to their maximum potential, that is $\lbrack\ch{B_B^'}\rbrack$ is constant, in the \po2{} range of interest.

    \begin{table}
    \centering
    \caption{Equations for defect equilibrium under reducing conditions}
    \label{tab:defectequ}
    \begin{tabular}{c|c}
    Defect & Equation\\
    \hline
    n  & $K_r^{\frac{1}{2}} p_{O_2}^{-\frac{1}{4}}\left(\frac{3}{4} K_r^{\frac{1}{2}}p_{O_2}^{-\frac{1}{4}}+\left(\frac{1}{2}\lbrack\ch{B_B^'}\rbrack\right)^{\frac{3}{2}}\right)^{-\frac{1}{3}}$  \\[10pt]

    p  & $K_iK_r^{-\frac{1}{2}} p_{O_2}^{\frac{1}{4}}\left(\frac{3}{4} K_r^{\frac{1}{2}}p_{O_2}^{-\frac{1}{4}}+\left(\frac{1}{2}\lbrack\ch{B_B^'}\rbrack\right)^{\frac{3}{2}}\right)^{\frac{1}{3}}$  \\[10pt]

    $\lbrack\ch{V_O^{**}}\rbrack$   & $\left(\frac{3}{4} K_r^{\frac{1}{2}}p_{O_2}^{-\frac{1}{4}}+\left(\frac{1}{2}\lbrack\ch{B_B^'}\rbrack\right)^{\frac{3}{2}}\right)^{\frac{2}{3}}$\\[10pt]

    $\lbrack\ch{V_A^{''}}\rbrack$   & $\beta{}K_s^{\frac{1}{\beta+1}}\left(\frac{3}{4} K_r^{\frac{1}{2}}p_{O_2}^{-\frac{1}{4}}+\left(\frac{1}{2}\lbrack\ch{B_B^'}\rbrack\right)^{\frac{3}{2}}\right)^{-\frac{2}{\beta+1}}$ \\[10pt]

    $\lbrack\ch{V_B^{''''}}\rbrack$ & $K_s^{\frac{1}{\beta+1}}\left(\frac{3}{4} K_r^{\frac{1}{2}}p_{O_2}^{-\frac{1}{4}}+\left(\frac{1}{2}\lbrack\ch{B_B^'}\rbrack\right)^{\frac{3}{2}}\right)^{-\frac{2}{\beta+1}}$ \\[10pt]

    $\lbrack\ch{B_B^'}\rbrack$ & Constant \\

    \end{tabular}
    \end{table}

    Using the TGA and conductivity data with the equations given in Table \ref{tab:defectequ} the reaction equilibrium constants can be fitted to obtain a general defect equilibrium diagram for reducing \po2s as shown in Figure \ref{fig:defects}a.
    The electron mobility was calculated from the fitted data at \temp{600} to be \SI{0.0408}{\centi\meter\squared\per\volt\per\second}, which is within the range for small polaron defects for SMM.\cite{Marrero-lopez2010}
    By fitting and plotting the values of K\textsubscript{r} from various temperatures (Figure \ref{fig:defects}b), the enthalpy of formation for oxygen vacancies was found to be \SI{39.1}{\kilo\joule\per\mol} in SFCM under reducing conditions.

    \begin{figure}
      \includegraphics[width=6in]{defect.jpg}
      \caption{Oxygen vacancy and electronic defects in SFCM based on conductivity and TGA non-stoichiometry.}
      \label{fig:defects}
    \end{figure}

    While the fit for the oxygen vacancy concentration in Figure \ref{fig:defects}a fits well, the model predicts a steeper slope for the concentration of electrons.
    The cause of this is from the assumption that $\lbrack\ch{B_B^'}\rbrack$ is a constant, made while solving the defect reaction equations.
    In reality, while most B-site metals will have reduced, a portion will still be reducing at the low \po2{}, decreasing the overall number of conduction electrons, and reducing the \po2{} dependence of n.
    At low \po2{} both the concentrations of electrons and oxygen vacancies are on the same order of magnitude promoting mixed conduction.
    SFCM has an enthalpy of formation lower than that of comparable materials.
    SFM and SMM have much higher enthalpies of \SI{253}{\kilo\joule\per\mol} and \SI{110}{\kilo\joule\per\mol} respectively.\cite{Kircheisen2012,Marrero-lopez2010}
    \ce{SrFeO_{$3-\delta$}} has a more comparable, but still greater, enthalpy of \SI{80}{\kilo\joule\per\mol}.\cite{Holt1999}
    This low enthalpy of formation of oxygen vacancies in SFCM allows for a greater range of non-stoichiometry and thus SFCM's high performance.


\section{Conclusions}
    SFCM has potential as an electrode material in a number of electrochemical device applications.
    Specifically, it has a potential to replace nickel-based anodes in SOFCs due to its high conductivity and redox stability, creating an all-ceramic anode.
    SFCM forms impurity phases when heated at \temp{600} under oxidizing environments but becomes pure phase with reduction.
    It also exhibits very small chemical expansion with changing \po2{}, in line with its perovskite structure.
    SFCM is an improvement over previous double perovskite materials because of high conductivity at temperatures below \temp{600}.
    This is in part because SFCM supports the formation of oxygen vacancies at both surface and lattice positions shown by overlapping \textalpha{} and \textbeta{} oxygen desorption, starting as low at \temp{350} and continuing up until above \temp{800}.
    Oxygen vacancies form more rapidly than other MIEC materials with decreasing \po2{} as combination of B-site cations enables reduction though a large range of \po2{} resulting in a low enthalpy of formation for oxygen vacancies at \SI{39.1}{\kilo\joule\per\mol}.
    Proposed defect equilibrium equations were given for the low \po2{} range, supported by thermogravimetric and conductivity measurements for the complex system.
    Thermogravimetry at lower temperatures demonstrated SFCM's activity until \temp{400} at which the amount of oxygen vacancy formation slows dramatically.
