% !TEX root = ../mainthesis.tex

\chapter{Conclusions}
\section{Summary of Contributions}

\Glspl{sofc} are an interesting technology with the promise of helping meet growing energy demands by improving fuel conversion efficiency.
The mass deployment and adoption of \glspl{sofc} are limited by their high operating temperature and reliability concerns.
New materials have been developed to help lower the operating temperature of \glspl{sofc} to less than \temp{600}, but an understanding of how these materials affect reliability needed to be developed.
This work has done that, investigating factors which impact mechanical strength and thus the structural integrity of the cell.

To properly investigate the impacting factors on mechanical properties, a test apparatus was built which was capable of testing materials at conditions which are common in an \gls{sofc}.
Previously, only the temperature or \po2{} effects on strength or modulus could be measured.
This new apparatus not only can test the flexural properties of materials at temperatures in excess of \temp{600}, but it also allows for the control of the gaseous environment.
The interdependent effects of temperature and \po2{} are now able to be evaluated using this setup.

The use of \gls{gdc} has enabled the lowering of \gls{sofc} temperatures, with Ni-GDC commonly performing as the anode.
As a result of this trend, the mechanical properties of GDC, Ni-GDC, and Ni-GDC/GDC half-cells were investigated.
It was demonstrated how the amount and choice of pore-former can impact the mechanical strength due to total porosity percent and shape factor of the pores providing crack tip blunting.
Coupons of \gls{asl} Ni-GDC and half-cell Ni-GDC/GDC linearly increased strength upon heating with no dependence on half-cell orientation for strength.
Upon reduction, all cells showed significant decrease in strength as NiO reduced to Ni, increasing porosity and as \gls{gdc} reduced.
After reduction, the highest strength was shown by samples which had the electrolyte in compression.
Finally, it was demonstrated that reduced Ni-GDC/GDC half-cells could exposed to air for extended periods of time when below \temp{100} without oxidation of the Ni or significant impact to the flexural strength upon reheating under reducing atmospheres.

To develop a redox stable system, the phase purity and oxygen non-stoichiometry of a new ceramic anode material, \gls{sfcm}, was measured under oxidizing the reducing conditions at temperatures in the low temperature-\gls{sofc} range.
Under a H\textsubscript{2} \gls{sfcm} is pure phase, but forms \ce{Sr_2Co_{1.2}Mo_{0.8}O_6} as an impurity under oxidizing conditions.
\Gls{sfcm} itself has very little lattice parameter change at different \po2{} but the impurity \ce{Sr_2Co_{1.2}Mo_{0.8}O_6} has a cubic structure with a very different lattice parameter.
Oxygen desorbs from \Gls{sfcm} starting at \temp{350} and continues in excess of \temp{800}, enabling it to serve as an \gls{miec} at temperatures below \temp{600}.
\Gls{sfcm} posses a low enthalpy of formation of oxygen vacancies (\SI{39.1}{\kilo\joule\per\mol}), allowing a large number of oxygen vacancies to form, up to a $\Delta\delta = 0.176$ at \temp{600} in 97\% H\textsubscript{2}.
The Duncan-Wachsman method was then applied to propose an initial defect equilibrium diagram to match the observed behavior.

This information on the phase and point defect structure of \gls{sfcm} was used in understanding the mechanical properties of \gls{sfcm} and SFCM-GDC composite half-cells.
\Gls{sfcm} fracture toughness was measured to be \SI[separate-uncertainty = true]{0.124 +- 0.023}{\mega\pascal\sqrt{m}} at room temperature, increasing to \SI[separate-uncertainty = true]{0.286 +- 0.038}{\mega\pascal\sqrt{m}} at \temp{600}, due to the relaxation of residual stresses from sintering.
SFCM-GDC/GDC half-cells increased strength by 23.9\% after heating from room temperature to \temp{600} due to the increase in fracture toughness combined with thermal expansion applying compressive stresses to pre-existing flaws.
After reduction, the half-cells showed a decrease in strength as impurities decreased and oxygen vacancies were generated in \gls{sfcm} and \gls{gdc}.
Redox cycling was found to have no impact on the flaw distribution of SFCM-GDC/GDC half-cells, but the cells did decrease the overall characteristic strength due to the growth of microcracks from the changing amount of \gls{sfcm} impurities.

Combined, this work evaluates the impact of both temperature and \po2{} on the flexural strength of a current, state of the art electrolyte and anode and a new potential anode material, based on microstructure and point defects.
These results can feed back into the design and fabrication of \gls{sofc} cells and stacks to improve reliability.
With the knowledge that \gls{gdc} based cells only decrease strength upon reduction, not heating, pre-reduction steps or a gentler reduction process may be developed which reduce the likelihood of failure.
Additionally, this work shows how, while an improvement compared to Ni-GDC, \gls{sfcm} cells do weaken during redox cycling due to impurity formation.

\section{Future Work}
\begin{itemize}
    \item Finite element analysis to model the results into a larger cell
    \item Mitigation strategies, crack defection, pinning %add more
\end{itemize}

\section{Publications and Presentations}
    \subsection*{Publications}
        \begin{itemize}
            \item Stanley P, Hays TH, Langdo T, Wachsman ED. Mechanical Properties of SOFC Anode Support Materials at Operating Conditions. ECS Trans 2017 May 30;78(1):2285–91.
            \item Stanley P, Hays T, Langdo T, Gore C, Eric D. High Temperature Mechanical Behavior of Porous Ceria and Ceria-Based SOFCs. J Am Ceram Soc.  (Submitted)
            \item Stanley P, Hussain AM, Huang Y, Gritton JE, Neill JAO, Wachsman ED. Defect chemistry and oxygen non-stoichiometry in \ce{SrFe_{0.2}Co_{0.4}Mo_{0.4}O_{3-\textdelta}} for solid-oxide fuel cells. (in progress)
            \item Stanley P, Hussain AM, Hays T, Robinson I. Flexural strength and flaw distributions of \ce{SrFe_{0.2}Co_{0.4}Mo_{0.4}O_{3-\textdelta}} solid-oxide fuel cells at operating conditions. (in progress)
        \end{itemize}

    \subsection*{Presentations}
    \begin{itemize}
        \item Stanley P, Hays TH, Langdo T, Wachsman ED. Mechanical Properties of SOFC Anode Support Materials at Operating Conditions. SOFC-XV, Hollywood, FL  July 2017
        \item Stanley P, Hays TH,  Wachsman ED. Mechanical Characterization of SOFC Anode Support Materials at Operating Conditions. ECS Fall Meeting. National Harbor, MD Oct 2017
    \end{itemize}
