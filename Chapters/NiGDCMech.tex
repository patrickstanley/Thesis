% !TEX root = ../mainthesis.tex

\chapter[High Temperature Mechanical Behavior of Porous Ceria and Ceria-Based Solid-Oxide Fuel Cells]{High Temperature Mechanical Behavior of Porous Ceria and Ceria-Based SOFCs}

\section{Introduction}
This study presents the results of a broad range of flexural tests involving the materials used in ceria-based anode supported \glspl{sofc} along with the assembled half-cells.
Using a purpose-built temperature controlled environmental chamber installed in a \gls{utm}, porous doped ceria bars, \glspl{asl} composed of nickel and doped ceria cermet, and half-cells composed of an \gls{asl} and a doped ceria electrolyte were tested.
The various test conditions used included expected operating temperatures (\temp{450}{--}\temp{650}), and both reducing and oxygenated atmospheres.
These variations in test conditions are important because these cells must maintain their integrity from when they are first placed in a sealed stack to when they reach operating conditions.
In particular, this study was intended to determine at what point in their life \glspl{sofc} are most vulnerable to mechanical failure and the mechanisms involved.
Additionally, the effect of the anode-electrolyte interface on flexural strength was explored.

\section{Results}

\subsection{Porosity and Pore Former Choice}

As expected, \gls{gdc} bars with greater porosity displayed lower flexural strength values as compared to less porous samples.
The relationship between porosity and strength followed an exponential trend.
This behavior is well established for porous ceramics and is described by
Equation \ref{eq:porestrength}, where \(\sigma_{f}\) is the flexural strength with porosity,
\(\sigma_{o}\) is the flexural strength without porosity, \(\eta\) is a geometric constant dependent on the system, and \(P\) is the volume percent porosity of the sample.\cite{Rice1998}
\begin{equation}
    \sigma_f = \sigma_o e^{-\eta{}P}
    \label{eq:porestrength}
\end{equation}

The data was fit using this equation and both a fixed geometric constant, $\eta$, of 0.047 (Figure \ref{fig:gdcporsity}a) and a variable geometric constant (Figure \ref{fig:gdcporsity}b).
Chi-squared values, indicating the divergence of the fit from the measured data for each sample set, were calculated for the two fitting methods and are displayed in Table \ref{tab:porousgdcfit}.

\begin{figure}
    \includegraphics[width=.6\textwidth]{GDCPorosity.jpg}
    \caption{Flexural strength-porosity dependence for porous \gls{gdc} at \temp{650} and \temp{25} using spherical \gls{pmma} or graphite flake pore former: a) Measured strengths fitted using fixed geometric constant; b) Measured strengths fitted using different geometric constants}
    \label{fig:gdcporsity}
\end{figure}

\begin{table}[b]
\centering
\caption{Chi-squared values for fixed $\eta$ and variable $\eta$ fits of porous \gls{gdc} strength data (Figure \ref{fig:gdcporsity})}
\label{tab:porousgdcfit}
\begin{tabular}{lcc}
Sample Set       & $\chi^2$ (Fixed $\eta$)  & $\chi^2$ (Variable $\eta$)   \\
\hline
GDC PMMA 650     & 3.406               & 0.062                   \\
GDC Graphite 650 & 0.668               & 0.008                   \\
GDC Graphite 25  & 3.575               & 5.063
\end{tabular}
\end{table}

The deviance from the fit curve for the fixed \texteta{} method is quite large relative to the variable \texteta{} fit for both sample sets tested at \temp{650}.
Both fitting methods are fairly poor in the case of the room temperature graphite set.
The poor fitting results in this case are due to the \textgreater{}60\% porosity samples.
The flatter pore geometry caused by the graphite likely leads to comparatively greater strength loss at high porosity due to greater pore connectivity.
The comparison between the high temperature data sets is the most useful, given this, and the fact that \glspl{sofc} operate at this temperature.
The poor fit of fixed \texteta{} models suggests that a material-specific \texteta{} value may be an oversimplification of the porosity-stress relationship in ceramics.

Samples tested at higher temperatures displayed a slightly higher flexural strength.
Generally, it is expected that increasing temperature lowers strength of dense ceramics once plastic deformation processes become activated at high temperatures.\cite{Davidge1970}
Below the temperature at which plastic deformation starts to occur, temperature has a minimal effect on flexural strength of dense ceramics.
For dense \gls{gdc}, it has been demonstrated that the strength decreases by 19\% between room temperature and \temp{800}.\cite{Mogensen2000}
Porosity changes this behavior by essentially creating a composite material, keeping the rate of change constant or increasing with temperature.\cite{Giraud2008}
For porous \gls{gdc}, thermal expansion places a compressive stress on any surface flaws that are sites for crack initiation and propagation.
Additionally, the porosity helps reduce bulk stresses caused by thermal expansion.
These two effects increase the material's fracture strength more than any decrease from plastic deformation at \temp{625}.

Cross-sectional SEM of the fracture surfaces of the porous \gls{gdc} samples, both \gls{pmma} and graphite, is shown in Figure \ref{fig:GDCPoreSEM}.
Figure \ref{fig:GDCPoreSEM}a shows individual spherical pores left by the \gls{pmma} while Figure \ref{fig:GDCPoreSEM}b has an inter-connected network of long pores left by the graphite flakes.
\begin{figure}
    \includegraphics[width=0.8\textwidth]{GDCPoreSEMa.jpg}
    \includegraphics[width=0.8\textwidth]{GDCPoreSEMb.jpg}
    \caption{\Gls{sem} micrographs of porous \gls{gdc} bars made using: a) \gls{pmma} pore former; b) graphite flake pore former}
    \label{fig:GDCPoreSEM}
\end{figure}

The differences in quality of fit between Figure \ref{fig:gdcporsity}a and Figure \ref{fig:gdcporsity}b shows that the geometric factor is microstructurally dependent.
Samples with porosity formed using \gls{pmma} spheres showed significantly greater strength as compared to samples with graphite-formed porosity.
This can be explained by the effect of pore geometry on crack initiation and propagation in the ceramic.
If a crack enters a pore, the pore can now be considered the new crack tip.
The energy required to advance the crack is highly dependent on the geometry of the tip.
For a spherical pore, this geometric factor is maximized and results in higher resilience to fracture.
This is summarized by Equation \ref{eq:cracktip} where \(\sigma_{o}\) is the stress at the crack tip, c is the length of the crack tip and \(\rho\) is the radius of curvature of the crack tip.\cite{Carter2007}
\begin{equation}
    \sigma_{f} = \frac{\sigma_{o}}{2}\sqrt{\frac{\rho}{c}}
    \label{eq:cracktip}
\end{equation}

Based on the mechanical behavior of these samples, porous ceramics should be designed and constructed such that the pore geometry is as low aspect ratio as possible to maximize strength.

\subsection{Flexural Stress Orientation}

Figure \ref{fig:NiGDCHalfCells} shows the measure flexural strengths of a number of \gls{sofc} coupon samples.
Anode support coupons were tested along with half-cells consisting of anode supports and electrolyte layers at temperatures ranging from \temp{25} to \temp{650}.
The highest temperature test was repeated with coupons under reducing conditions (3\% H\textsubscript{2}, balance Ar).
\begin{figure}
    \includegraphics[width=0.8\textwidth]{NiGDCStrengthTemp.jpg}
    \caption{Temperature dependent strength of Ni-GDC anode supports and half-cells in both air (filled data points), and reducing atmosphere (3\% H\textsubscript{2}, balance Ar, hollow data points), tested in both ``electrolyte-up'' and ``electrolyte-down'' orientations resulting in the electrolyte layer experiencing tension and compression, respectively}
    \label{fig:NiGDCHalfCells}
\end{figure}

All unreduced sample types showed increased strength at elevated temperatures with little difference between types at a given temperature.
The high-temperature reduced coupons displayed large differences in strength depending on the orientation of the sample.
Tests in which the dense electrolyte layer was placed in compression resulted in the highest strength values, while the samples were weakest when the electrolyte was placed in tension.
In reduced samples, the anode support layer becomes a ceramic-metal composite and is therefore somewhat elastic while the electrolyte remains a brittle ceramic.
The electrolyte-in-compression condition maximizes the mechanical performance of the coupon by placing the layers in their preferred stress state.

Figure \ref{fig:ASLDirection} shows the box plots for the unreduced \gls{sofc} coupon samples.
A Student's t-test was used to statically determine if the means of samples tested at different conditions were equal based on 95\% confidence.
Details on the use of the test are given in Appendix \ref{app:ttest}.
There is no discernible difference in the strength between the three sample types at each temperature.
At elevated temperature there was a significant difference in the modulus of \gls{asl} only samples and samples with electrolyte in tension with a p-value of 0.0091.
This difference was not present at room temperature.
As the materials reach elevated temperature, small differences in elasticity become magnified due to different thermal effects on dense and porous layers.
\begin{figure}
    \includegraphics[width=\textwidth]{ASLDirection.jpg}
    \caption{Flexural test measurements of coupons sample sets: a) strength at \temp{25} b) strength at \temp{650} c) modulus at \temp{25} d) modulus at \temp{650}}
    \label{fig:ASLDirection}
\end{figure}

Additionally, \gls{sem} analysis of the fractured half-cells showed very good adhesion between layers (Figures \ref{fig:HalfCellSEMa} and \ref{fig:HalfCellSEMb}).
Delamination is a common failure mode in layered ceramics and one that would be particularly damaging to \glspl{sofc} due to resulting ionic conductivity loss between layers.\cite{Sevecek2016}
In the half-cell coupons, it was clear that the fracture plane contained mixed transgranular and intergranular fracture.
Some grains were sheared through while others remained whole.
The striations visible in Figures \ref{fig:HalfCellSEMa}b and \ref{fig:HalfCellSEMb}b are characteristic of fracture proceeding through a grain, while other grains remained whole.
\begin{figure}
    \includegraphics[width=.8\textwidth]{HalfCellSEMa.jpg}
    \includegraphics[width=.8\textwidth]{HalfCellSEMb.jpg}
    \caption{\Gls{sem} micrographs of unreduced half-cell fracture surface tested in air at \temp{25} at a) 2k magnification and b) 5k magnification}
    \label{fig:HalfCellSEMa}
\end{figure}

\begin{figure}
    \includegraphics[width=.8\textwidth]{HalfCellSEMc.jpg}
    \includegraphics[width=.8\textwidth]{HalfCellSEMd.jpg}
    \caption{\Gls{sem} micrographs of reduced half-cell fracture surface tested in reducing atmosphere at \temp{650} at a) 1k magnification and b) 5k magnification}
    \label{fig:HalfCellSEMb}
\end{figure}

\subsection{Reduction and Strength}

Mass loss of \gls{sofc} half-cell coupons exposed to reducing atmosphere at various elevated temperatures is shown in Figure \ref{fig:NiGDCReduction}.
Mass loss is attributed to the reduction of NiO, used as a precursor in fabrication,
to Ni metal, which serves as the catalyst for fuel oxidation and electronic conductor, and the reduction of ceria to \ce{CeO_{2-\textdelta}}.
Ni-GDC/GDC half cells showed the expected trend of increasing reduction rate at higher temperatures.
Reduction curves were fit to an exponential decay and the summary of parameters is shown in Table \ref{tab:NiGDCRedFit}.
At \temp{650} and \temp{700} the reduction occurs very quickly,
reaching steady state values after 18 hours. \temp{550} showed a much slower mass loss than even \temp{575}.
At temperatures lower than \temp{550}, the kinetics are slow enough to tolerate a brief exposure to oxygen without re-oxidizing the sample.
Of note in the reduction is that each temperature appears to approach a different asymptote, showing that the amount of NiO and \gls{gdc} reduced at steady state is dependent on the temperature of the cell.
This will affect the mechanical properties of the cells as it changes both the porosity and amount of nickel metal in the samples.

\begin{figure}
    \includegraphics[width=\textwidth]{NiGDCReduction.jpg}
    \caption{Thermogravimetric analysis curves for Ni-GDC/GDC half-cells showing mass loss over time at temperatures ranging from \temp{550} to \temp{750} in 3\% H\textsubscript{2} 3\% H\textsubscript{2}O balance N\textsubscript{2}, \SI{50}{sccm} flow.}
    \label{fig:NiGDCReduction}
\end{figure}

\begin{table}
\centering
\caption{Summary of fit parameters for reduction of NiO-GDC/GDC half-cell coupons under 3\% H\textsubscript{2}, 3\% H\textsubscript{2}O, 94\% N\textsubscript{2} at different temperatures.}
\label{tab:NiGDCRedFit}
\begin{tabular}{ccc}
Temperature (\temp{}) & Exponential Rate (\SI{}{\per\hour}) & Asymptote (Mass \%)  \\
\hline
550                                    & 0.0463                  & 89.78                 \\
575                                    & 0.135                   & 89.09                 \\
600                                    & 0.182                   & 88.36                 \\
650                                    & 0.512                   & 87.75                 \\
750                                    & 0.821                   & 87.87
\end{tabular}
\end{table}

\subsection{Reoxidation at Low Temperatures}
Ni-GDC/GDC fuel cells are sensitive to re-oxidation once reduced.
Re-oxidation causes fracture of cells due to the large volume difference between NiO and Ni.\cite{Nakajo2012}
Based on the results from the TGA, the reaction rates below \temp{300} are sufficiently slow to allow for the exposure of a reduced cell to oxygen without detrimental results.\cite{Richardson2003}
To confirm this, a sample was measured by TGA to ensure no mass gain during oxygen exposure and the strength of samples were tested to ensure there was no discernible difference between reduction methods.

For TGA analysis, a piece of a half cell was heated to \temp{650} with a \SI{10}{\celsius\per\minute} ramp rate and reduced.
It was then cooled while still under reducing atmospheres.
Once below \temp{100}, it was exposed to simulated air
(21\% O\textsubscript{2}) for 18 hours, placed back into reducing atmosphere, and heated back up to \temp{650}.
Figure \ref{fig:rereduction} shows the change in mass overlaid onto the temperature and oxygen partial pressure experienced by the sample.
Following this treatment, the cell showed no mass gain during the oxygen exposure and continued to reduce at the same rate as before once it was returned to the initial conditions.
This shows that cells which are cooled appropriately do not re-oxidize and could be handled in between a batch reduction of cells, and their assembly into a stack configuration.

\begin{figure}
    \includegraphics[width=\textwidth]{Rereduction.jpg}
    \caption{Thermogravimetric analysis of Ni-GDC/GDC half-cell showing no mass gain after reduction and exposure to room temperature and simulated air: a) Mass loss of interrupted reduction with exposure to ambient condition; b) Temperature of sample during cycle c) Oxygen partial pressure measured at the sample during cycle}
    \label{fig:rereduction}
\end{figure}

Box plots of flexural strength and modulus of the in-situ reduced and batch reduced coupons are shown in Figure \ref{fig:reductionmethod}.
Mechanical strength of half-cell coupons which had been reduced in-situ with an 18 hour reduction time showed no difference in strength when compared with coupons which had been previously batch reduced, cooled, and reheated under reducing environments.
Both of these sample types showed a dramatic decrease in strength and Young's modulus compared to the un-reduced samples, but no statistically significant difference between treatments.
The decrease in strength and modulus is due to the increase in porosity and conversion of NiO to metallic nickel in the cells.
The decrease in variation between measurements in reduced cells is likely due to the increased number of large voids that form during NiO reduction.

\begin{figure}
    \includegraphics[width=\textwidth]{ReductionMethod.jpg}
    \caption{Flexural properties, a) strength and b) modulus, of Ni-GDC/GDC half-cell coupons after reduction via two different methods, compared to unreduced cells. Strength and modulus show a decrease upon reduction but no significant difference between methods.}
    \label{fig:reductionmethod}
\end{figure}

The statistical similarity of the properties of the samples reduced via each method indicates that planar, Ni-GDC/GDC based \gls{sofc} \gls{asl} and half-cells are able to be safely reduced, cooled, and handled in ambient conditions without leading to damaging re-oxidation of the nickel anode material.
This resilience could enable some degree of large-batch reduction of cell anodes prior to \gls{sofc} stack assembly, leading to more rapid startup and a greater degree of sealing control.

\section{Conclusions}

A temperature and atmosphere controlled three-point bend fixture was designed and built for use in a universal testing machine.
\gls{sofc} coupons and component materials were evaluated for flexural strength at room temperature and \gls{sofc} operating temperatures.
In addition, the effects of porosity percent and pore geometry on flexural strength in \gls{gdc} were investigated.
Furthermore, the impact of temperature on the reduction rate of NiO in NiO-GDC \gls{sofc} anodes was examined along with the resilience to re-oxidation at ambient conditions of this \gls{sofc} component.

Pore geometry had a significant impact on the flexural strength of
\Gls{gdc}, with spherical pores showing the greatest resistance to fracture.
This supports the concept of pores acting as the new crack tip once a crack has advanced to the pore.
Additionally, samples tested at \temp{650} were stronger than those at room temperature.
This is likely due to localized compressive stresses from thermal expansion of the material.
This hypothesis is further supported by the results of testing NiO-GDC
anode support coupons and half-cells.
Coupons tested at \temp{25}, \temp{450}, and \temp{650} displayed a linear strength dependence with temperature.
There was no statistical difference in strength between anode support layers and half-cells composed of anode support and electrolyte at a given temperature in air.
Half-cells in which NiO was reduced to Ni by exposure to H\textsubscript{2} at \temp{650} displayed significant differences in strength when the electrolyte layer was subjected to compressive stress as opposed to tensile stress.
Placing the ceramic electrolyte in compression and the metal-ceramic composite anode in tension resulted in the highest strength.

The reduction temperature of NiO-GDC/GDC half-cells was shown to have an effect on the rate of NiO reduction and amount of NiO reduced.
At lower temperatures, the oxidation rate of Ni-GDC is slow enough that the anode can be exposed to air for significant periods below \temp{100}.
Any re-oxidation, combined with the cooling and re-heating of a cell back to
\temp{650}, showed no effect on the mechanical properties when compared to cells which had been reduced in-situ at \temp{650}.
These results indicate that it is possible to reduce and cool cell components to an extent without any additional effects to mechanical properties, allowing for more flexibility during cell manufacturing, stack assembly, and with quality control screenings.

This work leads to three important conclusions for the mechanical properties of \gls{gdc}-based \glspl{sofc} using Ni anodes.
Porous \gls{gdc} used in anode supported \glspl{sofc} should be fabricated such that the pore geometry is spherical as this maximizes energy required to advance a crack through the ceramic.
Care should be taken in stack construction to ensure any out-of-plane cells are placed to compress the electrolyte and to place the reduced anode in tension so as to lower the chance of fracture.
Finally, it is possible to reduce the anodes of Ni-GDC \glspl{sofc} and then handle them at ambient conditions for quality control and stack assembly.
This will remove a degree of variability from the manufacture of cells and stacks.
