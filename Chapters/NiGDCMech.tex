% !TEX root = ../mainthesis.tex

%Chapter 3

%\renewcommand{\thechapter}{3}

\chapter{High Temperature Mechanical Behavior of Porous Ceria and Ceria-Based SOFCs}

\section{Introduction}
For planar SOFCs, a standard stack design consists of a number of square cells separated by metal interconnects.\cite{Ki2010}
These interconnects act as current collectors, gas channels, and separators to prevent fuel and air from mixing.
Sealing of these stacks is accomplished by compressing the layered structure which includes sealing material.
As more compressive force is applied to the stack, the quality of the seals improves.
However, any cell which is not perfectly flat will experience flexural stress as a result and this can lead to cells fracturing.
Due to the brittle nature of the ceramic materials that constitute SOFCs, a very thorough understanding of the mechanical limits of these devices is critical to their successful deployment.

The vulnerability of SOFCs to mechanical failure is a well-known issue.
However, much of the research into this phenomenon has focused on yttrium stabilized zirconia (YSZ) based devices.\cite{Roa2011,Yu2007,Nakajo2012}
This material has been the standard for the field and has desirable mechanical properties but requires high temperatures to function well.
As efforts are made to lower SOFC operating temperatures, a shift to ceria-based electrolytes has occurred.
Less attention has been given to the mechanical behavior of doped ceria materials across SOFC operating conditions.

Much of the study of fracture in ceramic materials has been done on technical ceramics for medical applications and for coating metal components.\cite{Deng2004,Ambrozic2007,Yonezu2014,Meille2012,Nie2010,Olagnon1999,Sorensen2001,Barinov1997}
These materials are optimized for fracture toughness and durability and very rarely experience temperatures above a few hundred degrees Celsius.
The fracture surface analysis and correlations between microstructure and strength described for these materials are a valuable starting point for investigating fuel cell materials.
However, there is a lack of extensive investigation into the properties of IT-SOFC materials at their expected operating temperatures and environments.

Efforts have been made to comprehensively examine the mechanical properties of SOFC materials and the effect of non-standard conditions on those properties.
Nakajo et al. conducted a wide ranging study of materials used in anode supported SOFCs which included some attention to temperature and atmosphere effects.\cite{Nakajo2012}
While providing a solid base of material knowledge, this work did not fully cover materials beyond
YSZ and there remains a need for further testing, especially for doped ceria.
Flexural strength and Young's modulus measurements for gadolinium doped ceria (GDC) have been carried out in ambient conditions by Yasuda et al.\cite{Yasuda2012}
They characterized the effects of sintering temperature on density and the resulting mechanical properties.
Further testing of this material system at elevated temperatures and anode gas reducing environments must be done to fully understand the mechanical behavior of
GDC.

This article presents the results of a broad range of flexural tests involving the materials used in ceria-based anode supported SOFCs along with the assembled half-cells.
Using a purpose-built temperature controlled environmental chamber installed in a universal testing machine (UTM), porous doped ceria bars, anode support layers (ASL)
composed of nickel and doped ceria cermet, and half-cells composed of an
ASL and a doped ceria electrolyte were tested.
The various test conditions used included expected operating temperatures (\temp{450}{--}\temp{650}), and both reducing and oxygenated atmospheres.
These variations in test conditions are important because these cells must maintain their integrity from when they are first placed in a sealed stack to when they reach operating conditions.
In particular, this study was intended to determine at what point in their life SOFCs are most vulnerable to mechanical failure and the mechanisms involved.
Additionally, the effect of the anode-electrolyte interface on flexural strength was explored.

\section{Experimental}

\subsection{Sample Preparation}

GDC bars of varying porosity were fabricated using uniaxial pressing in a rectangular die.
GDC10 (\ce{Ce_{0.9}Gd_{0.1}O_{2}}) powder was mixed with the desired volume of poly(methyl methacrylate) (PMMA) microspheres (\SI{5}{\micro\meter} diameter) or graphite flake to create the green body, with 0.1wt\% of polyvinyl butyral added to aid in pressing and a drop of fish oil to act as a dispersant during milling.
The two different pore formers were used to create differing pore geometry.
The PMMA or graphite was removed by a \temp{400} pre-sintering step, and the remaining ceramic structure was sintered at \temp{1500} for 4 hours.
Archimedes density measurements were used to confirm porosity percent.
Samples were polished with 600 grit sandpaper on all sides before testing.

The ASL and half-cell coupons used in the flexural testing were made using tape casting.
NiO and GDC powders in a 6:4 weight ratio were mixed with ethanol, toluene, and fish oil.
The mixture was ball milled on a rotary mill for 24 hours.
Polyvinyl butyral and benzyl butyl phthalate were added, followed by another 24 hours milling.
The resulting slurry was degassed and tape cast using a 700 micron blade height.
The tapes were left to dry overnight before being cut into \SI{12}{\centi\meter} by \SI{12}{\centi\meter} squares.
Three squares were stacked and hot pressed at \temp{49} and 2 tons for 30 minutes.
Following this lamination, the tape was cut into rectangular coupons measuring 25mm by 10mm.
The coupons were fired at \temp{1400} for 4 hours.
Final bar dimensions were 8.16 by 24.15 by 2.98 mm on average.
For half-cell coupons GDC electrolyte slurry was prepared and tape cast with a \SI{40}{micron} blade height.
After the lamination of the three ASL layers, a single layer of electrolyte tape was laminated to the ASL using the same pressure for 2 hours.
The sintering procedure for half-cell coupons was identical to the ASL coupons.
Samples were polished with 600 grit sandpaper on edges before testing.

\subsection{Thermogravimetric Analysis}

To understand the degree of reduction in test coupons, thermogravimetric analysis (TGA) was used to measure the mass loss as a function of time,
temperature, and gas environment.
A Cahn D200 microbalance was used to measure the weight changes as the sample was heated in a furnace with controlled atmosphere.
A small section of test coupon was placed in a crucible suspended from the balance and heated to \temp{650} at \SI{10}{\celsius\per\minute}, the environment was switched from \SI{50}{sccm} of 21\% O\textsubscript{2} in
N\textsubscript{2} to \SI{50}{sccm} of 3\% H\textsubscript{2}, 3\%
H\textsubscript{2}O in N\textsubscript{2}.
Mass, temperature, and
\po2{} measurements were taken at 30 second intervals.
\po2{} measurements were taken using a calibrated YSZ sensor placed immediately before the sample.
Samples were cooled at \SI{10}{\celsius\per\minute} in the desired atmosphere.

\subsection{Development of Mechanical Test Apparatus}

To develop a testing apparatus capable of simulating the various conditions experienced by operating SOFCs, appropriate materials were chosen based on thermal and chemical stability criteria.
Alumina was chosen to build the bend fixture due to its chemical stability and high hardness.
The sample rests crossways on two stationary \SI{6.35}{\milli\meter} diameter rods which are placed in troughs separated by \SI{20}{\milli\meter}.
The upper half of the fixture consists of a \SI{6.35}{\milli\meter} rod used to apply stress to the sample from the UTM crosshead.
Both pieces are attached using silica-based cement to \SI{300}{\milli\meter} rods attached at the anchor points of the UTM.
Assembly and alignment is assisted with the use of a 3D printed jig to ensure repeatability.
The complete alumina, 3-point bend fixture is seen in Figure \ref{fig:3ptbend} with the two bottom rollers, a sample, and the top roller making contact, applying flexural stresses to the sample.

To create the atmosphere control system, a combination of standard and custom vacuum system parts were used to enclose the bend fixture.
For the main body of the chamber, a 3-inch inner diameter, stainless steel tee was made with QF80 and QF50 flanges.
Welded NPT ports on top and bottom allow for gas inlet and outlet and the introduction of a thermocouple through a Swagelok Ultra-Torr fitting.
Figure \ref{fig:chamber} shows the engineering drawings and photo of the custom pieces from A\&N Corp.
Flexible bellows allow for the motion of the cross-bar to be translated into the fixture, requiring the subtraction of the spring force to be removed during analysis.
The alumina fixture was then inserted through each end and the chamber incorporated into the furnace with QF50 to 1/2 inch tube adapters that fit over the alumina rods.

To heat the fixture and chamber, a 1 ft. cube was assembled from steel plates with cutouts on top, bottom and front.
Steel wire mesh was used to create a framework inside to hold nickel-chromium alloy heating elements.
Silica-based wool insulation was packed between the center cavity and the case walls.
A K-type thermocouple was inserted into the chamber and used with an external PID controller to cycle the heating elements.
The furnace was placed on a supporting scaffold to hold it in position during tests.
The test fixture, atmosphere chamber, and furnace were affixed together as one unit and inserted into the UTM.
The design was compared and qualified to a regular steel 3-point bend apparatus to ensure accurate results at room temperature.

\subsection{Bend Testing}

All tests conducted in the UTM (Tinius Olsen 10ST with 250 N load cell)
were done at a rate of \SI{0.2}{\milli\meter\per\minute} with a \SI{20}{\milli\meter} lower span.
After measuring the force at fracture, stress was calculated using Equation \ref{eq:3ptstrength}
for 3-point flexural of rectangular samples, where \(\sigma_{f}\) is the stress at failure, F is the load at failure, L is the span of the fixture, b is the width of the sample and d is the thickness of the sample.
At room temperature, the coupons or bars were loaded into the fixture and tested in batches of 5 per condition.
Occasionally samples would break or be damaged before testing, reducing the sample set.
For sample sets at elevated temperatures in ambient atmosphere, the samples were loaded into the front of the chamber, acting as a staging area,
prior to heating the on fixture in the center.
Upon transferring the next coupon from the staging area to the fixture, a 20 minute waiting period was used to ensure that the coupon had reached thermal equilibrium prior to testing.

Following the completion of the set, all coupon pieces were cooled at a rate of \temp{10}/min.
For testing in reducing atmosphere, each coupon was reduced and tested individually.
Half-cell coupons were tested in two orientations, "electrolyte-up" and "electrolyte-down." These orientations cause the dense electrolyte layer to experience compressive or tensile stress.
The "electrolyte-up" orientation subjects the electrolyte to compressive stress and vice-versa.

Following the destructive tests, SEM analyses of the fracture surfaces were performed.
The purpose of this was to observe the trans-granular,
inter-granular, or mixed nature of crack propagation, and to find any anomalous features on the fracture surface, as well as to observe pore geometry.

\subsection{Atmospheric Treatment}

To test individual coupons under reducing environments, samples were loaded into the test chamber at \temp{400}, after previous pre-heating in the staging area.
After 20 minutes, the temperature would be increased to the desired set-point of \temp{650} at a ramp rate of \SI{10}{\celsius\per\minute} and the gas would be switched first to N\textsubscript{2} to flush the chamber, then to humidified argon containing 3\% H\textsubscript{2}.
It would then be allowed to sit for 18 hours before testing to allow reduction.
After testing, the chamber would be flushed again with N\textsubscript{2} and cooled to \temp{400} before opening to change samples and repeat the process.

If samples had been previously reduced as a batch by exposure to hydrogen in a tube furnace and cooled under hydrogen, the atmospheric treatment was shortened, as supported by TGA results.
Pre-reduced samples were loaded into the chamber at or below \temp{400}, the chamber flushed N\textsubscript{2}, then with H\textsubscript{2} in Ar.
After this the chamber would be heated to \temp{650}, the sample tested after a 30
minute period, and cooled back down to \temp{400}.
At this point, the chamber would be flushed with N\textsubscript{2}.
While the chamber maintained a positive pressure of N\textsubscript{2}, it could be opened, a new sample loaded, closed, and allowed to flush.
This procedure allows for the batch reduction of many test coupons and replaced the 18 hour reduction time with a shorter 30 minute period.

\section{Results}

\subsection{Porosity and Pore Former Choice}

As expected, GDC bars with greater porosity displayed lower flexural strength values as compared to less porous samples.
The relationship between porosity and strength followed an exponential trend.
This behavior is well established for porous ceramics and is described by
Equation \ref{eq:porestrength}, where \(\sigma_{f}\) is the flexural strength with porosity,
\(\sigma_{o}\) is the flexural strength without porosity, \(\eta\) is a geometric constant dependent on the system, and \(P\) is the volume percent porosity of the sample.\cite{Rice1998}
\begin{equation}
    \sigma_f = \sigma_o e^{-\eta{}P}
    \label{eq:porestrength}
\end{equation}

The data was fit using this equation and both a fixed geometric constant, $\eta$, of 0.047 (Figure \ref{fig:gdcporsity}a) and a variable geometric constant (Figure \ref{fig:gdcporsity}b).

\begin{figure}
    \includegraphics[width=\textwidth]{GDCPorosity.jpg}
    \caption{Flexural strength-porosity dependence for porous GDC10 at
    \temp{650} and \temp{25} using spherical PMMA or graphite flake pore former: a)
    Measured strengths fitted using fixed geometric constant; b) Measured strengths fitted using different geometric constants}
    \label{fig:gdcporsity}
\end{figure}

Chi-squared values, indicating the divergence of the fit from the measured data for each sample set, were calculated for the two fitting methods and are displayed in Table \ref{tab:porousgdcfit}.

\begin{table}
\centering
\caption{Chi-squared values for fixed $\eta$ and variable $\eta$ fits of porous GDC strength data (Figure \ref{fig:gdcporsity}}
\label{tab:porousgdcfit}
\begin{tabular}{lcc}
Sample Set & $\chi^2$  & $\chi^2$   \\
& (Fixed $\eta$) & (Variable $\eta$) \\
\hline
GDC PMMA \temp{650}                 & 3.406       & 0.062           \\
GDC Graphite \temp{650}             & 0.668       & 0.008           \\
GDC Graphite \temp{25}              & 3.575       & 5.063
\end{tabular}
\end{table}

The deviance from the fit curve for the fixed \texteta{} method is quite large relative to the variable \texteta{} fit for both sample sets tested at \temp{650}.
Both fitting methods are fairly poor in the case of the room temperature graphite set.
The poor fitting results in this case is due to the
\textgreater{}60\% porosity samples.
The flatter pore geometry caused by the graphite likely leads to comparatively greater strength loss at high porosity due to likely greater pore connectivity.
The comparison between the high temperature data sets is the most useful, given this, and the fact that SOFCs operate at this temperature.
The poor fit of fixed \texteta{} models suggests that a material-specific \texteta{} value may be an oversimplification of the porosity-stress relationship in ceramics.

Samples tested at higher temperatures displayed a slightly higher flexural strength.
Generally, it is expected that increasing temperature lowers strength of dense ceramics once plastic deformation processes become activated at high temperatures.\cite{Davidge1970}
Below the temperature at which plastic deformation starts to occur, temperature has a minimal effect on flexural strength of dense ceramics.
For dense GDC it has been demonstrated that the strength decreases by 19\% between room temperature and \temp{800}.\cite{Mogensen2000}
Porosity changes this behavior, by essentially creating a composite material, keeping the rate of change constant or increasing with temperature.\cite{Giraud2008}
For porous GDC, thermal expansion places a compressive stress on any surface flaws that are sites for crack initiation and propagation.
Additionally, the porosity helps reduce bulk stresses caused by thermal expansion.
These two effects increase the material's fracture strength more than any decrease from plastic deformation at \temp{625}.

Cross-sectional SEM of the fracture surfaces of the porous GDC samples,
both PMMA and graphite, is shown in Figure \ref{fig:GDCPoreSEM}.
Figure \ref{fig:GDCPoreSEM}a shows individual spherical pores left by the PMMA while Figure \ref{fig:GDCPoreSEM}b has an inter-connected network of long pores left by the graphite flakes.
\begin{figure}
    \includegraphics[width=\textwidth]{GDCPoreSEM.jpg}
    \caption{SEM micrographs of porous GDC bars made using: a) PMMA pore former; b) graphite flake pore former}
    \label{fig:GDCPoreSEM}
\end{figure}

The differences in quality of fit between Figure \ref{fig:gdcporsity}a and Figure \ref{fig:gdcporsity}b shows that the geometric factor is microstructurally dependent.
Samples with porosity formed using PMMA spheres showed significantly greater strength as compared to samples with graphite-formed porosity.
This can be explained by the effect of pore geometry on crack initiation and propagation in the ceramic.
If a crack enters a pore, the pore can now be considered the new crack tip.
The energy required to advance the crack is highly dependent on the geometry of the tip.
For a spherical pore, this geometric factor is maximized and results in higher resilience to fracture.
This is summarized by Equation \ref{eq:cracktip} where \(\sigma_{o}\) is the stress at the crack tip, c is the length of the crack tip and \(\rho\) is the radius of curvature of the crack tip.\cite{Carter2007}
\begin{equation}
    \sigma_{f} = \frac{\sigma_{o}}{2}\sqrt{\frac{\rho}{c}}
    \label{eq:cracktip}
\end{equation}

Based on the mechanical behavior of these samples, porous ceramics should be designed and constructed such that the pore geometry is as low aspect ratio as possible to maximize strength.

\subsection{Flexural Stress Orientation}

Figure \ref{fig:NiGDCHalfCells} shows the measure flexural strengths of a number of SOFC coupon samples.
Anode support coupons were tested along with half-cells consisting of anode supports and electrolyte layers at temperatures ranging from \temp{25} to \temp{650}.
The highest temperature test was repeated with coupons under reducing conditions (3\% H\textsubscript{2}, balance Ar).
\begin{figure}
    \includegraphics[width=0.8\textwidth]{NiGDCStrengthTemp.jpg}
    \caption{Temperature dependent strength of Ni-GDC anode supports and half-cells in both air (filled data points), and reducing atmosphere (3\% H\textsubscript{2}, balance Ar, hollow data points), tested in both "electrolyte-up" and "electrolyte-down" orientations resulting in the electrolyte layer experiencing tension and compression, respectively}
    \label{fig:NiGDCHalfCells}
\end{figure}

All unreduced sample types showed increased strength at elevated temperatures with little difference between types at a given temperature.
The high-temperature reduced coupons displayed large differences in strength depending on the orientation of the sample.
Tests in which the dense electrolyte layer was placed in compression resulted in the highest strength values, while the samples were weakest when the electrolyte was placed in tension.
In reduced samples, the anode support layer becomes a ceramic-metal composite and is therefore somewhat elastic while the electrolyte remains a brittle ceramic.
The electrolyte-in-compression condition maximizes the mechanical performance of the coupon by placing the layers in their preferred stress state.

Figure \ref{fig:ASLDirection} shows the box plots for the unreduced SOFC coupon samples.
A Student's t-test was used to statically determine if the means of samples tested at different conditions were equal based on 95\% confidence.
There is no discernible difference in the strength between the three sample types at each temperature.
At elevated temperature there was a significant difference in the modulus of ASL only samples and samples with electrolyte in tension with a p-value of 0.0091.
This difference was not present at room temperature.
As the materials reach elevated temperature, small differences in elasticity become magnified due to different thermal effects on dense and porous layers.
\begin{figure}
    \includegraphics[width=0.8\textwidth]{ASLDirection.jpg}
    \caption{Flexural test measurements of coupons sample sets: a) Strength at \temp{25} b) Strength at \temp{650} c) Modulus at \temp{25} d) Modulus at \temp{650}}
    \label{fig:ASLDirection}
\end{figure}

Additionally, SEM analysis of the fractured half-cells showed very good adhesion between layers (Figure \ref{fig:HalfCellSEM}).
Delamination is a common failure mode in layered ceramics and one that would be particularly damaging to SOFCs due to resulting ionic conductivity loss between layers.\cite{Sevecek2016}
In the half-cell coupons, it was clear that the fracture plane contained mixed transgranular and intergranular fracture.
Some grains were sheared through while others remained whole.
The striations visible in Figure \ref{fig:HalfCellSEM} are characteristic of fracture proceeding through a grain, while other grains remained whole.
\begin{figure}
    \includegraphics[width=0.8\textwidth]{HalfCellSEM.jpg}
    \caption{SEM micrographs of unreduced and reduced half-cell fracture surface: tested in air at \temp{25} (top) and tested in reducing atmosphere at \temp{650} (bottom), showing good adhesion between anode and electrolyte}
    \label{fig:HalfCellSEM}
\end{figure}

\subsection{Reduction and Strength}

Mass loss of SOFC half-cell coupons exposed to reducing atmosphere at various elevated temperatures is shown in Figure \ref{fig:NiGDCReduction}.
Mass loss is attributed to the reduction of NiO, used as a precursor in fabrication,
to Ni metal, which serves as the catalyst for fuel oxidation and electronic conductor, and the reduction of ceria to \ce{CeO_{2-\textdelta}}.
Ni-GDC/GDC half cells showed the expected trend of increasing reduction rate at higher temperatures.
Reduction curves were fit to an exponential decay and the summary of parameters is shown in Table \ref{tab:NiGDCRedFit}.
At \temp{650} and \temp{700} the reduction occurs very quickly,
reaching steady state values after 18 hours. \temp{550} showed a much slower mass loss than even \temp{575}.
At temperatures lower than \temp{550} the kinetics are slow enough to tolerate a brief exposure to oxygen without re-oxidizing the sample.
Of note in the reduction is that each temperature appears to approach a different asymptote, showing that the amount of NiO and GDC reduced at steady state is dependent on the temperature of the cell.
This will affect the mechanical properties of the cells as it changes both the porosity and amount of nickel metal in the samples.

\begin{figure}
    \includegraphics[width=0.8\textwidth]{NiGDCReduction.jpg}
    \caption{Thermogravimetric analysis curves for Ni-GDC/GDC half-cells showing mass loss over time at temperatures ranging from \temp{550} to \temp{750} in 3\% H\textsubscript{2} 3\% H\textsubscript{2}O balance N\textsubscript{2}, \SI{50}{sccm} flow.}
    \label{fig:NiGDCReduction}
\end{figure}

\begin{table}
\centering
\caption{Summary of fit parameters for reduction of NiO-GDC/GDC half-cell coupons under 3\% H\textsubscript{2}, 3\% H\textsubscript{2}O, 94\% N\textsubscript{2} at different temperatures.}
\label{tab:NiGDCRedFit}
\begin{tabular}{ccc}
Temperature (\temp{}) & Exponential Rate (\SI{}{\per\hour}) & Asymptote (Mass \%)  \\
\hline
550                                    & 0.0463                  & 89.78                 \\
575                                    & 0.135                   & 89.09                 \\
600                                    & 0.182                   & 88.36                 \\
650                                    & 0.512                   & 87.75                 \\
750                                    & 0.821                   & 87.87
\end{tabular}
\end{table}

Ni-GDC/GDC fuel cells are sensitive to re-oxidation once reduced.
Re-oxidation causes fracture of cells due to the large volume difference between NiO and Ni.\cite{Nakajo2012}
Based on the results from the TGA, the reaction rates below \temp{300} are sufficiently slow to allow for the exposure of a reduced cell to oxygen without detrimental results.\cite{Richardson2003}
To confirm this a sample was measured by TGA to ensure no mass gain during oxygen exposure and the strength of samples were tested to ensure there was no discernible difference between reduction methods.

For TGA analysis, a piece of a half cell was heated to \temp{650} with a \SI{10}{\celsius\per\minute} ramp rate and reduced.
It was then cooled while still under reducing atmospheres.
Once below \temp{100}, it was exposed to simulated air
(21\% O\textsubscript{2}) for 18 hours, placed back into reducing atmosphere, and heated back up to \temp{650}.
Figure \ref{fig:rereduction} shows the change in mass overlaid onto the temperature and oxygen partial pressure experienced by the sample.
Following this treatment, the cell showed no mass gain during the oxygen exposure and continued to reduce at the same rate as before once it was returned to the initial conditions.
This shows that cells which are cooled appropriately do not re-oxidize and could be handled in between a batch reduction of cells, and their assembly into a stack configuration.

\begin{figure}
    \includegraphics[width=0.8\textwidth]{Rereduction.jpg}
    \caption{Thermogravimetric analysis of Ni-GDC/GDC half-cell showing no mass gain after reduction and exposure to room temperature and simulated air: a) Mass loss of interrupted reduction with exposure to ambient condition; b) Temperature of sample during cycle c) Oxygen partial pressure measured at the sample during cycle}
    \label{fig:rereduction}
\end{figure}

Box plots of flexural strength and modulus of the in-situ reduced and batch reduced coupons are shown in Figure \ref{fig:reductionmethod}.
Mechanical strength of half-cell coupons which had been reduced in-situ with an 18 hour reduction time showed no difference in strength when compared with coupons which had been previously batch reduced, cooled, and reheated under reducing environments.
Both of these sample types showed a dramatic decrease in strength and Young's modulus compared to the un-reduced samples, but no statistically significant difference between treatments.
The decrease in strength and modulus is due to the increase in porosity and conversion of NiO to metallic nickel in the cells.
The decrease in variation between measurements in reduced cells is likely due to the increased number of large voids that form during NiO reduction.

\begin{figure}
    \includegraphics[width=0.8\textwidth]{ReductionMethod.jpg}
    \caption{Flexural properties, a) strength and b) modulus, of Ni-GDC/GDC half-cell coupons after reduction via two different methods, compared to unreduced cells. Strength and modulus show decrease upon reduction but no significant difference between methods.}
    \label{fig:reductionmethod}
\end{figure}

The statistical similarity of the properties of the samples reduced via each method indicates that planar, Ni-GDC/GDC based SOFC ASL and half-cells are able to be safely reduced, cooled, and handled in ambient conditions without leading to damaging re-oxidation of the nickel anode material.
This resilience could enable some degree of large-batch reduction of cell anodes prior to SOFC stack assembly, leading to more rapid startup and a greater degree of sealing control.

\section{Conclusions}

A temperature and atmosphere controlled three-point bend fixture was designed and built for use in a universal testing machine.
SOFC coupons and component materials were evaluated for flexural strength at room temperature and IT-SOFC operating temperatures.
In addition, the effects of porosity percent and pore geometry on flexural strength in GDC were investigated.
Furthermore, the impact of temperature on the reduction rate of NiO in NiO-GDC SOFC anodes was examined along with the resilience to re-oxidation at ambient conditions of this SOFC component.

Pore geometry had a significant impact on the flexural strength of
GDC10, with spherical pores showing the greatest resistance to fracture.
This supports the concept of pores acting as the new crack tip once a crack has advanced to the pore.
Additionally, samples tested at \temp{650} were stronger than those at room temperature.
This is likely due to localized compressive stresses from thermal expansion of the material.
This hypothesis is further supported by the results of testing NiO-GDC
anode support coupons and half-cells.
Coupons tested at \temp{25}, \temp{450}, and \temp{650} displayed a linear strength dependence with temperature.
There was no statistical difference in strength between anode support layers and half-cells composed of anode support and electrolyte at a given temperature in air.
Half-cells in which NiO was reduced to Ni by exposure to H\textsubscript{2} at \temp{650} displayed significant differences in strength when the electrolyte layer was subjected to compressive stress as opposed to tensile stress.
Placing the ceramic electrolyte in compression and the metal-ceramic composite anode in tension resulted in the highest strength.

The reduction temperature of NiO-GDC/GDC half-cells was shown to have an effect on the rate of NiO reduction and amount of NiO reduced.
At lower temperatures, the oxidation rate of Ni-GDC is slow enough that the anode can be exposed to air for significant periods below \temp{100}.
Any re-oxidation, combined with the cooling and re-heating of a cell back to
\temp{650}, showed no effect on the mechanical properties when compared to cells which had been reduced in-situ at \temp{650}.
These results indicate that it is possible to reduce and cool cell components to an extent without any additional effects to mechanical properties, allowing for more flexibility during cell manufacturing, stack assembly, and with quality control screenings.

This work leads to three important conclusions for the mechanical properties of GDC-based SOFCs using Ni anodes.
Porous GDC used in anode supported SOFCs should be fabricated such that the pore geometry is spherical as this maximizes energy required to advance a crack through the ceramic.
Care should be taken in stack construction to ensure any out-of-plane cells are placed to compress the electrolyte and place the reduced anode in tension so as to lower the chance of fracture.
Finally, it is possible to reduce the anodes of Ni-GDC SOFCs and then handle them at ambient conditions for quality control and stack assembly.
This will remove a degree of variability from the manufacture of cells and stacks.
