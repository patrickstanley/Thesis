% !TEX root = ../mainthesis.tex

%Chapter 2

%\renewcommand{\thechapter}{2}

\chapter{Experimental Procedures}

\section{Sample Preparation}

    \subsection{\ce{SrFe_{0.2}Co_{0.4}Mo_{0.4}O_{3-\delta}} Powder}
        \ce{SrFe_{0.2}Co_{0.4}Mo_{0.4}O_{3-\delta}} (SFCM) was created from stoichiometric amounts of strontium carbonate (\ce{SrCO_3}, Sigma-Aldrich), iron oxide (\ce{Fe_2O_3}, Sigma-Aldrich), cobalt oxide (\ce{Co_2O_3}, Inframat Advanced Materials), and molybdenum oxide (\ce{MoO_3}, Alfa-Aesar) using conventional solid-state methods.
        The constituents were ball milled for 24 hours in ethanol and dried using a \SI{100}{\celsius} oven.
        Afterwards the powder was calcined at \SI{1100}{\celsius} for four hours.

    \subsection{Bars}
        GDC bars of varying porosity were fabricated using uniaxial pressing in a rectangular die.
        GDC10 (\ce{Ce_{0.9}Gd_{0.1}O_{2}}) powder was mixed with the desired volume of poly(methyl methacrylate) (PMMA) microspheres (\SI{5}{\micro\meter} diameter) or graphite flake to create the green body, with 0.1wt\% of polyvinyl butyral added to aid in pressing and a drop of fish oil to act as a dispersant during milling.
        The two different pore formers were used to create differing pore geometry.
        The PMMA or graphite was removed by a \temp{400} pre-sintering step, and the remaining ceramic structure was sintered at \temp{1500} for 4 hours.
        Archimedes density measurements were used to confirm porosity percent.
        Samples were polished with 600 grit sandpaper on all sides before testing.

        Dense SFCM bars were used to maximize the total mass and mass changes during thermogravimetric testing.
        Samples were made by combining SFCM powder with 0.6 wt\% polyethylene glycol 600, 1.8 wt\% ethylene glycol, and 0.6 wt\% glycerol in isopropyl alcohol and ball milling overnight.
        After drying at \SI{100}{\celsius}, the powder was ground by mortar and pestle, then pressed uniaxially into rectangular bars at \SI{30}{\mega\pascal}, then isostatically pressed at \SI{30}{\mega\pascal}.
        Bars were sintered by heating to \SI{400}{\celsius} for one hour and then \SI{1340}{\celsius} for four hours, using a \SI{3}{\celsius\per\minute} heating and cooling rate.
        This produced bars with 97\% theoretical density by Archimedes' technique.

        To create fracture toughness test samples, SFCM powder was ball milled overnight with a mixture of 0.6 wt\% polyethylene glycol 600, 1.8 wt\% ethylene glycol, and 0.6 wt\% glycerol in isopropyl alcohol.
        It was then dried at \SI{100}{\celsius}, ground by mortar and pestle, pressed uniaxially into rectangular bars at \SI{30}{\mega\pascal}, then isostatically pressed at \SI{30}{\mega\pascal}.
        Sintering followed by heating to \SI{1340}{\celsius} for four hours at \SI{3}{\celsius/min}  with a one hour hold at \SI{400}{\celsius} to allow the binders to burn out.
        This process achieved dense samples at 97\% average theoretical density with no apparent flaws in the bars.
        Bars were then cut and sanded to final dimensions of \SI{3}{mm} x \SI{4}{mm} x \SI{25}{mm}.
        The chevron notch was cut using the jig described by Jenkins, Chang and Okura following the ASTM procedure.\cite{Jenkins1988, ASTM2016a}

    \subsection{Test Coupons}
        The ASL and half-cell coupons used in the flexural testing were made using tape casting.
        NiO and GDC powders in a 6:4 weight ratio were mixed with ethanol, toluene, and fish oil.
        The mixture was ball milled on a rotary mill for 24 hours.
        Polyvinyl butyral and benzyl butyl phthalate were added, followed by another 24 hours milling.
        The resulting slurry was degassed and tape cast using a 700 micron blade height.
        The tapes were left to dry overnight before being cut into \SI{12}{\centi\meter} by \SI{12}{\centi\meter} squares.
        Three squares were stacked and hot pressed at \temp{49} and 2 tons for 30 minutes.
        Following this lamination, the tape was cut into rectangular coupons measuring 25mm by 10mm.
        The coupons were fired at \temp{1400} for 4 hours.
        Final bar dimensions were 8.16 by 24.15 by 2.98 mm on average.
        For half-cell coupons GDC electrolyte slurry was prepared and tape cast with a \SI{40}{micron} blade height.
        After the lamination of the three ASL layers, a single layer of electrolyte tape was laminated to the ASL using the same pressure for 2 hours.
        The sintering procedure for half-cell coupons was identical to the ASL coupons.
        Samples were polished with 600 grit sandpaper on edges before testing.

        Tape casting was used to create test coupons of porous SFCM-GDC ASL and SFCM-GDC/GDC half-cells.
        Using ethanol as a solvent, SFCM-GDC was ball milled with polyvinyl butyral, benzyl butyl phthalate, \SI{12}{\micro\meter} poly(methyl methacrylate) (16 wt\% with respect to SFCM-GDC), and Menhaden fish oil.
        The tape was cast to a thickness of \SI{110}{\micro\meter} on Mylar then laminated using a hot press to a final thickness of \SI{660}{\micro\meter}.
        Dense GDC was casted to \SI{30}{\micro\meter} and laminated to the top of the SFCM-GDC to create half-cells.
        Individual coupons were cut from the green tape, sintered at \SI{1200}{\celsius} for four hours, with holds at low temperatures to burn out organic binder and pore former.
        The final thicknesses were measured to be \SI{400}{\micro\meter} for the SFCM-GDC ASL and \SI{\sim 20}{\micro\meter} thick GDC electrolyte.
        Test coupons had their edges sanded to remove defects left from cutting, following ASTM standards.\cite{ASTM2008}
        Top and bottom surfaces were not sanded to preserve possible defects left from tape casting procedure, which would be representative of industrially manufactured SOFCs.

\section{X-Ray Diffraction}
    X-ray diffraction (XRD) was used confirm phase purity of the SFCM after synthesis and during the testing process.
    A Bruker D8 Advance with LynxEye was used with a Cu K\textsubscript{\textalpha{}} source.
    A step size of \SI{0.02}{\degree} was used with a dwell of \SI{0.8}{\second} was used.
    Rietveld refinement was performed on samples as synthesized, after oxidation in pure \ce{O_2} and after reduction in pure \ce{H_2} at \temp{600} for both.
    GSAS-II was used to perform the refinement calculations and VESTA was used to visualize the unit cell.\cite{Toby2013,Momma2011}

\section{Archimedes Density Measurements}

\section{Thermogravimetric Analysis}
    Mass of samples during oxidation and reduction cycling was measured using a Cahn D200 microbalance.
    The samples were placed in a crucible suspended from a platinum wire attached to the microbalance and enclosed by an alumina tube inside a furnace.
    Heating control was achieved with a PID loop and temperature measurement done by a K-type thermocouple placed immediately below the sample inside the alumina tube.
    Gas flow was controlled at a constant \SI{50}{sccm} by mass flow controllers.
    21\% O\textsubscript{2} in dry N\textsubscript{2} was used for the oxidizing condition while 3\% H\textsubscript{2} in N\textsubscript{2} humidified with 3\% H\textsubscript{2}O was used for the reducing conditions.

    To understand the degree of reduction in test coupons, thermogravimetric analysis (TGA) was used to measure the mass loss as a function of time,
    temperature, and gas environment.
    A Cahn D200 microbalance was used to measure the weight changes as the sample was heated in a furnace with controlled atmosphere.
    A small section of test coupon was placed in a crucible suspended from the balance and heated to \temp{650} at \SI{10}{\celsius\per\minute}, the environment was switched from \SI{50}{sccm} of 21\% O\textsubscript{2} in
    N\textsubscript{2} to \SI{50}{sccm} of 3\% H\textsubscript{2}, 3\%
    H\textsubscript{2}O in N\textsubscript{2}.
    Mass, temperature, and
    \po2{} measurements were taken at 30 second intervals.
    \po2{} measurements were taken using a calibrated YSZ sensor placed immediately before the sample.
    Samples were cooled at \SI{10}{\celsius\per\minute} in the desired atmosphere.

    Changes in mass of SFCM were measured by a Cahn D200 microbalance with the sample suspended down a quartz tube into a furnace.
    The furnace used to heat the sample and quartz tube was controlled by a PID controller with a K-type thermocouple placed immediately below the sample inside the quartz tube.
    Gas flow was controlled at consistent \SI{50}{sccm} by mass flow controllers, which mixed dry nitrogen, oxygen, hydrogen, and humidified nitrogen to obtain the \po2{} desired.
    Measurements of the \po2{} were taken by a calibrated YSZ oxygen sensor at \temp{800} located before the sample.
    Again, intermediate \po2{} ranges were not able to be tested due to SFCM's incompatibility with the species required to create that environment.

    Further details of the operation of the TGA and its various components are given in Appendix \ref{app:TGA}.

    To prepare the sample, it was pre-weighed, wrapped in platinum wire and suspended from the balance, placed in the furnace with simulated air (21\% O\textsubscript{2}, 79\% N\textsubscript{2}) flowing.
    Once the mass had stabilized, the furnace was heated to \SI{800}{\celsius} to allow for any organic contamination to burn off.
    After the mass stabilized at this elevated temperature the sample was introduced to various environments.
    The mass of the sample would be noted only after steady state had been reached for a condition.
    After testing a sample, a bar of alumina was cut to the same dimensions as the sample and the process was repeated to obtain a blank which could be subtracted from the measurements to remove buoyancy effects.

    Oxygen non-stoichiometry was calculated using Equation \ref{eq:TGA}, where $\Delta\delta$ is the change in oxygen stoichiometry, $MW_{SFCM}$ is the molecular weight of SFCM (\SI{208.74}{\gram\per\mol}), $MW_O$ is the molecular weight of oxygen (\SI{16.0}{\gram\per\mol}), $w_{sample}$ is the weight of the sample, and $\Delta{}w$ is the weight change as recorded by the TGA.
    The oxygen vacancy concentration ($\lbrack\ch{V_O^{**}}\rbrack$) is calculated using Equation \ref{eq:TGAtoV}, where $\rho$ is the density of SFCM and $N_A$ is Avogadro's number.
    To calculate the oxygen vacancy concentration, the non-stoichiometry of SFCM at a point needs to be established.
    For this work, based on the plateau present in the data at oxidizing conditions, it is assumed that in a pure oxygen environment ($log(\po2{})=0$) all oxygen vacancies are filled with no oxygen interstitial or surface species, thus $\delta=0$.
    \begin{equation}
        \Delta\delta = \frac{MW_{SFCM}}{MW_O\ w_{sample}}\Delta{}w
        \label{eq:TGA}
    \end{equation}
    \begin{equation}
        \lbrack\ch{V_O^{**}}\rbrack =\frac{\delta\rho N_A}{MW_{SFCM}}
        \label{eq:TGAtoV}
    \end{equation}

\subsection{Temperature Programed Desorption}
    The effluent from the TGA was used as the inlet to a mass spectrometer (MS) to perform temperature programmed desorption.
    The sample was prepared as before and heat treated to remove any carbon contaminants but was allowed to cool under simulated air.
    It was then heated to \SI{800}{\celsius} at \SI{5}{\celsius\per\minute} under a \SI{50}{sccm} flow of nitrogen as the MS measured the \SI{32}{m/z} signal which corresponded to O\textsubscript{2} desorption.
    Additional m/z signals were monitored to observed for other species.

\section{Conductivity}
    Electric conductivity across \po2{} was measured using the four-wire technique and a Stanford SR 830 lock-in amplifier.
    A bar shape sample was used with dimensions of \SI{6.46x3.3x1.3}{\milli\meter}.
    Gold paste was used as a current collector, and the current range was between \SI{0.005} to \SI{0.05}{A}.
    A yttria-stabilized zirconia (YSZ) oxygen sensor operating at \temp{800} was used to monitor the changes in oxygen partial pressures.
    Intermediate \po2{} ranges were not tested due to an incompatibility between SFCM and the \ce{CO} and \ce{CO_2} required to obtain those \po2{}.

    Conductivity during redox cycling was measured using rectangular bars of SFCM and SFCM-GDC (2:1) composite.
    The samples were connected to a Keithley 2400 source meter by silver paste and wire.
    Using an in-house built reactor, the sample could be heated and the gas environment could be controlled.
    An initial measurement was taken after heating and 50 hours of exposure to 10\% H\textsubscript{2} in N\textsubscript{2}, then the sample exposed cycled between air and reducing conditions over a period of 14 days.

\section{Mechanical Testing}
    Measurements of mechanical properties were collected using a Tinius Olsen 10ST Universal Testing Machine equipped with a \SI{250}{N} load cell.
    Experiments where any samples would be tested at elevated temperatures or under reducing environments were conducted using a custom built three point flexural test fixture placed inside a gas-tight chamber and furnace.
    Otherwise, a fully-articulating four point flexural test fixture was used.
    Samples tested at ambient conditions were placed on the appropriate testing fixture and loaded until failure.
    For samples tested at elevated temperatures the sample was heated in the test chamber at \SI{10}{\celsius\per\minute}, allowed to equilibrate for 20 minutes, then tested.
    Samples to be reduced were placed in the chamber, heated and exposed to reducing gas for 18 hours before being tested.

    All tests conducted in the UTM (Tinius Olsen 10ST with 250 N load cell) were done at a rate of \SI{0.2}{\milli\meter\per\minute} with a \SI{20}{\milli\meter} lower span.
    After measuring the force at fracture, stress was calculated using Equation \ref{eq:3ptstrength}
    for 3-point flexural of rectangular samples, where \(\sigma_{f}\) is the stress at failure, F is the load at failure, L is the span of the fixture, b is the width of the sample and d is the thickness of the sample.
    At room temperature, the coupons or bars were loaded into the fixture and tested in batches of 5 per condition.
    Occasionally samples would break or be damaged before testing, reducing the sample set.
    For sample sets at elevated temperatures in ambient atmosphere, the samples were loaded into the front of the chamber, acting as a staging area,
    prior to heating the on fixture in the center.
    Upon transferring the next coupon from the staging area to the fixture, a 20 minute waiting period was used to ensure that the coupon had reached thermal equilibrium prior to testing.
    \begin{equation}
        \sigma_{f} = \frac{3FL}{2bd^{2}}
        \label{eq:3ptstrength}
    \end{equation}

    Following the completion of the set, all coupon pieces were cooled at a rate of \temp{10}/min.
    For testing in reducing atmosphere, each coupon was reduced and tested individually.
    Half-cell coupons were tested in two orientations, "electrolyte-up" and "electrolyte-down." These orientations cause the dense electrolyte layer to experience compressive or tensile stress.
    The "electrolyte-up" orientation subjects the electrolyte to compressive stress and vice-versa.

    Following the destructive tests, SEM analyses of the fracture surfaces were performed.
    The purpose of this was to observe the trans-granular,
    inter-granular, or mixed nature of crack propagation, and to find any anomalous features on the fracture surface, as well as to observe pore geometry.

    \subsection{Development of Mechanical Test Apparatus}
        To develop a testing apparatus capable of simulating the various conditions experienced by operating SOFCs, appropriate materials were chosen based on thermal and chemical stability criteria.
        Alumina was chosen to build the bend fixture due to its chemical stability and high hardness.
        The sample rests crossways on two stationary \SI{6.35}{\milli\meter} diameter rods which are placed in troughs separated by \SI{20}{\milli\meter}.
        The upper half of the fixture consists of a \SI{6.35}{\milli\meter} rod used to apply stress to the sample from the UTM crosshead.
        Both pieces are attached using silica-based cement to \SI{300}{\milli\meter} rods attached at the anchor points of the UTM.
        Assembly and alignment is assisted with the use of a 3D printed jig to ensure repeatability.
        The complete alumina, 3-point bend fixture is seen in Figure \ref{fig:3ptbend} with the two bottom rollers, a sample, and the top roller making contact, applying flexural stresses to the sample.

        \begin{figure}
            \includegraphics{3ptbend.jpg}
            \caption{Assembled Alumina 3-Point Bend Fixture}
            \label{fig:3ptbend}
        \end{figure}

        To create the atmosphere control system, a combination of standard and custom vacuum system parts were used to enclose the bend fixture.
        For the main body of the chamber, a 3-inch inner diameter, stainless steel tee was made with QF80 and QF50 flanges.
        Welded NPT ports on top and bottom allow for gas inlet and outlet and the introduction of a thermocouple through a Swagelok Ultra-Torr fitting.
        Figure \ref{fig:chamber} shows the engineering drawings and photo of the custom pieces from A\&N Corp.
        Flexible bellows allow for the motion of the cross-bar to be translated into the fixture, requiring the subtraction of the spring force to be removed during analysis.
        The alumina fixture was then inserted through each end and the chamber incorporated into the furnace with QF50 to 1/2 inch tube adapters that fit over the alumina rods.

        \begin{figure}
            \includegraphics{chamberdrawing.jpg}
            \includegraphics{chamberpic.jpg}
            \caption{Design drawing and photo of mechanical testing atmospheric chamber}
            \label{fig:chamber}
        \end{figure}

        To heat the fixture and chamber, a 1 ft. cube was assembled from steel plates with cutouts on top, bottom and front.
        Steel wire mesh was used to create a framework inside to hold nickel-chromium alloy heating elements.
        Silica-based wool insulation was packed between the center cavity and the case walls.
        A K-type thermocouple was inserted into the chamber and used with an external PID controller to cycle the heating elements.
        The furnace was placed on a supporting scaffold to hold it in position during tests.
        The test fixture, atmosphere chamber, and furnace were affixed together as one unit and inserted into the UTM.
        The design was compared and qualified to a regular steel 3-point bend apparatus to ensure accurate results at room temperature.

    \subsection{Atmospheric Treatment}
        To test individual coupons under reducing environments, samples were loaded into the test chamber at \temp{400}, after previous pre-heating in the staging area.
        After 20 minutes, the temperature would be increased to the desired set-point of \temp{650} at a ramp rate of \SI{10}{\celsius\per\minute} and the gas would be switched first to N\textsubscript{2} to flush the chamber, then to humidified argon containing 3\% H\textsubscript{2}.
        It would then be allowed to sit for 18 hours before testing to allow reduction.
        After testing, the chamber would be flushed again with N\textsubscript{2} and cooled to \temp{400} before opening to change samples and repeat the process.

        If samples had been previously reduced as a batch by exposure to hydrogen in a tube furnace and cooled under hydrogen, the atmospheric treatment was shortened, as supported by TGA results.
        Pre-reduced samples were loaded into the chamber at or below \temp{400}, the chamber flushed N\textsubscript{2}, then with H\textsubscript{2} in Ar.
        After this the chamber would be heated to \temp{650}, the sample tested after a 30
        minute period, and cooled back down to \temp{400}.
        At this point, the chamber would be flushed with N\textsubscript{2}.
        While the chamber maintained a positive pressure of N\textsubscript{2}, it could be opened, a new sample loaded, closed, and allowed to flush.
        This procedure allows for the batch reduction of many test coupons and replaced the 18 hour reduction time with a shorter 30 minute period.

    \subsection{Fracture Strength}
        A loading rate of \SI{0.2}{mm\per\minute} was used for all strength measurements.
        Strength was calculated from the maximum force measured before failure according to Equation \ref{eq:4ptstrength} or \ref{eq:3ptstrength} depending on if the four point or three point fixture is used respectively, where $\sigma{}$ is the strength, \textit{P} is the maximum force, \textit{L} is the span width of the test fixture, \textit{b} is the width of the sample and \textit{d} is the thickness of the sample. Equation \ref{eq:4ptstrength} is for a fixture where the top span is 1/2 the width of the bottom span.\cite{ASTM2008}
        \begin{equation}
            \sigma = \frac{3PL}{4bd^{2}}
            \label{eq:4ptstrength}
        \end{equation}

    \subsection{Fracture Toughness}
        The fracture toughness of chevron notched samples were measured using a loading rate of \SI{0.001}{mm\per\minute}.
        Fracture toughness was calculated from the maximum force using Equation \ref{eq:4ptkivb} or \ref{eq:3ptkivb} for four point or three point fixtures.\cite{ASTM2016a,Wu1984}
        $Y^{*}_{min}$ is the shape factor as calculated by Equation \ref{eq:shapefactor}, $S_o$ and $S_i$ are the outer and inner spans, \textit{B} is the width of the sample, \textit{W} is the height of the sample, $a_0$ is the distance from the tip of the chevron notch to the bottom of the sample, and $a_1$ is the average distance from the side of the chevron notch to the bottom of the sample.
        Each sample was measured after failure, but in this study the approximate values were $S_o = 40 mm$, $S_i = 20 mm$, $B = 3.0 mm$, $W = 4.0 mm$, $a_0 = 0.80 mm$, $a_1 = 3.8 mm$.
        Fracture toughness was measured only under air at room temperature and up to \SI{600}{\celsius}.
        Under reduction the fracture toughness bars would spontaneously fracture, preventing measurements under that condition.
        \begin{equation}
            K_{Ivb} = Y^{*}_{min}  \left [\frac{P[S_o-S_i]}{BW^{3/2}}\right ]10^{-6}
            \label{eq:4ptkivb}
        \end{equation}
        \begin{equation}
            K_{Ivb} = Y^{*}_{min}  \left [\frac{P}{BW^{1/2}}\right ]10^{-6}
            \label{eq:3ptkivb}
        \end{equation}
        \begin{equation}
            Y^{*}_{min} = \frac{0.38742-3.0919(a_0/W)+4.2017(a_1/W)-2.3127(a_1/W)^2+0.6379(a_1/W)^3}{1.0000-2.9686(a_0/W)+3.5056(a_0/W)^2-2.1374(a_0/W)^3+0.0130(a_1/W)}
            \label{eq:shapefactor}
        \end{equation}

    \section{Scanning Electron Microscopy}
