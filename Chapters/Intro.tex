% !TEX root = ../mainthesis.tex

%Chapter 1

%\renewcommand{\thechapter}{1}

\chapter{Introduction}

\section{Motivation for Solid-Oxide Fuel Cell Research}
    The world depends on fossil fuels for its daily energy needs.
    This is a fact that is going to stay with us for the foreseeable future.
    By 2040, it is estimated that fossil fuels will still supply  78\% of total energy demand.\cite{U.S.EnergyInformationAdministration2016}
    Traditionally, the annual increase in demand is met by an increase in supply of fuel.
    This increase is supplied by developments in mining and drilling operations, although recent advances have been found to be controversial, such as hydraulic fracturing.\cite{Osborn2011,Vengosh2014}
    These concerns do not even account for the increased emissions caused from combustion of fossil fuels and their effect on the global environment.\cite{Solomon2009,Hansen1981,U.S.EnvironmentalProtectionAgency2017}

    An alternative to increasing the supply and use of fossil fuels is to increase the efficiency of converting the fuel into useful work.
    Combustion generators convert chemical energy to thermal energy, then to mechanical energy, and finally into electrical energy.
    Each energy conversion step has intrinsic losses which limit efficiency.
    \Glspl{sofc} have the ability to solve this for applications where electrical power is desired.
    \glspl{sofc} allow for side stepping of these losses by allowing for the direct conversion of chemical energy to electrical energy.
    Power generation plants have efficiencies around 30\%, while a stand-alone \gls{sofc} generator can convert fuel to electricity at 45 to 65\% efficiency.\cite{Wachsman2011a,Lasseter2004}
    \glspl{sofc} are able to run on a variety of fuel sources, such as hydrogen, methane or even biogas.\cite{Park2000}
    This fuel flexibility and higher efficiency positions \glspl{sofc} to bridge the gap in the energy economy as it transitions from fossil fuels to renewable sources.
    However, many technical hurdles still remain which currently limit \glspl{sofc}' viability and must be addressed.

\section{Solid-Oxide Fuel Cell Operation}
    The layout for a hydrogen-fueled \gls{sofc} is presented in Figure \ref{fig:sofc}.
    Fuel is exposed to the anode side of the cell, while an oxygen rich gas (usually air) is exposed to the cathode side.
    Separating the two sides is the cell itself, with anode, electrolyte, and cathode.
    The electrolyte is only conductive to oxygen ions, not electrons.
    A perfect electrolyte would have no electronic conduction, allowing no leakage current from anode to cathode and forcing all electrons to travel via an alternative route.
    At the cathode, diatomic oxygen gas will disassociate into oxygen ions, each with a 2$-$ charge from the addition of electrons from the cathode to the oxygen.
    The oxygen ions then travel though the electrolyte and react with the fuel at the anode, freeing the previously captured electrons.
    To maintain charge balance, electrons must travel in the opposite direction to the cathode.
    Because the electrolyte only allows oxygen ions to diffuse though it, the electrons travel though an external circuit performing useful work in the process.
    A limiting factor in this movement is the ability of oxygen to transport into, through, and out of the cell.
    As the rate of oxygen transport is increased, the ability for electrons to perform work also increases.

    \begin{figure}
      \centering
      \includegraphics[width=0.5\textwidth]{sofc.png}
      \caption[Diagram showing flow of materials in the operation of an \gls{sofc}.]{Diagram showing flow of materials in the operation of an \gls{sofc}.\cite{Sakurambo}}\label{fig:sofc}
    \end{figure}

    The driving force for the movement of oxygen ions and electrons comes from the chemical potential gradient across the cell.
    When the anode is exposed to fuel, it creates a low \gls{po2} in comparison to the cathode, which is exposed to air.
    This gradient drives the motion of oxygen through the cell from the high \po2{} of the cathode to the low \po2{} of the anode, and generates electric current in the opposite direction to maintain a charge balance.
    The electrical potential at open circuit conditions (voltage with no current draw) is expressed by the Nernst equation, given in Equation \ref{eq:nernst} for the reaction given in Equation \ref{eq:rxn}, where \(f_A\) is the fugacity of species A, \(E^o\) is the standard potential for the reaction, and $n$ is the number of charges involved in the reaction, \(R\) is the ideal gas constant and \(F\) is Faraday's constant.\cite{Larminie2001}
    Commonly, the fugacity is replaced by the partial pressure of the species.
    For the case of a hydrogen-fueled \gls{sofc}, Equation \ref{eq:nernst} can be simplified to Equation \ref{eq:nernsth2}.\cite{Pilatowsky2008}
    From these equations it can be seen that the partial pressures of the involved species have a large role to play in the open circuit voltage of the cell.
    It is critical that the fuel and air remain separated by the cell.
    If there is a leak either through the cell or around it, then the electric potential across the cell will decrease harming performance.
    \begin{equation}
      \label{eq:rxn}
    \ch{aA + bB  <-> cC + dD}
    \end{equation}
    \begin{equation}
      \label{eq:nernst}
    E = E^o - \frac{RT}{nF}\ln\left(\frac{f_C^c f_D^d}{f_A^a f_B^b}\right)
    \end{equation}
    \begin{equation}
      \label{eq:nernsth2}
    E = E^o + \frac{RT}{2F}\ln\left(\frac{p_{H_2} {(p_{O_2})}^{1/2}}{p_{H_2 O}}\right)
    \end{equation}

    While the electric potential is established by the partial pressure gradient, the electric current is limited by the ionic conduction of oxygen ions from the cathode to the anode.
    In ionic conduction, oxygen ions hop from site to site via oxygen vacancies, locations where an oxygen atom is missing from the site it usually occupies in the crystal lattice.
    Because conduction depends on vacancies, the more vacancies that are present in the lattice, the more available sites for oxygen to move between, the greater the conductance, and the greater power output of an cell.
    The relationship between oxygen vacancy concentration and oxygen conductance is given by Equation \ref{eq:conductivity}, where \(\sigma\subscript{i}\) is the ionic conductivity, \(\lbrack\ch{V_O^{**}}\rbrack \) is the concentration of oxygen vacancies in the lattice, \(\mu_O \) is the ionic mobility of the oxygen, and \(Z_e\) is the charge of the conducting species.\cite{Mogensen2000}
    Thus, by adding more oxygen vacancies the ionic conductivity increases.
    While additional oxygen vacancies are a benefit to the electrical performance of the cell, it can have detrimental effects to the mechanical properties.
    \begin{equation} \label{eq:conductivity}
      \sigma\subscript{i} = \lbrack\ch{V_O^{**}}\rbrack Ze\mu_O
    \end{equation}

    During operation, oxygen vacancies are created by removal of lattice oxygen from the anode as it reacts with the fuel, as shown in the half-cell reaction of Equation \ref{eq:vacancyformation}.
    This reaction takes place on the surface of the anode, but for convenience it is written where oxygen has associated back into a diatomic state and the fuel subsequently react with it.
    This reaction of fuel with oxygen creates an environment with a very low \po2{} at the surface of the anode.
    A steady state concentration of oxygen vacancies will be reached as the cell comes into equilibrium with the environment, but this concentration will depend on the exact environmental, material, and performance conditions of the cell.
    \begin{equation} \label{eq:vacancyformation}
    \ch{O_O^x  <-> 1/2 O2 + V_O^{**} + 2 e'}
    \end{equation}

    The cathode and anode materials must facilitate the incorporation of oxygen into the cell and the reaction with the fuel.
    They must also serve to transport the electrons into and out of the cell as oxygen changes state.
    As a result, the anode and cathode must posses both electrical and ionic conductivity.
    To achieve this, a mixture of the electrolyte material and an electronically conductive material are used to create these structures.
    This mixture of materials creates specific sites where the cell is active, known as \glspl{tpb} and demonstrated in Figure \ref{fig:tpb}.
    The \gls{tpb} is the location where gas, electronic conductor, and ionic conductor meet and oxygen can incorporate into the cell or react with the fuel.
    To assist gas diffusion to the \gls{tpb} and to maximize the length of \glspl{tpb}, pores are added, greatly increasing the available surface area of the electrodes.
    The more \glspl{tpb} that are present, the more exchange can occur between the gas and the cell at any given time and the current output of the cell is increased.
    This added porosity does not come without the disadvantage of decreasing the mechanical strength of the cell.\cite{Pihlatie2009,Laurencin2010}
    \begin{figure}
      \centering
      \includegraphics[width=0.5\textwidth]{tpb.png}
      \caption[Diagram showing the triple phase boundary and the importance as being the site where incorporation and reactions occur.]{Diagram showing the triple phase boundary and the importance as being the site where incorporation and reactions occur.\cite{Wachsman2011a}}\label{fig:tpb}
    \end{figure}

    An operating \gls{sofc} would consist of a series of individual cells stacked together for combined power output.
    The cells would be sealed between current collectors and interconnects and heated to operating temperature.
    The elevated temperature enables oxygen conduction though the electrolyte by increasing vacancy concentration and mobility, but decreases the theoretical voltage.
    Historically, \glspl{sofc} operated near \temp{1000}, but advances in materials now allow cells to operate at \temp{650} and development continues to decrease the operating temperature.
    At lower operating temperatures it becomes easier to create better seals and the stack can be built from more readily available materials.%addcite
    Additionally, the cells are subjected to less thermal expansion, reducing the stresses on the cell.

\section{Solid-Oxide Fuel Cell Materials}
    Each structure of an \gls{sofc} is fabricated from combinations of materials designed to optimize that part's function then laminated with the other structures to create the cell.
    For example, a cell can be comprised of a nickel metal-\gls{gdc} anode, a \gls{gdc} electrolyte, and then a \gls{lsm} cathode.\cite{Liu2002,Haile2003}
    The individual parts are cast into thin flexible sheets (tapes) from bulk powders, with additives such as pore formers where appropriate, and then laminated together and co-fired to create a single cell.
    As a result, a completed cell has distinct layers in it where the material and properties abruptly change.
    This composite structure can then have inter-diffusion between layers, smoothing out the abrupt changes, but creating new structures and compositions that were not present upon lamination.\cite{Yokokawa2008}

    The traditional material used for electrolytes in \glspl{sofc} has been \gls{ysz}.
    \Gls{ysz} is very stable, has a ratio of ionic conductivity to total conductivity (transference number) of approximately 1, and good conductivity at temperatures in excess of \temp{800} and up to very low \po2{}s.
    A high transference number minimizes leakage current due to electronic conduction in the electrolyte, maintaining cell performance.
    As mentioned previously, \gls{gdc} is another choice for an electrolyte.
    Other rare-earth doped cerias have been used as electrolytes, but \gls{gdc} tends to outperform them under a variety of conditions.
    At \temp{600}, \gls{gdc} exhibits an order of magnitude higher conductivity than \gls{ysz}, but has a smaller electrolytic domain where the transference number is near 1.\cite{Inaba1996}
    Under conditions of high temperature or very low \po2{}, \gls{gdc} will start to leak electrons through the electrolyte, reducing cell performance.
    As \gls{sofc} operating temperatures are reduced for improved efficiency, \gls{gdc} is favored for the electrolyte material.

    The anode usually consists of a mixture of the electrolyte material, which provides good ionic conductivity, with an electronic conductor that can facilitate fuel oxidation reactions on the surface.
    In combination with an ionic conductor it creates a cermet which functions as the anode.
    Nickel has commonly been used in the anode due to its high conductivity and catalytic properties.%citation would be good
    A challenge with nickel is that at ambient conditions it oxidizes into nickel oxide, but at operating conditions it reduces to the desired nickel metal.
    There is a large lattice parameter change between nickel and nickel oxide, such that, upon reduction it decreases volume by over 15\%.
    As a result, the overall porosity of the anode layer increases as the \gls{sofc} is put into service and the anode is exposed to reducing conditions.\cite{Gutierrez-Mora2002,Yu2007}
    This increased porosity can greatly decrease the flexural strength and modulus of the anode and of the overall cell.\cite{Callister2014,Barsoum2003}
    Additionally, nickel is prone to poisoning from sulfur contaminants in the fuel, so alternative anodes are of interest for use in \glspl{sofc}.

    As an alternative, other metal-ceramic systems have been researched to serve the function of the anode.
    These have included copper, cobalt, and platinum systems, but each suffer from a significant drawback such as coking, long-term performance degradation or cost.
    A different approach is to use an all-ceramic anode, made of a \gls{miec}.
    Perovskite materials have easily been developed into \glspl{miec} with the use of a transition metal occupying the octahedral B-site.
    \gls{sfcm} is a recently developed double perovskite \gls{miec} with multiple dopants on the B-site with a high conductivity of \SI{30}{S\per\centi\meter}.\cite{Pan,Hussaina,Hussain}
    Due to the all-ceramic nature of these anodes, their thermal expansion coefficients better match that of the other ceramic components of the cell, reducing sintering stresses and improving redox cycling durability.
    Conversely, all-ceramic anodes tend to have lower conductivity and catalytic performance.

    This work focuses on the structures in the cell which give mechanical support, the anode and electrolyte.
    Specifically, Ni-\gls{gdc} and \gls{sfcm}-\gls{gdc} anodes with \gls{gdc} electrolytes will be investigated..
    While \gls{ysz} systems have been studied extensively, the mechanical properties of \gls{gdc} systems have not been throughly explored.
    As \glspl{sofc} are developed which operate at lower temperatures, \gls{gdc} is the preferred choice for an electrolyte material due to its high conductivity.
    Ni-\gls{gdc} serves as a reliable starting place to characterize the mechanical properties of the anode, focusing on the changes which occur during reduction as NiO reduces to Ni metal.
    \gls{sfcm} being a new material has no available data on the mechanical properties, let alone the non-stoichiometry which occurs in the material under reducing conditions.

    Various cathode materials exist which assist in the incorporation of oxygen into the lattice.
    The cathode must have many similar properties to that of the anode, but be stable in oxidizing environments instead of reducing.
    As a result, perovskite structures again are found in many of the materials.
    \Gls{lsm} is one of the most commonly used materials, but others include doped \ce{SrCoO_3}, \ce{La_{0.6}Sr_{0.4}Co_{0.2}Fe_{0.8}O_3}, \ce{Ba_xSr_{1-x}Ti_{1-y}Fe_yO_{3-y/2+\textdelta}}. \cite{Minh2004, Cascos2016, Chen2015, Kuhn2013}
    The cathode does not tend to play a structural role in the fabrication of a cell, and thus is not investigated or developed in this work.

    Outside the of the cell itself, several materials are needed to create a functioning stack.
    Metal interconnects support individual cells, holding them together, creating connections for the current to flow though current collectors and gas to flow though channels to the the cell.
    In intermediate and high temperature stacks, expensive metal alloys such as Inconel must be used to survive the extreme heat.
    As temperatures are lowered to below \temp{600} more common metals can be used for interconnects, such as steel.
    Sealing materials are another critical component to assembly of an \gls{sofc} stack.
    Glasses with transition temperatures near the operating range are used at high temperatures as they can soften and flow into place creating gas tight seals.
    At lower temperatures new options are available, such as \gls{ysz}, vermiculite or graphite felts.%addcite

\section{Mechanical Properties}
    Ideally, an \gls{sofc} is as thin as possible to minimize diffusion path lengths and ionic resistances in the cell.\cite{Chan2001}
    Realistically, the cell must be able to withstand the stresses of being manufactured, sealed, heating, and use.
    This means that a compromise must be made as to how the cell is supported and which components do the supporting.
    Traditionally, electrolyte supported cells were used with \gls{ysz} electrolytes where the thick dense electrolyte provides the structural support for the cell.
    Recently anode supported cells made from \gls{gdc} have been able to provide lower resistances due to the thinner electrolyte while adequately supporting the cell with a thicker anode which fuel can diffuse further into.\cite{Fleischhauer2014a,Laurencin2008}
    Now the anode layer with its porosity, must support the majority of the stresses the cell is subjected to.

    Materials fail when the applied stresses exceed the strength.
    In a homogeneous material, stresses are applied uniformly throughout the material.
    In practice, defects will exist in the material creating points of inhomogeneities.
    These could be intentional with pores which are added to the material or unintended flaws such as inclusions, contamination or grain agglomeration.
    These microstructural flaws are locations where stresses are concentrated and lead to the ultimate failure of the material.

    Stresses are concentrated most by large, sharp featured flaws.
    As a result, in a uniform stress field, the largest flaw will usually cause failure.
    Ceramic materials will tend to fail where the sample is under tensile load and at a location near the point of maximum applied stress.
    Samples, even if processed together from the same raw materials, will each have a random sampling of flaws from the total batch, and a particular one which causes failure.
    No matter the care and attention paid to processing, some distribution of flaws will exist.
    The analysis of the distribution of flaws from a batch of samples is knows as Weibull analysis and is explained in detail in Appendix \ref{app:weibull}.
    To perform Weibull analysis a large number of samples must be tested, but as a result the overall distribution of flaws can be obtained.
    This allows for the observation of the flaw distribution and if it changes at different points or if another phenomenon is occurring.

    Many of the materials used in \glspl{sofc} have crystal structures which promote the generation of oxygen vacancies.
    As oxygen vacancies are produced, the inter-atomic bonding of the structure changes, which can decrease the elastic modulus and fracture toughness of the crystal.\cite{Bishop2014,Duncan2006}
    If the strength of the atomic bonds weakens on average, due to added vacancies, it then follows that the elastic modulus and fracture toughness of the crystal would also decrease.
    This relationship has been shown to fit for single grains of \gls{gdc} using nanoindentation, but this does not necessarily hold true for an actual cell.\cite{Wang2007}
    Grain boundaries can play a large role in the mechanical properties of a bulk sample.
    For this reason microstructure combined with environmental conditions can play a large role on the overall mechanical properties of a fuel cell.

    The first external stress to the cell, and the one most likely to result in failure, comes from the  sealing of an individual cell into a stack, creating a gas tight seal between the anode and cathode sides.
    While it is a compressive force that is applied, the cells are never perfectly flat, resulting in a flexural stresses as the cell is pressed flat.
    Figure \ref{fig:flatness} highlights this fact by using an optical profilometer to measure the flatness of a \SI{10}{\centi\meter} by \SI{10}{\centi\meter} Ni-GDC/GDC half-cell.
    It can be seen that the cell curves by over a millimeter, mostly at edges, where the seal would be taking place.
    It is for this reason that the study of the failure of \gls{sofc} materials utilizes flexural testing.

    \begin{figure}
      \centering
      \includegraphics[width=\textwidth]{3dsurface.png}
      \caption{Optical profilometer measurements of a \SI{10}{\centi\meter} by \SI{10}{\centi\meter} Ni-GDC/GDC half-cell where the z-axis has been magnified 6X to highlight curvature}\label{fig:flatness}
    \end{figure}

    %Kic
    %Cracking?
    %Fractography?
    %Testing at conditions

    %Literature review

    To help develop \glspl{sofc} into a viable technology which can stretch the gap in the energy economy, research is needed to improve the reliability of the cells during manufacturing and operation.
    This work aims to develop the understanding of fracture mechanics of real-world structures, by understanding how temperature and partial pressure of oxygen affect the strength, fracture toughness, and flaw distributions of \gls{sofc} components.
    This is done by analyzing the flexural properties of bulk bars and half-cell coupons at ambient conditions and at conditions similar to those found in an \gls{sofc}.
    Microscopy was used to look for changes in microstructure after various treatments in addition to Weibull analysis.
    Additionally, thermogravimetry, conductivity, and temperature programed desorption were used to understand the atomic defect structure of a new material to relate it to the mechanical properties.

    %Summary of contributions
