% !TEX root = ../mainthesis.tex

%Chapter 1

%\renewcommand{\thechapter}{1}

\chapter{Introduction}

\section{Motivation for Solid-Oxide Fuel Cell Research}

The world depends on fossil fuels for its daily energy needs.
This is a fact that is going to stay with us for the foreseeable future.
By 2040, it is estimated that fossil fuels will still supply  78\% of total energy demand.\cite{U.S.EnergyInformationAdministration2016}
Traditionally, the annual increase in demand is met by an increase in supply of fuel.
This increase is supplied by developments in mining and drilling operations, although recent advances have been found to be controversial, such as hydraulic fracturing.\cite{Osborn2011,Vengosh2014}
These concerns do not even account for the increased emissions caused from combustion of fossil fuels and their effect on the global environment.\cite{Solomon2009,Hansen1981,U.S.EnvironmentalProtectionAgency2017}

An alternative to increasing the supply and usage of fossil fuels is to increase our efficiency of turning the fuel into useful work.
Solid-oxide fuel cells (SOFCs) have the ability to solve this for applications where electrical power is desired.
SOFCs allow for the direct conversion of chemical energy to electrical energy, whereas devices such as combustion generators must convert chemical energy to thermal energy, to mechanical energy, and finally to electrical energy.
Each energy conversion has intrinsic losses, limiting efficiency.
Power generation plants have efficiencies around 30\%, while a stand-alone SOFC generator can convert fuel to electricity at 45 to 65\% efficiency.\cite{Wachsman2011a,Lasseter2004}
SOFCs are able to run on a variety of fuel sources, such as hydrogen, methane or even biogas.\cite{Park2000,Minh2004}
This fuel flexibility and higher efficiency positions SOFCs to bridge the gap in the energy economy as it transitions from fossil fuels to renewable sources.
However, many technical hurdles still remain which currently limit SOFCs' viability and must be addressed.

\section{Solid-Oxide Fuel Cell Operation}

Figure \ref{fig:sofc} demonstrates how a hydrogen fueled SOFC operates.
Fuel is exposed to the anode side of the cell, while an oxygen rich gas (usually air) is exposed to the cathode side.
Separating the two sides is the cell itself, with anode, electrolyte, and cathode.
The electrolyte is only conductive to oxygen ions, not electrons.
At the cathode, diatomic oxygen gas will disassociate into oxygen ions, each with a 2$-$ charge from the addition of electrons to the oxygen.
The oxygen ions then travel though the electrolyte and react with the fuel at the anode side, freeing the previously captured electrons.
To maintain charge balance, electrons must travel in the opposite direction, to the cathode.
Because the electrolyte only allows oxygen ions to diffuse though it, the electrons travel though an external circuit performing useful work in the process.

\begin{figure}
  \centering
  \includegraphics[width=0.5\textwidth]{sofc.png}
  \caption[Diagram showing flow of materials in the operation of a SOFC.]{Diagram showing flow of materials in the operation of a SOFC.\cite{Sakurambo}}\label{fig:sofc}
\end{figure}

SOFC materials rely on ionic conduction to transport oxygen from the cathode into the anode.
In ionic conduction, oxygen ions hop from site to site via oxygen vacancies, locations where an oxygen atom is missing from the lattice.
Because conduction depends on vacancies, the more vacancies in the lattice, the more available sites for oxygen to move between, the greater the conductance, and the greater power output of an cell.
The relationship between oxygen vacancy concentration and oxygen conductance is given by Equation \ref{eq:conductivity}, where \(\sigma\subscript{i}\) is the ionic conductivity, \(\lbrack\ch{V_O^{**}}\rbrack \) is the concentration of oxygen vacancies in the lattice, \(\mu_O \) is the ionic mobility of the oxygen, and \(Z_e\) is the charge of the conducting species.\cite{Mogensen2000}
Thus, by adding more oxygen vacancies the ionic conductivity increases.
While this a benefit to the electrical performance of the cell, it can have detrimental effects to the mechanical properties.
\begin{equation} \label{eq:conductivity}
  \sigma\subscript{i} = \lbrack\ch{V_O^{**}}\rbrack Ze\mu_O
\end{equation}

During operation, oxygen vacancies are created by removal of lattice oxygen from the anode as it reacts with the fuel, as shown in the half-cell reaction of Equation \ref{eq:vacancyformation}.
This reaction takes place on the surface of the anode, but for convenience it is written as having associated back into a diatomic state where the fuel can then react with it.
This reaction of fuel with oxygen creates an environment with a very low partial pressure of oxygen (pO\subscript{2}).
The final concentration of oxygen vacancies will depend on the exact environmental and material conditions and location in the cell, but some steady state value will eventually be reached.
\begin{equation} \label{eq:vacancyformation}
\ch{O_O^x  <-> 1/2 O2 + V_O^{**} + 2 e'}
\end{equation}

With the anode exposed to low pO\subscript{2}, and the cathode exposed to air, a chemical potential gradient is established across the cell.
This gradient drives the motion of oxygen through the cell and generates the electric current.
The electrical potential at open circuit conditions (voltage with no current draw) is expressed by the Nernst equation, given in Equation \ref{eq:nernst} for the reaction given in Equation \ref{eq:rxn}, where \(f_A\) is the fugacity of species A, \(E^o\) is the standard potential for the reaction, and n is the number of charges involved in the reaction, \(R\) is the ideal gas constant and \(F\) is Faraday's constant.\cite{Larminie2001}
Commonly, the fugacity is replaced by the partial pressure of the species.
For the case of a hydrogen fueled SOFC, Equation \ref{eq:nernst} can be simplified to Equation \ref{eq:nernsth2}.\cite{Pilatowsky2008}
From these equations it can be seen that the partial pressures of the involved species have a large role to play in the open circuit voltage of the cell.
\begin{equation}
  \label{eq:rxn}
\ch{aA + bB  <-> cC + dD}
\end{equation}
\begin{equation}
  \label{eq:nernst}
E = E^o - \frac{RT}{nF}\ln\left(\frac{f_C^c f_D^d}{f_A^a f_B^b}\right)
\end{equation}
\begin{equation}
  \label{eq:nernsth2}
E = E^o + \frac{RT}{2F}\ln\left(\frac{p_{H_2} {(p_{O_2})}^{1/2}}{p_{H_2 O}}\right)
\end{equation}

For the SOFC to efficiently function, oxygen and fuel must be able reach the active sites in cathode and anode respectively.
These active sites, or triple phase boundaries (TPB), are the points where gas, electronic conductor, and ionic conductor meet, as demonstrated in Figure \ref{fig:tpb}.
The more TPB that are present, the more exchange can occur between the gas and the cell at any given time and the current output of the cell is increased.
To maximize length of TPBs, pores can be added, greatly increasing the available surface area of the electrodes.
This added porosity does not come without the disadvantage of decreasing the mechanical strength of the cell.\cite{Pihlatie2009,Laurencin2010}
\begin{figure}
  \centering
  \includegraphics[width=0.5\textwidth]{tpb.png}
  \caption[Diagram showing the triple phase boundary and the importance as being the site where incorporation and reactions occur.]{Diagram showing the triple phase boundary and the importance as being the site where incorporation and reactions occur.\cite{Wachsman2011a}}\label{fig:tpb}
\end{figure}

The cathode and anode materials must facilitate the incorporation of oxygen into the cell and the reaction between the oxygen and the fuel.
Typically, the cell operates at temperatures ranging from \SI{600}{\celsius} to \SI{1000}{\celsius} in order to increase the ionic conductivity and performance.
It is critical that the fuel and air remain separated by the cell.
If there is a leak either through the cell or around it, then the electric potential across the cell will decrease leading to failure.


\section{Solid-Oxide Fuel Cell Materials}

Each structure of an SOFC is fabricated from a single material or from combinations of materials designed to optimize that part's function.
For example, a cell can be comprised of a nickel metal-gadolinium doped ceria (GDC) anode, a GDC electrolyte, and then a lanthanum strontium manganite (LSM) cathode.\cite{Liu2002,Haile2003}
These individual components are cast into thin flexible sheets (tapes) from bulk powders, with additives such as pore formers where appropriate, and then laminated together and co-fired to create a single cell.
As a result, a completed cell has distinct layers in it where the materials and properties abruptly change.
This composite structure can then have inter-diffusion between layers, smoothing out the abrupt changes, but creating new structures and compositions that were not present upon lamination.\cite{Yokokawa2008}

Nickel has traditionally been used as an anode material in SOFCs, due to its high conductivity and catalytic properties in combination with an ionic conductor such as GDC. %citation would be good
A challenge with nickel is that at ambient conditions it oxidizes into nickel oxide, but at operating conditions it reduces to the desired nickel metal.
There is a large lattice parameter change between nickel and nickel oxide, so that as it reduces, it decreases volume by over 15\%.
This means that the overall porosity of the anode layer increases as the SOFC is put into service.\cite{Gutierrez-Mora2002,Yu2007}
This increased porosity can greatly decrease the flexural strength and modulus of the anode and of the overall cell.\cite{Callister2014,Barsoum2003}
Additionally, nickel is prone to poisoning from sulfur contaminants in the fuel, so alternative anodes are of interest for use in SOFCs.

A first approach has been to develop other metal-ceramic systems to serve the function of the anode.
These have included copper, cobalt, and platinum systems, but each suffer from a significant drawback such as coking, long-term performance degradation or cost.
A different approach is to use an all-ceramic anode, made of a mixed ionic-electronic conductor (MIEC).
Perovskite materials have easily been developed into MIECs with the use of a transition metal occupying the octahedral B-site.
Examples of MIEC ceramics are \ce{La_{\hbox{1--x}}Sr_{x}Cr_{\hbox{1--y}}Mn_{y}O_{3}}, \ce{Sr_{0.94}Ti_{0.9}Nb_{0.1}O_{3-\delta}}, \ce{Ba_{0.98}La_{0.02}SnO_3}, or \ce{Sr_{2}MgMoO_{6}}.\cite{Goodenough2007,Zha2005,Primdahl2001,Hussain2013,MohammedHussain2012,Hussain2016,Huang2006}.
Due to the all-ceramic nature of these anodes, their thermal expansion coefficients better match that of the other ceramic components of the cell, reducing sintering stresses and improving redox cycling durability.
Conversely, all-ceramic anodes tend to have lower conductivity and catalytic performance.

As mentioned previously, GDC is one of two common choices for electrolyte materials.
The other, more traditionally used material is yttria-stabilized zirconia (YSZ).
YSZ is very stable, has a ratio of ionic conductivity to electronic conductivity (transference number) near 1, and good conductivity at temperatures in excess of \temp{800}.
At lower temperatures GDC exhibits conductivity, but does suffer from a lower transference number, especially under extremely reducing conditions.
Other rare-earth doped cerias have been used as electrolytes, but GDC tends to outperform them under a variety of conditions.
As SOFC operating temperatures are reduced for improved efficiency, GDC is favored for the electrolyte and anode composite material.

Various cathode materials exist which assist in the incorporation of oxygen into the lattice.
Lanthanum strontium manganite (LSM) is one of the most common.
The cathode does not tend to play a structural role in the fabrication of a cell, and thus is not investigated or developed in this work.

\section{Mechanical Properties}

Ideally, an SOFC is as thin as possible to minimize diffusion path lengths and ionic resistances in the cell.\cite{Chan2001}
Realistically, the cell must be able to withstand the stresses of being manufactured, sealed, heating, and use.
This means that a compromise must be made as to how the cell is supported and which components do the supporting.
Traditionally, electrolyte supported cells were used with yittra-stabilized zirconia electrolytes, but recently anode supported cells have been able to provide lower resistances while adequately supporting the cell.\cite{Fleischhauer2014a,Laurencin2008}
Now the anode layer, which is needed to the porous for gas diffusion, must support the majority of the stresses the cell is subjected to.

Materials fail when the applied stresses exceed the strength.
In a homogeneous material, stresses are applied uniformly throughout the material.
In practice, defects will exist in the material creating points of inhomogeneities.
These could be intentional with pores which are added to the material or unintended flaws such as inclusions, contamination or grain agglomeration.
These microstructural flaws are locations where stresses are concentrated and lead to the ultimate failure of the material.

Stresses are concentrated most by large, sharp featured flaws.
As a result, in a uniform stress field, the largest flaw will usually cause failure.
Ceramic materials will tend to fail where the sample is under tensile load and at a location near the point of maximum applied stress.
Samples, even if processed together from the same raw materials, will each have a random sampling of flaws from the total batch, and a particular one which causes failure.
No matter the care and attention paid to processing, some distribution of flaws will exist.
The analysis of the distribution of flaws from a batch of samples is knows as Weibull analysis and is explained in detail in Appendix \ref{app:weibull}.
To perform Weibull analysis a large number of samples must be tested, but as a result the overall distribution of flaws can be obtained.
This allows for the observation of the flaw distribution and if it changes at different points or if another phenomenon is occurring.

Many of the materials used in SOFCs have crystal structures which promote the generation of oxygen vacancies.
As oxygen vacancies are produced, the inter-atomic bonding of the structure changes, which can change bulk properties of the crystal.\cite{Bishop2014,Duncan2006}
If the strength of the atomic bonds weakens on average, due to added vacancies, it then follows that the elastic modulus and fracture toughness of the crystal would also decrease.
This relationship has been shown to fit for single grains of GDC using nanoindentation, but this does not necessarily hold true for an actual cell.\cite{Wang2007}
Grain boundaries can play a large role in the mechanical properties of a bulk sample.
For this reason microstructure combined with environmental conditions can play a large role on the overall mechanical properties of a fuel cell.

The first external stress to the cell, and the one most likely to result in failure, comes from the  sealing of an individual cell into a stack, creating a gas tight seal between the anode and cathode sides.
While it is a compressive force that is applied, the cells are never perfectly flat, resulting in a flexural stresses as the cell is pressed flat.
It is for this reason that the study of the failure of SOFC materials utilizes flexural testing.

%Kic
%Cracking?
%Fractography?
%Testing at conditions

To help develop SOFCs into a viable technology which can stretch the gap in the energy economy, research is needed to improve the reliability of the cells during manufacturing and operation.
This work aims to develop the understanding of fracture mechanics of real-world structures, by understanding how temperature and partial pressure of oxygen affect the strength, fracture toughness, and flaw distributions of SOFC components.
This is done by analyzing the flexural properties of bulk bars and half-cell coupons at ambient conditions and at conditions similar to those found in an SOFC.
Microscopy was used to look for changes in microstructure after various treatments in addition to Weibull analysis.
Additionally, thermogravimetry, conductivity, and temperature programed desorption were used to understand the atomic defect structure of a new material to relate it to the mechanical properties.
