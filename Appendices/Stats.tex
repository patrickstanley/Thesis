% !TEX root = ../mainthesis.tex

%Appendix -- January 2015
%\appendix
%\renewcommand{\thechapter}{A}
%\renewcommand{\chaptername}{Appendix}

\chapter{Statistics}

\section{Student's t-test}
The Student's t-test, or a two sample t-test, is a statistical test commonly used to determine if to sets of data are different from each other.
A t-test creates a confidence interval around a mean for in which the true mean is thought to exist.
By applying this to two sample sets and using the null hypothesis test on the difference of the means, it can be determined if the two means are distinct, with a particular confidence.
In the case of randomly dividing samples to receive separate treatments, this constitutes an unpaired test.

Like all statistical tests, this relies on a few base assumptions about the population.
First, it is assumed that the population has a normal distribution, which reasonably fits for the samples which were studied.
Second, the two populations should have the same variance, meaning that the spread of the data compared to the mean should be equivalent, as determined by the F test.
In the event where the variances are not equal, a variation of the t-test may be used.
Finally, the two populations should be sampled independently from each other.

\begin{equation}
    \label{eq:studentt}
    t=\frac{\bar{X}_1 - \bar{X}_2}{\sqrt{\frac{\sigma^2_1}{n_1}+\frac{\sigma_2^2}{n_2}}}
\end{equation}



\section{Weibull Statistics}
\label{app:weibull}.
