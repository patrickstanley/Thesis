% !TEX root = ../mainthesis.tex

%Appendix -- January 2015
%\appendix
%\renewcommand{\thechapter}{A}
%\renewcommand{\chaptername}{Appendix}

\chapter{TGA Manual}
\label{app:TGA}
\section{Theory of Operation}
Thermogravimetry works on the principle of measuring small changes in mass as the environment changes.
Changes in environment can come from changing temperature or gas composition.
This system uses a Cahn microbalance to measure mass, a furnace to heat the sample up to \temp{1100}, and \glspl{mfc} to control the gaseous environment.

\section{Mass Measurement}
    The TGA uses a Cahn D200 microbalance for the mass sensing capability.
    The microbalance works by measuring the current required to maintain balance between the sample and a counter balance.
    The sample can be hung at two different points on the arm of the balance, Loop A or Loop B.
    The choice of loops determines the capacity and sensitivity of the measurements, as shown in Table \ref{tab:tga}.
    Full specifications are given in the microbalance manual.
    Loop A is used most commonly for its increased sensitivity.

    \begin{table}
    \centering
    \caption{Cahn D200 Microbalance Specifications}
    \label{tab:tga}
    \begin{tabular}{lll}
                               & Loop A & Loop B                                   \\
    \cline{2-3}
    Capacity (g)               & 1.5    & 3.5                                      \\
    Maximum Weight Change (mg) & 150    & 750                                      \\
    Sensitivity (\SI{}{\micro\gram})           & 0.1    & 1                                        \\
    Repeatability              & \multicolumn{2}{l}{10\% total load on both pans}
    \end{tabular}
    \end{table}

    The microbalance has an independent flow of nitrogen gas to it to maintain a consistent atmosphere.
    Only a small amount of flow is required (\SI{5} to \SI{10}{sccm}) to keep positive pressure, but if the protective cover is removed additional time will be required to purge the balance.

    \subsection{Setup and Operation}
        Alignment of the balance is key to producing a low noise signal.
        Ideally, the hang-down wire should run through the center of the tubes and at minimum it must not touch anywhere or allow the sample pan to touch the walls of the furnace.
        The hang-down wire is made from platinum or a platinum-rhodium mixture and is in two pieces, to facilitate the changing out of the bottom portion in case of contamination and degradation.
        Two successful methods of straightening the wire are to lay it flat on a surface and roll it out from the center, similar to working dough, or to hang it with a weight and soften it using heat allowing it to stretch straight.
        Hooks on the ends are created using tweezers and a razor blade for a sharp bend.

        With the hang-down wire straight and in place, the furnace tube can be adjusted to place the sample pan in the middle, equidistance from all walls.
        Moving the tube can be achieved by adjusting the supporting clamp at the bottom of the furnace or adjusting the insulation holding it in place at the top.
        Alignment usually requires looking straight down the tube from above the microbalance.
        A light can be placed at the bottom of the tube, replacing the thermocouple, to give illumination as to the location of the sample pan in the tube.
        Alternatively, the furnace can be heated to \temp{800} so that the tube and pan glow, giving a better light, but adjustments must wait for the furnace to cool.
        Usually, both methods are used, the flashlight for a first alignment and the hot furnace to check afterwards.

        With everything aligned, the counterbalance needs to be properly adjusted.
        Place on the sample pan any appropriate crucible which will be used and open the microbalance cover and remove the cover from the counterbalance by gently pulling.
        At this point, the microbalance is turned off using the switch on the back of the microbalance controller unit.
        Counterbalance weights are added or removed to have the empty sample pan and crucible balance with the counter.
        A light touch to the arm of the balance may be needed to off-set static and friction but ensure this is done with clean tweezers.
        Once the system balances, or does so as closely as possible, the covers may be replaced and the balance turned back on.
        Static, especially in the winter time, commonly causes the counterbalance to swing wildly as the cover is placed back over it.
        Using an un-gloved hand or wiping the inside down with deionized water helps reduce the static.

        The thermocouple in the bottom of the furnace should be placed as close to the sample as possible.
        To achieve this, the thermocouple is incrementally raised while observing the mass readout from the microbalance.
        Once the balance becomes unstable or jumps, the thermocouple should be backed out \SI{0.5} to \SI{1}{\centi\meter} and tightened in place.
        The thermocouple is covered with a quartz sheath to act as a thermowell and protect the thermocouple.
        If changing between materials systems or significant degradation has occurred to the thermowell, it should be replaced.

        At this point, the system should be burned out to remove any contamination on the sample pan or hang-down wire, and time is needed for it to stabilize.
        The furnace should be heated to \temp{1000} for 4 hours in an oxygenated environment.
        This also provides time for the system to stabilize, as monitored by creating a run with the MicroScan Acquisition software.
        It can take up to 24 hours for it to stabilize.
        If random jumps occur in the mass, recheck the alignment and thermocouple placement.

        After stabilization, a sample can be loaded in to a crucible and sample pan.
        While the microbalance is calibrated to measure the mass of the sample added, it is good practice to measure and record the mass using a separate balance beforehand.
        Crucibles may be reused where appropriate.
        After the sample is placed, the furnace is raised slowly, to ensure not altering the alignment, and connected to the fittings above.
        The furnace should be at the maximum height position when finished.
        Time should then be given to allow the mass reading to stabilize before starting a test.

        The program MicroScan Acquisition controls the TGA and records data.
        After opening, select ``Balance \textgreater{} Establish Connection'' to connect to the D200.
        If an error occurs, power cycle the balance, restart the program, and check the serial cable leading to the balance.
        Upon successful connection, the current readout of the balance will be displayed.
        The balance may be tared before adding a sample by selection ``Balance \textgreater{} Tare.''
        Starting a new run is achieved by selecting ``File \textgreater{} Open Method.''
        Figure \ref{fig:tgamicroaq} shows the ``Open Method'' screen with the current balance readout from MicroScan Acquisition.

        \begin{figure}
            \begin{center}
            \includegraphics[width=0.8\textwidth,keepaspectratio]{tgamicroacq.png}
            \end{center}
            \caption{MicroScan Acquisition with Open Method open and the current read out of the balance displayed}
            \label{fig:tgamicroaq}
        \end{figure}

        From here, the total length of the run is defined along with the sample rate.
        The sample rate will need to match the rate given in the other programs.
        Additionally, the run title can be defined and the loop can be changed if needed.
        Once the run length and sample rate are entered, selecting ``Ok'' brings up the menu where the run can be started.
        Selecting ``Start Run'' causes MicroScan Acquisition to prompt for the location to save the data in a proprietary format, and once the file path is given, it then starts collecting data.
        During the run, collected data is plotted and can be viewed using the controls at the top of the plot.
        After a run is complete, the saved file is converted to a \gls{csv} by opening it with MicroScan Analysis and selecting ``File \textgreater{} Export... \textgreater{} Comma/Tab Separated...''

\section{Heating}
    Heating is done by a furnace surrounding the alumina or quartz tube and sample.
    It is controlled by a Eurotherm controller and readings are taken from a thermocouple placed inside the furnace tube.
    As a result, there is a lag in the system between the furnace elements heating and the thermocouple sensing the heat.
    The controller can be programmed with up to 8 segments in the temperature profile and programming can be performed from the front panel of the unit or by the Eurotherm iTools software.
    The iTools software must be closed before attempting to use the LabVIEW program which reads the temperature from the controller.

\section{Gas Delivery}
    Gas flows are controlled by the MKS Type 647C Multi-channel Flow Ratio/Pressure Controller and the individual MKS Type M100B \glspl{mfc}.
    Each \gls{mfc} controls one gas by measuring the flow with a thermal mass sensor and adjusting it using a valve to meet the desired set point.
    Each \gls{mfc} has a rated size and must have the zero and \gls{gfc} properly set.
    The zero can be set physically using a screw on the \gls{mfc} if a large adjustment is needed or electronically using the controller.
    The zero should be set with both sides of the \gls{mfc} open to the atmosphere to ensure no leak from a leftover pressure differential.
    The \gls{gfc} can be theoretically calculated based on the heat capacity of the gas compared to a standard.
    This assumes that the \gls{mfc} is properly calibrated, which may not be a safe assumption.
    As such, it is best to calibrate the \gls{mfc} by adjusting the \gls{gfc} so that it measures correct flow rates as given by a bubble meter.

    If an \gls{mfc} leaks despite being set to off, the tension spring for the valve located inside the \gls{mfc} may need adjustment.
    The manual for the \gls{mfc} outlines the procedure.
    If the tension is correct and the valve still leaks, service is advised.
    An alternative is to add an external shut-off valve, which can be closed either manually or automatically.
    This has been implemented for one of the current \glspl{mfc}.

    Different gases can be supplied to the \glspl{mfc} as needed for the experimental setup.
    If the gas to an \gls{mfc} is changed, a new \gls{gfc} will need to be obtained.
    Gases can humidified after the \gls{mfc} with the use of a bubbler, but they should not be humidified beforehand for risk of damaging the \gls{mfc}.
    Care should also be taken when mixing or changing gas compositions to avoid potentially dangerous combinations, such as mixing fuels and oxidizers.
    When changing between a reducing and oxidizing environment, it is good practice to allow sufficient time for a sweep gas (such as N\textsubscript{2}) to reduce the risk of the different atmospheres mixing.

    The \glspl{mfc} can be controlled by a computer with LabVIEW.
    They can either be manually set or given a scheduled program to follow.
    Typically, a constant total flow rate will be used among all tests.
    A good starting point for the total flow rate is \SI{50}{sccm}.
    After turning on the MKS controller, the master flow control must be turned on by pushing ``ON'' followed by ``0/All.''

\section{pO\textsubscript{2} Measurement}
    \po2{} of the gas flow is measured by a home-built \gls{ysz} sensor located after the \glspl{mfc} combine into one flow.
    Gas flows down a two-bore alumina tube along with a platinum wire to the end of a \gls{ysz} tube.
    The wire is lightly connected to the inside of the \gls{ysz} tube using platinum paste.
    The gas then flows in the reverse direction, between the alumina and \gls{ysz} tubes to the outlet.
    Outside the \gls{ysz} tube at the end, a separate piece of platinum wire is connected to the end of the tube.
    This setup is then placed in a furnace and heated to \temp{800} where the difference of partial pressure can be measured according to the Nernst equation.
    Voltage potentials from the two wires are measured by a Keithley 2000 digital multimeter.

    Once the system is setup, very little needs to be done to maintain the sensor.
    If the sensor needs to cooled in anticipation of a power outage or in order to be worked on, a ramp rate of \SI{1}{\celsius\per\minute} is to be used to avoid damaging the \gls{ysz} tube.
    For accurate \po2{} measurements, after any modification a calibration curve needs to be created using a reference meter.
    Voltage measurements and \po2{} calculations are recorded by the computer using a LabVIEW program.

\section{Controls}
    Almost all of the components of the \gls{tga} system are computer controllable.
    As previously mentioned and explained, the microbalance is controlled by the MicroScan Acquisition program.
    LabVIEW programs interface with the Keithley, MKS controller, and Eurotherm temperature controller.

    \subsection{Gas Controls}
        Gas flows can be controlled with either the Gas Controller or Gas Scheduler programs.
        Gas Controller allows for the immediate control of any gas flow channel while Gas Scheduler changes the gas flows at predetermined times allowing for the automated changing of composition.
        Figure \ref{fig:gascontrolfront} shows the front panel of Gas Controller, Figure \ref{fig:gasschedfront} is the front panel of Gas Scheduler, and the code for both programs is explained in Appendix \ref{app:gasvi}

        \begin{figure}
            \begin{center}
            \includegraphics[width=0.75\textwidth,keepaspectratio]{Gas_Controllerp.png}
            \end{center}
            \caption{Front panel display of Gas Controller program}
            \label{fig:gascontrolfront}
        \end{figure}

        \begin{figure}
            \begin{center}
            \includegraphics[width=0.6\textwidth,keepaspectratio]{Gas_Schedulerp.png}
            \end{center}
            \caption{Front panel display of Gas Scheduler program}
            \label{fig:gasschedfront}
        \end{figure}

        To start using Gas Controller, ensure proper channels are set for the MKS Controller Port, Arduino Port, and Servo Channel.
        On and Off position adjust the position of the external shut-off valve controlled by the Arduino.
        These values should only change if changing USB to serial adapters or with new hardware.
        Starting the LabVIEW program immediately begins the control to gasses to the specified values.
        The radio buttons on the left turn channels on or off and set them to the set point.
        To end the program, push the STOP button within the program front panel.
        The program displays the current reading for the flow to ensure flow is occurring as specified.

        The Gas Scheduler requires the input of a tab delimited text file, referred to as a schedule.
        An example schedule comes with the program.
        In this schedule, the time for the desired gas flow is given in seconds, followed by the set point of each channel.
        The program ends on the row with a negative time value, retaining those set points.

        Before starting the program, the file path to the schedule must be given by clicking the folder icon next to the Path to Program File field.
        Again, ensure that the com ports and channels are correct.
        The program records the measured flows at a set interval, as defined by the Recording Period (in seconds) and records them to a \gls{csv}.
        Additionally, comments can be added to the \gls{csv} in the optional comments field.
        After starting the program, the start button needs to be pressed.

    \subsection{Temperature and \po2{} Measurements}
        The program Temp \& pO2 Measurements reads the values from the Eurotherm controller and the Keithley multimeter, plots them, and records the results to a \gls{csv}.
        Figure \ref{fig:temppo2measure} shows the front panel of the program with the user controls.

        Before running the program, ensure that the com ports and channels are correct.
        The Measurement Period should match that given to MicroScan Acquisition.
        Enabling the timer provides for the program to stop at after the designated time.
        Comments can be added to the \gls{csv} file by filling in the File Comments field.
        When the program starts running, it will ask for the location to save the \gls{csv}.
        As a note, the timer for the program starts as soon as the program does, before defining the \gls{csv} location, but the first measurement does not occur until afterwards.

        \begin{figure}
            \begin{center}
            \includegraphics[width=0.9\textwidth,keepaspectratio]{Temp_&_pO2_Measurementsp.png}
            \end{center}
            \caption{Front panel display of Temp \& pO2 Measurements program}
            \label{fig:temppo2measure}
        \end{figure}

    \subsection{Starting a Run}
        Multiple programs needs to be started simultaneously for a run of the \gls{tga}.
        Because each program starts differently, the following is a recommendation for the order to start the programs.
        Time values from MicroScan should then be used in subsequent analysis.
        This procedure minimizes the amount of time difference between measurements on the \gls{tga}, temperature and \po2{}.

        To begin, enter all values into the various programs.
        In MicroScan Acquisition, follow the full setup, including clicking Start Run, but stop immediately afterwards with the file browser up.
        In Temp \& pO2 Measurements, follow the full setup including designating where to save the data.
        As soon as the file path to save data is submitted, switch to MicroScan Acquisition and complete the setup, submitting the file path also.
        If using Gas Scheduler, start that program as normal followed by the temperature program on the Eurotherm.

\section{Interfacing with other devices}
    The effluent of the \gls{tga} can be fed to other devices for advanced analysis, such as an \gls{ms}.
    When doing this, one thing to note is physical distance between the \gls{tga} and the other instrument.
    The longer the gas line connecting the two instruments, the longer it will take gas to travel from one instrument to the other.
    During this time period, diffusion will occur up and down the line, broadening any signal to the second instrument.
    This delay and broadening effect will need to be investigated for each experimental setup.

\section{Troubleshooting}
    \begin{itemize}
      \item \textbf{Unsteady mass:} check alignment and thermocouple placement
      \item \textbf{No gas flow on any channel:} check master flow control
      \item \textbf{No gas flow on one channel:} check if the channel is on and manual valves are open
      \item \textbf{MFC leaks when off:} Check zero, adjust spring tension as described in MFC manual, add external shutoff valve or return for repairs
      \item \textbf{No or noisy voltage from \po2{} sensor:} Check wire connections from sensor, rebuild sensor with new platinum paste
      \item \textbf{Communications Error:} Check that no other program is communicating with the hardware, restart the hardware and unplug then plug-in the USB connection
    \end{itemize}
