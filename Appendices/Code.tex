% !TEX root = ../mainthesis.tex

%Appendix -- January 2015
%\appendix
%\renewcommand{\thechapter}{B}
%\renewcommand{\chaptername}{Appendix}

\chapter{Code}
    The programs referenced in this appendix are published as open-source software and are in the public domain.
    The code and latest versions of all parts are available on GitHub at \href{https://github.com/patrickstanley}{https://github.com/patrickstanley/} with the provided repository names and contributions are welcome.
    This work builds upon provided libraries and previous users experiences.

\section{Arduino PID Relay Furnace Controller with Serial Connectivity}
    \label{app:PID}

    The purpose of this Arduino sketch is to power a heating furnace
    (controlled by a solid-state relay) with PID controls.
    In addition, serial communications are used to monitor, log, and interact with the controller.
    A running average is used to reduce thermocouple measurement noise for the derivate component of the controls.
    This sketch is available at \href{https://github.com/patrickstanley/arduino-pid}{https://github.com/patrickstanley/arduino-pid}.

    \subsection{Hardware and other libraries}
        This was built to be used on a Arduino Uno R3, but should be easily run on a variety of different boards.
        For temperature sensing, an \href{https://www.adafruit.com/product/3263}{Adafruit MAX31856} is used with the provided \href{https://github.com/adafruit/Adafruit_MAX31856}{library}.
        Also used is the PID controller \href{https://github.com/br3ttb/Arduino-PID-Library}{library by br3ttb}.
        This sketch pieces the two libraries together and adds serial communications functionality.

    \subsection{Variables}
        The following is a list of control variables which may be changed per setup.

        \begin{itemize}
        \item
          RelayPin: The physical pin the relay is connected to.
        \item
          Adafruit\_MAX31856(CS, DI, DO, CLK): Pins for the MAX31856
        \item
          WindowSize: Length of on/off cycle, needed to account for AC power
        \item
          printdelay: Delay time in milliseconds for printout on serial
        \item
          MaxOP: Maximum operating power, used to extend life of elements
        \item
          myPID(\&Input, \&Output, \&workingSet,P,I,D, DIRECT): PID values need to be adjusted for the particular setup.
          If inverse behavior is experienced (e.g. heats when it should cool) switch DIRECT.
        \item
          workingSet and Setpoint: Initial setpoint for power on
        \item
          ramprate: Sets a ramp rate for the working setpoint. Measured in
          milliseconds for 1C change.
        \end{itemize}

    \subsection{Serial communications}
        Baud rate is set to 9600.
        Output is tab delimitated of Setpoint, Working
        Setpoint, Thermocouple Temperature, and Output Power.
        To change Setpoint, send a new number over serial and press return.
        The endline character of the serial monitor may have to changed send \textbackslash n as end of line.

    \subsection{Arduino Sketch}
        \lstinputlisting[language=C++,breaklines=true,postbreak=\mbox{\textcolor{red}{\(\hookrightarrow\)}\space}]{../../../Arduino/thermo_pid_serial/thermo_pid_serial.ino}

\section{MKS Gas Controller and Scheduler}
    \label{app:gasvi}
    The purpose of this Labview Program is...
    The VI gas controller and gas scheduler VI's, along with a sample gas schedule, are available under the repository "mks-mfc."

    \subsection{Hardware and VI's}

    \subsection{Program Overview}
    \begin{landscape}
        \begin{figure}
            \begin{center}
            \includegraphics*[width=1.5\textwidth,keepaspectratio,viewport=0 0 1052 506]{Gas_Controllerd.png}
            \end{center}
            \caption{MFC Controller program}
        \end{figure}

        \begin{figure}\ContinuedFloat
            \begin{center}
            \includegraphics*[width=1.5\textwidth,keepaspectratio,viewport=1052 0 2104 506]{Gas_Controllerd.png}
            \end{center}
            \caption[]{MFC Controller program (cont.)}
        \end{figure}

        \begin{figure}\ContinuedFloat
            \begin{center}
            \includegraphics*[width=1.5\textwidth,keepaspectratio,viewport=2104 0 3156 506]{Gas_Controllerd.png}
            \end{center}
            \caption[]{MFC Controller program (cont.)}
        \end{figure}

        \begin{figure}\ContinuedFloat
            \begin{center}
            \includegraphics*[width=1.5\textwidth,keepaspectratio,viewport=3156 0 4208 506]{Gas_Controllerd.png}
            \end{center}
            \caption[]{MFC Controller program (cont.)}
        \end{figure}

        \begin{figure}\ContinuedFloat
            \begin{center}
            \includegraphics*[width=1.5\textwidth,keepaspectratio,viewport=4208 0 5260 506]{Gas_Controllerd.png}
            \end{center}
            \caption[]{MFC Controller program (cont.)}
        \end{figure}

        \begin{figure}\ContinuedFloat
            \begin{center}
            \includegraphics*[width=1.5\textwidth,keepaspectratio,viewport=5260 0 6312 506]{Gas_Controllerd.png}
            \end{center}
            \caption[]{MFC Controller program (cont.)}
        \end{figure}%MFC Controller VI Images
    \end{landscape}

    \begin{landscape}%MFC Scheduler images
        \begin{figure}
            \begin{center}
            \includegraphics*[width=1.5\textwidth,keepaspectratio]{Gas_Schedulerd.png}
            \end{center}
            \caption{Gas Scheduler program}
        \end{figure}
    \end{landscape}

\section{Temperature and \po2{} Measurements}
    \label{app:tgavi}
    The purpose of this Labview Program is...
    The VI is available at on GitHub under the repository "eurotherm-keithely-measure."

    \subsection{Hardware and VI's}

    \subsection{Program Overview}
    \begin{landscape}%Temp and o2 measure program images
        \begin{figure}
            \begin{center}
            \includegraphics*[width=1.5\textwidth,keepaspectratio]{Temp_&_pO2_Measurementsd.png}
            \end{center}
            \caption{Temperature and \po2{} measurement program}
        \end{figure}
    \end{landscape}
