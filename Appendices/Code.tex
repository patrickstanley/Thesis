% !TEX root = ../mainthesis.tex

\chapter{Code}
    The programs referenced in this appendix are published as open-source software and are in the public domain.
    The code and latest versions of all parts are available on GitHub at \href{https://github.com/patrickstanley}{https://github.com/patrickstanley/} with the provided repository names and contributions are welcome.

\section{Arduino PID Relay Furnace Controller with Serial Connectivity}
    \label{app:PID}

    The purpose of this Arduino sketch is to power a heating furnace
    (controlled by a solid-state relay) with PID controls.
    In addition, serial communications are used to monitor, log, and interact with the controller.
    A running average is used to reduce thermocouple measurement noise for the derivate component of the controls.
    This sketch is available at \href{https://github.com/patrickstanley/arduino-pid}{https://github.com/patrickstanley/arduino-pid}.

    \subsection{Hardware and Other Libraries}
        This was built to be used on a Arduino Uno R3, but should be easily run on a variety of different boards.
        For temperature sensing, an \href{https://www.adafruit.com/product/3263}{Adafruit MAX31856} is used with the provided \href{https://github.com/adafruit/Adafruit_MAX31856}{library}.
        Also used is the PID controller \href{https://github.com/br3ttb/Arduino-PID-Library}{library by br3ttb}.
        This sketch pieces the two libraries together and adds serial communications functionality.

    \subsection{Variables}
        The following is a list of control variables which may be changed per setup.

        \begin{itemize}
        \item
          RelayPin: The physical pin the relay is connected to.
        \item
          Adafruit\_MAX31856(CS, DI, DO, CLK): Pins for the MAX31856
        \item
          WindowSize: Length of on/off cycle, needed to account for AC power
        \item
          printdelay: Delay time in milliseconds for printout on serial
        \item
          MaxOP: Maximum operating power, used to extend life of elements
        \item
          myPID(\&Input, \&Output, \&workingSet,P,I,D, DIRECT): PID values need to be adjusted for the particular setup.
          If inverse behavior is experienced (e.g. heats when it should cool) switch DIRECT.
        \item
          workingSet and Setpoint: Initial setpoint for power on
        \item
          ramprate: Sets a ramp rate for the working setpoint. Measured in
          milliseconds for 1C change.
        \end{itemize}

    \subsection{Serial Communications}
        Baud rate is set to 9600.
        Output is tab delimitated of Setpoint, Working
        Setpoint, Thermocouple Temperature, and Output Power.
        To change Setpoint, send a new number over serial and press return.
        The newline character of the serial monitor may have to changed send \textbackslash n as end of line.

    \subsection{Arduino Sketch}
        \lstinputlisting[language=C++,breaklines=true,postbreak=\mbox{\textcolor{red}{\(\hookrightarrow\)}\space}]{../../../Arduino/thermo_pid_serial/thermo_pid_serial.ino}

\section{MKS Gas Controller and Scheduler}
    \label{app:gasvi}
    The purpose of these LabVIEW programs is to communicate with and control \glspl{mfc} for remote and automated operation of the \gls{tga}.
    Individual \glspl{mfc} can be turned on or off and have their set point changed.
    After basic controls are developed, a new VI, a scheduling program, is provided which reads a text file automatically changing flow rates at predetermined times.
    The VI gas controller and gas scheduler VIs, along with a sample gas schedule, are available under the GitHub repository ``mks-mfc.''
    The VIs are saved for LabVIEW version 2013 and later.
    This work is the continuation and development of programs previously written by others.

    \subsection{Hardware and Other Libraries}
        These programs are written for the use of a MKS 647C Multi Channel Flow Ratio/Pressure Controller with Type M100B \glspl{mfc}.
        It can accommodate up to 8 channels for \glspl{mfc} and communicates via serial.
        Drivers for LabVIEW communications with the MKS 647C are available online from the \gls{ni} website.
        Serial communications occur over the VISA standard and require the installation of the VISA drivers from \gls{ni}.

        An Arduino Uno R3 is utilized with a servo motor and 3D printed parts to manually turn a shut-off valve, compensating for a leaking \gls{mfc}.
        Details on the installation of the Arduino controlled shut-off including STL files for 3D printing are available at \href{https://hackaday.io/project/52519-automated-gas-flow-shut-off}{https://hackaday.io/project/52519-automated-gas-flow-shut-off}.
        To control the Arduino from LabVIEW, the LINX library is used and must be downloaded from the VI package manager.
        Additionally, the Arduino will need to be flashed with the LINX control sketch before use.
        This component can be easily removed if not required for the application.

    \subsection{Program Overview}
    The block diagram of the Gas Controller VI, which manually controls the \glspl{mfc}, is provided in Figure \ref{fig:mfcvi} and is the fundamental starting block for the gas scheduling program.
    Starting on the far left of the program, initial communications variables such as baud rate and port information are passed to the sub-VIs which open communications.
    The program then enters a while loop until the stop button is pressed or an error occurs, at which point serial communications are closed by the appropriate sub-VI.

    Within the while loop, the program sequentially goes to each channel, checking the desired status, setting the control variables and reading the flow.
    If the program is set to not use an \gls{mfc}, the internal valve is shut and the flow is read to ensure no leakage.
    If an \gls{mfc} is set to be used, the percent of full flow is calculated based on the size of the \gls{mfc}, the gas correction factor, and the desired flow rate.
    The percent of full flow is then passed to the \gls{mfc} and the flow is read.
    For the case of channel 3, which possesses a leaky \gls{mfc}, the addition of an external shutoff valve controlled by Arduino was implemented.
    If channel 3 is to be used, the Arduino sends a \gls{pwm} signal to the servo motor which corresponds to the valve being in the open position.
    In the case where channel 3 is to be closed, a \gls{pwm} signal is sent, which closes the valve.

    Finally, after each channel has been set, the program checks for the dangerous combination of hydrogen and oxygen gasses.
    If the channels for hydrogen and oxygen (in this case channels 3 and 4) are both open, the program immediately shuts the valves and gives an error warning the user and ends the program.
    As long as this is not the case, the program proceeds to loop back to setting the values for channel 1.

    \begin{landscape}
        \begin{figure}
            \begin{center}
            \includegraphics*[width=1.5\textwidth,keepaspectratio,viewport=0 0 1052 506]{Gas_Controllerd.png}
            \end{center}
            \caption{MFC Controller program}
            \label{fig:mfcvi}
        \end{figure}

        \begin{figure}\ContinuedFloat
            \begin{center}
            \includegraphics*[width=1.5\textwidth,keepaspectratio,viewport=1052 0 2104 506]{Gas_Controllerd.png}
            \end{center}
            \caption[]{MFC Controller program (cont.)}
        \end{figure}

        \begin{figure}\ContinuedFloat
            \begin{center}
            \includegraphics*[width=1.5\textwidth,keepaspectratio,viewport=2104 0 3156 506]{Gas_Controllerd.png}
            \end{center}
            \caption[]{MFC Controller program (cont.)}
        \end{figure}

        \begin{figure}\ContinuedFloat
            \begin{center}
            \includegraphics*[width=1.5\textwidth,keepaspectratio,viewport=3156 0 4208 506]{Gas_Controllerd.png}
            \end{center}
            \caption[]{MFC Controller program (cont.)}
        \end{figure}

        \begin{figure}\ContinuedFloat
            \begin{center}
            \includegraphics*[width=1.5\textwidth,keepaspectratio,viewport=4208 0 5260 506]{Gas_Controllerd.png}
            \end{center}
            \caption[]{MFC Controller program (cont.)}
        \end{figure}

        \begin{figure}\ContinuedFloat
            \begin{center}
            \includegraphics*[width=1.5\textwidth,keepaspectratio,viewport=5260 0 6312 506]{Gas_Controllerd.png}
            \end{center}
            \caption[]{MFC Controller program (cont.)}
        \end{figure}%MFC Controller VI Images
    \end{landscape}

    The Gas Scheduler program uses the Gas Controller to automate the switching of gasses based on the instructions provided in a text file.
    The text file, referred to as the gas schedule, is a nine column tab delimited file where the first column is the amount in seconds for which that gas flow is to be run, followed by the flows for each channel.
    The program sequentially goes though the rows until it encounters a negative time, at which point it ends, leaving that row's gas flows running.
    The program also additionally records the measured flows to a \gls{csv} file for manual inspection later.

    Figure \ref{fig:gasschedvi} gives the block diagram of the program.
    Before the start of the program, the path to the gas schedule must be specified.
    At the start of the program, the location to record measured gas flows is requested and setup as a \gls{csv} file type with a header, variables to count the row of the program and time elapsed are initialized and communications variables are passed to sub-VIs.
    After this, the program enters a while loop.
    This while loop is ended if an error occurs, if the stop button is pressed, or if the amount of time in the stage (given by the first column in the gas schedule) is less than zero.

    Once the start button is pushed, a timer is started to measure the amount of time elapsed in the stage.
    If the time has elapsed the amount prescribed (starting with zero seconds), the program reads the gas schedule, increments the stage number, and passes the flow rates to a gas controller sub-VI, after which the VI loops.
    If time has not elapsed, the program records the flows based on a second elapsed time timer to limit the total number of recordings.
    After a program end condition is met, the communications channels are closed.

    \begin{landscape}%MFC Scheduler images
        \begin{figure}
            \begin{center}
            \includegraphics*[width=1.5\textwidth,keepaspectratio]{Gas_Schedulerd.png}
            \end{center}
            \caption{Gas Scheduler program}
            \label{fig:gasschedvi}
        \end{figure}
    \end{landscape}

\section{Temperature and \po2{} Measurements}
    \label{app:tgavi}
    The purpose of this LabVIEW program is to periodically measure the temperature from a Eurotherm process controller and calculate \po2{} based on voltage from a Keithley multimeter for the \gls{tga} system.
    The VI is available on GitHub under the repository ``eurotherm-keithley-measure'' and is saved for LabVIEW 2013 and later.
    This work is the continuation and development of programs previously written by others.

    \subsection{Hardware and Other Libraries}
    Temperature measurements are taken directly from a Eurotherm 2408 controller with optional serial communications module.
    A driver package for all 2400 series Eurotherm controllers is available from the \gls{ni} website.
    Serial communications again occur by VISA and require the \gls{ni} VISA drivers.

    \po2{} was calculated from voltage measurements taken of a \gls{ysz} sensor by a Keithley 2000 digital multimeter.
    Communications were established with the Keithley via GPIB (IEEE-488) and thus require the NI-488 driver from \gls{ni}.
    Additionally, drivers for the Keithley 2000 are available online from \gls{ni} and are required for the VI.

    \subsection{Program Overview}
    A copy of the block diagram is given in Figure \ref{fig:tempvi}.
    When started, the program asks for the location to save the measurements.
    The program then creates the \gls{csv} file time in seconds, the voltage read, the calculated log \po2{}, and the temperature from the controller.
    Communications variables, such as baud rate and channels, are sent the appropriate sub-VIs to begin communications to the device and a timer is started.
    The program backs up the Keithley's current settings, resets the device and prepares it to measure voltage on one channel, while displaying ``Running'' on the front panel display.
    After the program is complete, the Keithley is reset once again and its original settings are restored before communications are closed to it and the Eurotherm.

    The rest of the program occurs in a while loop.
    The loop is delayed a fixed amount of time, set by the Measurement Period.
    If the timer functionality is turned on, the loop repeats for a fixed amount of time based on the Length of Run variable, after which a stop signal is sent.
    During each loop, the process value (temperature) of the Eurotherm is measured and any alarms or flags are checked for.
    If any occur, they are passed to the error handler and the program stops.
    The voltage of the specified channel of the Keithley is measured and the log(\po2{}) is calculated from it.
    These values are then displayed, plotted, saved to the \gls{csv} file.

    \begin{landscape}%Temp and o2 measure program images
        \begin{figure}
            \begin{center}
            \includegraphics*[width=1.5\textwidth,keepaspectratio]{Temp_&_pO2_Measurementsd.png}
            \end{center}
            \caption{Temperature and \po2{} measurement program}
            \label{fig:tempvi}
        \end{figure}
    \end{landscape}
