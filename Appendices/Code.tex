% !TEX root = ../mainthesis.tex

%Appendix -- January 2015
%\appendix
%\renewcommand{\thechapter}{B}
%\renewcommand{\chaptername}{Appendix}

\chapter{Code}

\section{Arduino PID Relay Furnace Controller with Serial
Connectivity}\label{arduino-pid-relay-furnace-controller-with-serial-connectivity}
\label{app:PID}

The purpose of this Arduino sketch is to power a heating furnace
(controlled by a relay), with PID controls. In addition serial
communications are added to monitor, log, and interact with the
controller.

\subsection{Hardware and other
libraries}\label{hardware-and-other-libraries}

This was built to be used on a Arduino Uno R3, but should be easily run
on a variety of different boards. For temperature sensing, an
\href{https://www.adafruit.com/product/3263}{Adafruit MAX31856} is used
with the \href{https://github.com/adafruit/Adafruit_MAX31856}{library}
from them. Also used is the PID controller
\href{https://github.com/br3ttb/Arduino-PID-Library}{library by br3ttb}.
All I have really done is put the two together with the ability to
communicate over serial.

\subsection{Variables}\label{variables}

Here are a list of variables for things you may want to change based on
your setup.

\begin{itemize}
\item
  RelayPin: The physical pin your relay is connected to.
\item
  Adafruit\_MAX31856(CS, DI, DO, CLK): Pins for your MAX31856
\item
  WindowSize: Length of on/off cycle, needed to account for AC power
\item
  printdelay: Delay time in milliseconds for printout on serial
\item
  MaxOP: Maximum operating power, used to extend life of elements
\item
  myPID(\&Input, \&Output, \&workingSet,P,I,D, DIRECT): PID need to be adjusted for your setup.
  If inverse behavior is experienced (e.g. heats when it should cool) switch DIRECT.
\item
  workingSet and Setpoint: Inital setpoint for power on
\item
  ramprate: Sets a ramp rate for the working setpoint. Measured in
  milliseconds for 1C change.
\end{itemize}

\subsection{Serial communications}\label{serial-communications}

Baud rate is set to 9600. Output is tab deliminated of Setpoint, Working
Setpoint, Thermocouple Temperature, and Output Power. To change set
point, send a new number over serial and press return. You may have to
adjust your serial monitor to send \textbackslash n as end of line.

\lstinputlisting[language=C++,breaklines=true,postbreak=\mbox{\textcolor{red}{\(\hookrightarrow\)}\space}]{../../../Arduino/thermo_pid_serial/thermo_pid_serial.ino}
